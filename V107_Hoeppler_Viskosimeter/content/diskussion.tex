\section{Diskussion}
\label{sec:Diskussion}
\subsection{Vorbereitungen}
Im Vorfeld sollen die Glaskugeln ausgemessen werden, was mit einer Schieblehre erfolgt und ausreichend genau ist. Das Wiegen jedoch bleibt aus, da 
die Werte für das Gewicht der Beschreibung entnommen werden. Dies ist insofern relevant, als dass die Versuchsapparatur schon lange verwendet wird
und das tatsächliche Gewicht abweichen kann. An beiden Kugeln befinden sich abgesplitterte Stellen, welche nicht nur das Gewicht reduzieren, sondern
auch Turbulenzen in der Strömung auslösen könnten.
Da die Glaskugeln nur marginal kleiner sind als die Röhre, kann die Abgenutzheit der Versuchsaurüstung durchaus messbare Abweichungen verursachen im Vergleich
zum Ursprungszustand.
Nichtsdestotrotz funktionieren alle Utensilien einwandfrei und es lässt sich sauber arbeiten. Einzig die Stoppuhren mit einer eher unintuitiven Oberfläche und einer
leicht ausgelösten Weckfunktion sind fragwürdige Messtechnik und zuverlässig durch ein Smartphone auszutauschen.

\subsection{Fallzeiten und Viskosität}
Bis auf die zu überbrückende Dauer des verlangsamten freien Falls gibt es bei der Messung keine Schwierigkeiten. Durch die unterschiedliche Dichter der beiden
Kugeln von etwa $9\%$ ergeben sich unterschiedliche Werte für die Viskosität bei Raumtemperatur. Dies ist wahrscheinlich auf den Herstellungsprozess zurückzuführen.
Die kleinen Splitterstellen können den Unterschied allein nicht erklären.

Die Literaturwerte der Viskosität für die große Kugel liegen etwa $29\%$ auseinander, behalten aber im Verlauf ihren relativen Abstand bei. Der erhöhte Wert $\eta_{gr}$ bedeutet, dass das Wasser im Vergleich
"zäher", der Fall also langsamer ist. Denkbare Erklärungen sind übersehende Verunreinigungen und die eben genannten Splitterstellen.

\subsection{Reynoldszahl und Andrade'sche Gleichung}
Turbulenzen lassen sich zwar nicht ausschließen, jedoch liegen die ermittelten Reynoldszahlen weit unter der kritischen Grenze von $R_{crit} \approx 2300$.
Die ausgebildeten Strömungen sind also äußerst \textbf{stabil} und \textbf{laminar}. Gut erkennbar ist der vielfache Anstieg von $14$ auf $73$ entsprechend der geringsten und höchsten Temperatur.
Dies ist plausibel, da die Viskosität des Wassers mit steigender Temperatur fällt, es also "flüssiger" wird. Bei höheren Strömungsgeschwindigkeiten werden Turbulenzen immer wahrscheinlicher.

Die Temperaturabhängigkeit wird außerdem in der Andrade'schen Gleichung festgehalten. In Abbildung \ref{fig:andrade} sind die Messwerte linearisiert, wodurch die Achsen jedoch ihre Aussagekraft verlieren.
Dennoch ist der exponentielle Zusammenhang deutlich sichtbar.

\subsection{Fazit}
Der Versuch ist gradlinig und gut umsetzbar. Mögliche Verbesserungen beschränken sich auf die Stoppuhren und die Dauer der Messungen. Bei Wartezeiten von mehr als 60 Sekunden pro Messung
kann die Konzentration nach ein paar Stunden durchaus nachlassen, sodass man den ein oder anderen Durchlauf verpasst.
In diesem Experiment wird auch die Viskosität mit ihrer Temperaturabhängigkeit in Zusammenhang gebracht und intuitiv verständlich gemacht.