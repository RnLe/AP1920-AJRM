\section{Durchführung}
\label{sec:Durchführung}

Direkt zu Anfang werden die Eckdaten der zu verwendenden Glaskugeln aufgenommen. Hierbei handelt es sich um zwei Stück, 
die einen geringfügigen Unterschied im Radius aufweisen. 
Die Masse wird notiert und der Durchmesser mithilfe einer Schiebelehre gemessen. 
Daraus lässt sich bei der Auswertung die Dichte bestimmen. 

Im Viskosimeter sollte im äußeren Zylinder bereits Wasser enthalten sein, das Raumtemperatur haben sollte, sodass keine 
großen Temperaturschwankungen bei der ersten Messung auftreten. Ist dies noch nicht getan, wird dies an dieser Stelle nachgeholt. 

Das zu untersuchende Fluid ist Wasser, genauer bidestilliertes Wasser. Es handelt sich um besonders reines Wasser, welches 
zweifach destilliert worden ist. 

Das Viskosimeter hat am inneren Zylinder oben und unten jeweils einen Schraubverschluss. 
Der obere samt vorhandener Stopfen wird geöffnet beziehungsweise entfernt und das Rohr mit dem reinen Wasser befüllt. 
Da Luftbläschen für die Messung im Folgeschritt entfernt werden, kann bereits an dieser Stelle darauf geachtet werden, 
möglichst wenig Blasen beim Befüllen entstehen zu lassen. 
Das Beseitigen der Bläschen geschieht mithilfe eines Glasstabs, der am unteren Ende zu einer kleinen Kreisscheibe ausgedehnt ist. 
Die Luftblasen werden nun von unten mit dem besagten Stab zur Öffnung getragen, bis das Wasser blasenfrei ist. 

Ist dies geschehen, wird die kleinere der beiden Kugeln in das Wasser gelassen und der Deckel so verschlossen, dass keine 
Luft in der Röhre enthalten ist. 
Nun wird bei Raumtemperatur, die am Thermometer des Thermostats abzulesen ist, die Fallzeit gemessen, die die sich mit 
konstanter Geschwindigkeit fortbewegende Kugel benötigt, um von der ersten der drei Markierungen bis zur letzten zu gelangen. 
Die erste Markierung ist so gewählt, dass bei Erreichen dieser sich das Gleichgewicht der Auftriebs-, Gewichts- und 
Reibungskraft eingestellt hat und die Kugel nicht weiter beschleunigt. 
Die Distanz zwischen der ersten und dritten Markierung beträgt $x=\SI{100}{\milli\m}$. 
Ist die Kugel unten angekommen, wird das Viskosimeter um $\SI{180}{\degree}$ gedreht und die Messung wiederholt. 
Insgesamt sollen zehn Messzeiten aufgenommen werden. 

Diese Messung wird genau so nochmal mit der größeren Kugel wiederholt. 
Dafür muss der untere Verschluss geöffnet und die Kugel herausgeholt werden, was unvermeidlich mit einem Auslaufen der inneren 
Röhre einhergeht. 
Diese wird nun wieder aufgefüllt, auf dieselbe Art und Weise blasenfrei gemacht und die zweite Kugel wird in das Gefäß gelassen. 
Nach weiteren zehn Messungen wird mithilfe des Thermostats die Temperatur des umgebenden Wassers schrittweise erhöht. 
Bei einer Erhöhung der Temperatur muss jeweils eine Weile gewartet werden, um sicherzugehen, dass das temperierende Wasser 
und das reine Wasser dieselbe Temperatur angenommen haben. 
In Summe sollen jeweils zehn Fallzeiten für zehn verschiedene Temperaturen des bidestillierten Wassers gemessen werden. 
Hierfür wird ausschließlich die große Glaskugel verwendet. 
Beim Erhitzen sollte maximal eine Temperatur von $\SI{70}{\celsius}$ erreicht werden. 
Außerdem empfiehlt es sich, in Abständen die Entlüftungsschraube zu lockern, um dem entstehenden Druck entgegen zu wirken. 
