\section{Zielsetzung}
\label{sec:Zielsetzung}

Motiviert wird das Experiment mit der Bestätigung der Diskretisierung der Energieniveaus innerhalb eines Atoms, die Teil der Atomspektroskopie ist.  
Hier werden exemplarisch Quecksibler-Atome betrachtet -- im Folgenden sei Quecksilber durch sein Elementsymbol Hg im Periodensystem abgekürzt. 
Zum einen soll die Differenz zwischen dem ersten angeregten Zustand und dem Grundzustand experimentell bestimmt werden, 
sowie die gesamte Ionisierungsenergie, die nötig ist, um ein Elektron ganz aus dem Coulomb-Feld des Kerns zu entfernen. 
Außerdem kann am Ende des Experiments eine Aussage über die Energieverteilung der Elektronen gemacht werden, die 
aus der Glühkathode austreten und durch Stöße Energie an die Hg-Atome übertragen. 