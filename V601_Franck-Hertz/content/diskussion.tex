\section{Diskussion}
\label{sec:Diskussion}

Da kein Literaturwert für das Kontaktpotential vorliegt, lässt sich nur schwerlich feststellen, wie groß die 
experimentelle Abweichung ist. 
Wie bereits aus den Abbildungen ersichtlich und wie in der Auswertung schon erwähnt, ist das Ablesen des Kontaktpotentials 
nur mit großer Unsicherheit möglich, da die Streuungen recht groß sind und der Peak bei einer Spannung eine vergleichsweise 
große Breite hat. 
Dass der Peak nicht infinitesimal dünn ist, kann mit der Dirac--Fermi-Statistik begründet werden. 
Diese gibt die Verteilung der Geschwindigkeiten -- präziser gesagt: die Wahrscheinlichkeit, dass ein Elektron eine gewisse 
Energie in Form von kinetischer Energie hat -- der Elektronen an, was impliziert, dass die Elektronen bereits 
beim Austritt aus dem Glühdraht keine einheitliche Geschwindigkeit in Richtung der Auffängerelektrode besitzen. 

Einen weiteren Einfluss könnte die mittlere freie Weglänge sein. 
Diese wird in Kapitel \ref{sub:Vorbereitung} ausgerechnet und in \ref{tab:Weglänge} dargestellt. 
Der Vergleich mit dem Abstand zwischen Glühkathode und Beschleunigungsanode ergibt jedoch, dass dieser Einfluss vernachlässigbar klein sein sollte, 
da die beiden Größen $a$ und $\bar{w}$ in der gleichen Größenordnung liegen. 

Eine andere Möglichkeit für die starke Streuung wäre, dass das Ablesen der Messpunkte aus dem Diagramm recht mühsam ist 
und leicht kleine Abweichungen auftreten können, vor allem wenn die Differenz zwischen zwei benachbarten Messwerten betrachtet wird. 
\\

Zum Vergleich werden aus \cite{kohlrausch} die Anregungsenergien in die drei verschiedenen Zustände herangezogen 
und deren Mittelwert gebildet, da hier im Experiment keine hinreichend große Auflösung bewerkstelligt werden kann, 
als dass zwischen den Feinstrukturenergien unterschieden werden könnte. 
\begin{table}
    \centering
    \caption{Zusammenfassung der Literaturwerte für die erste Anregungsenergie von Quecksilber\cite[460]{kohlrausch}.}
    \label{tab:Lit}
    \begin{tabular}{c c c}
        \toprule
        Grundzustand & angeregter Zustand & Anregungsenergie $\Delta E\,/\,\si{\electronvolt}$ \\
        \midrule
                    & $\ce{^3P0}$ & 4.67 \\
        $\ce{^1S0}$ & $\ce{^3P1}$ & 4.88 \\
                    & $\ce{^3P2}$ & 5.46 \\
        \midrule
        Mittelwert: & \multicolumn{2}{c}{$\SI{5.00}{\electronvolt}$} \\
        \bottomrule
    \end{tabular}
\end{table}

Der experimentelle Wert von $\SI{5.0+-0.1}{\electronvolt}$ bestätigt somit die theoretische Erwartung, sowie die Annahme, 
dass weitere elastische Stöße der Elektronen mit den Hg-Atomen keine signifikanten Veränderungen am Messwert 
zur Folge haben.

Der Vollständigkeit halber sei hier nochmal der Literaturwert der Ionisierungsenergie von Quecksilber von 
$\SI{10.43}{\electronvolt}$ aufgeführt, die hier nicht experimentell ermittelt worden ist. 