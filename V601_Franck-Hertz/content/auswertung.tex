\section{Auswertung}
\label{sec:Auswertung}

\subsection{Vorbereitung}

Die mittlere freie Weglänge wird über Gleichung \eqref{eqn:Weglänge} berechnet und mit dem Abstand zwischen Beschleunigungsanode 
und Glühkathode $a\approx \SI{1}{\centi\meter}$ verglichen. In Kapitel \ref{sub:Druck} wird ein ideales Verhältnis von etwa $1000$ bis $4000$ vorgeschlagen. 
Die verwendete Temperatur für die Aufnahme der Franck--Hertz-Kurve ist aus Abbildung REFERENZ ersichtlich. 
Die Raumtemperatur wird auf $\SI{295}{\kelvin}$ abgeschätzt.
Die entsprechenden Werte sind in Tabelle \ref{tab:Weglänge} aufgeführt. 

\begin{table}
    \centering
    \caption{Vergleich des Abstands Glühkathode-Beschleunigungsanode und der mittleren freien Weglänge}
    \label{tab:Weglänge}
    \begin{tabular}
        \toprule{l c l c}
        $T\,/\,\si{\kelvin}$ & $p\,/\,\si{\milli\bar}$ & $\bar{w}\,/\,\si{\milli\meter}$ & $a\,/\,\bar{w}$ \\
        \midrule
        295    & $\num{4.15e-3}$ &  6.99 & 1.43 \\
        446.15 & 11.1            & 0.261 & 38.3 \\
        \bottomrule
    \end{tabular}
\end{table}

Die mittlere freie Weglänge ist also bei beiden Temperaturen bei weitem größer, wie es als ideal angesehen wird. 