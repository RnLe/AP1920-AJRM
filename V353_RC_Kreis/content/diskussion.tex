\section{Diskussion}
\label{sec:Diskussion}

\subsection{Anregung durch eine Rechteckspannung}

muss ich mir noch überlegen, wieso tau KLEiner ist als der theoretische wert... wenn er größer wäre, könnte man das über 
zusätzliche verluste an den kabeln erklären, aber so... ich schaue mal... 

\subsection{Integrieren der Generatorspannung}

Bei allen drei Spannungen kann sehr gut die Phasenverschiebung um $\sfrac{\pi}{2}$ beobachtet werden, die wenn überhaupt 
nur minimal kleiner ist. 
Dies ist kohärent mit der Theorie, da nur für eine unendlich große Frequenz eine solche Phasendifferenz bewerkstelligt 
werden kann. Eine im mathematischen Sinne unendlich große Frequenz ist selbstredend in der Praxis nur näherungsweise zu 
realisieren. 

Bereits bei der Durchführung fällt auf, dass die Skala der Kondensatorspannung stark vergrößert werden muss, um Amplituden 
vergleichbarer Größenordnung auf dem Oszilloskop beobachten zu können. 
Dies rührt daher, dass bei hoher Frequenz ein vergleichsweise sehr kleiner Anteil der Spannung beim Kondensator ankommt, 
wie in der Theorie ausführlich erklärt wird. 

Somit entspricht die integrierte Sinusschwingung in \ref{fig:int_sinus} den Erwartungen.
Ein wenig komplizierter wird es bei den anderen beiden Schwingungen, logische Erklärungen lassen sich dennoch dafür finden: 
Hervorgehoben sei nochmal die Tatsache, dass die Spannung aufgrund der hohen Frequenz nur im geringen Maße zum Kondensator 
durchkommt: Der Kondensator lädt sich langsam auf, aber noch lange, bevor er einen vollständig geladenen Zustand erreichen 
kann, springt die treibende Spannung wieder um, sodass der Kondensator langsam wieder die Ladung verliert, die er 
eben erst aufgebaut hat. 

Unter Berücksichtigung dieses Vorgangs können die unterschiedlichen Spannungsverläufe der Rechtecks- und Dreiecksschwingungen 
in \ref{fig:rechtdrei} \subref{fig:int_rechteck} und \subref{fig:int_dreieck} erklärt werden:
Der Kondensator würde sich in \subref{fig:int_rechteck} vollständig aufladen und so die gleiche Struktur wie die generierte 
Rechtecksspannung aufweisen, wenn er genügend Zeit pro Periode hätte. Da er diese nicht hat, ist nur der bloße lineare
Anstieg der Spannung zu sehen, und es entsteht ein dreieckiger Schwingungsverlauf beim Kondensator. 
Genauso bei der Dreiecksschwingung: Der Kondensator reagiert nur zeitverzögert und im geringen Maße auf die linear verlaufende 
Dreiecksspannung, sodass die Kondensatorspannung \glqq abgerundete Ecken \grqq{} dort hat, wo die Dreiecksschwingung nicht 
differenzierbar ist. 

Die möglicherweise vorerst überraschende Unterschiedlichkeit der Spannungsverläufe lassen sich somit problemlos mit den 
theoretischen Grundlagen vereinbaren. 