\section{Diskussion}
\label{sec:Diskussion}

\subsection{Anregung durch eine Rechteckspannung}

Der theoretisch zu erwartende Wert für die Zeitkonstante ist $\tau _\text{theo}=\SI{1.405(56)}{\milli\second}$, der aus dem 
ersten Teil des Experiments ermittelte ${\tau _\text{exp}=\SI{0.82(1)}{\milli\second}}$. 
Da der experimentelle Wert deutlich außerhalb des Fehlerintervalls von $\tau _\text{theo}$ liegt, kann an dieser Stelle 
nicht von einer statistischen Abweichung ausgegangen werden. 
Der Fehler kann auch nicht an nicht berücksichtigten Innenwiderständen der Geräte oder an ohmschen Verlusten der Kabel 
liegen. 
Wahrscheinlicher ist ein systematischer Messfehler, der sich im Nachhinein nur schwerlich zu finden lässt. 
Möglichkeiten könnten überdrehte -- und deshalb keine richtig skalierenden -- Knöpfe am Oszilloskop sein, sodass alle 
abgelesenen Spannungswerte um einen Faktor verschieden von der tatsächlich angelegten Spannung sind. 

\subsection{Integrieren der Generatorspannung}

Bei allen drei Spannungen kann sehr gut die Phasenverschiebung um $\sfrac{\pi}{2}$ beobachtet werden, die wenn überhaupt 
nur minimal kleiner ist. 
Dies ist konsistent mit der Theorie, da nur für eine unendlich große Frequenz eine solche Phasendifferenz bewerkstelligt 
werden kann. Eine im mathematischen Sinne unendlich große Frequenz ist selbstredend in der Praxis nur näherungsweise zu 
realisieren. 

Bereits bei der Durchführung fällt auf, dass die Skala der Kondensatorspannung stark vergrößert werden muss, um Amplituden 
vergleichbarer Größenordnung auf dem Oszilloskop beobachten zu können. 
Dies rührt daher, dass bei hoher Frequenz ein vergleichsweise sehr kleiner Anteil der Spannung beim Kondensator ankommt, 
wie in der Theorie ausführlich erklärt wird. 

\begin{figure}
\centering
\begin{subfigure}{0.48\textwidth}
    \centering
    \includegraphics[height=5cm]{plots/erwart_int1.pdf}
    \label{fig:erw1}
\end{subfigure}
\begin{subfigure}{0.48\textwidth}
    \centering
    \includegraphics[height=5cm]{plots/erwart_int2.pdf}
    \label{fig:erw2}
\end{subfigure}
\caption{Die Spannungskurven und ihre zugehörigen Integrale.}
\label{fig:int}
\end{figure}

Um die Bilder vom Oszilloskop gut vergleichen zu können, sind in \ref{fig:int}  
die jeweiligen Spannungsarten mit zugehörigen Funktionsgraphen abgebildet, deren Ableitung die Spannungskurve ergibt. 
Der Übersicht halber wird an dieser Stelle auf eine explizite Darstellung der Funktionsdefinitionen von Dreiecks- und Rechteckskurven 
und deren Integrationen verzichtet. Sie bestehen aus trivialen linearen stückweise stetigen beziehungsweise differenzierbaren 
Funktionen. 

Wie durch einen Vergleich der Abbildungen klar ersichtlich ist, entsprechen die theoretischen Überlegungen bezüglich der 
integrierenden Eigenschaft eines RC-Kreises den experimentellen Beobachtungen. 