\section{Durchführung}
\label{sec:Durchführung}

\subsection{Einleitung}
Ziel der Durchführung ist es, die charakteristischen Eigenschaften eines RC-Gliedes zu messen und mit Hilfe eines Oszilloskops
zu visualisieren. Hierbei werden neben dem Auf- und Entladevorgang des Kondensators auch die frequenzabhängigen Amplituden und
Phasenverschiebungen gemessen.

\subsection{Allgemeiner Aufbau}
Das zu messende Schaltelement ist eine Serienschaltung aus einem Widerstand und einem Kondensator.
Als Messtechnik dient ein Oszilloskop in Verbund mit einem Sinusspannungsgenerator, über den sowohl die Spannung, wie Frequenz 
als auch die Art des erzeugten Signals eingestellt werden können. Mögliche Einstellungen sind Sinus-, Rechteck-, Dreieck- und gepulste Rechteckspannung.

Der Ausgang des Generators wird über ein T-Stück zum einen mit dem RC-Kreis verbunden, zum anderen auf einen Kanal des Oszilloskops gelegt.
Der andere Kanal wird parallel zum Kondensator geschaltet. Für genaue Messungen und statische Bilder wird der Triggerausgang des Generators 
mit dem externen Triggereingang des Oszilloskops verbunden, wodurch insgesamt vier Kabel notwendig sind.

\subsection{Messungen}
\subsubsection{Zeitkonstante}
Um die Zeitkonstante des RC-Kreises bestimmen zu können, ist nur die Darstellung des parallelgeschalteten Kanals notwendig.
Hierfür wird eine Rechteckspannung auf~$\approx \SI{250}{\hertz}$ eingestellt und die Spannungskurve am Oszilloskop abgelesen.
Dabei zeichnen sich die charakteristischen Auflade- und Entladekurven des Kondensators ab, über dessen zeitabhängigen Amplituden dann die Zeitkonstante bestimmt werden kann.

\subsubsection{Frequenzabhängige Messungen}
Es werden sowohl die Größen der Amplituden, als auch die Phasenunterschiede der Kondensatorspannung in Abhängigkeit von der Frequenz gemessen.
Der zweite Kanal gibt die Generatorspannung wieder und dient als visueller Vergleich zu der Spannung des RC-Kreises auf dem ersten Kanal.
Die Messreihe wird in circa 20 Messwerte auf je eine von drei Größenordnungen aufgeteilt: beginnend bei etwa $\SI{15}{\hertz}$ bis $\SI{150}{\hertz}$ über $\SI{150}{\hertz}$ bis $\SI{600}{\hertz}$, bis zum Bereich
$\SI{600}{\hertz}$ bis $\SI{2100}{\hertz}$.
%Den folgenden Abschnitt mit unserer Kästchen-Methode finde ich etwas kompliziert erklärt. Finde ich aber nicht schlimm, ich weiß ja, was gemeint ist. 
%Wenn du noch Zeit und Nerv hast, kannst du das ja vielleicht noch umformulieren, wenn du siehst, was ich meine mit kompliziert. 
%Ist denke ich aber auch nicht so wichtig. Also mach wie du denkst. :)
Um sich eine Kalibrierung zu sparen, reicht es, wenn die Kästchen auf dem Oszilloskop abgelesen werden, um Amplituden- und Phasenverschiebungen
bezüglich der Generatorspannung abzulesen. Für die Phasenverschiebung wird entsprechend pro Messung die Anzahl der Kästchen für eine halbe Periode
abgelesen und als Nenner für die Periodendauer in Kästchen der RC-Spannung verwendet. Hieraus ergibt sich ein Verhältnis in $\symup{\pi}$ und ist damit bereits die zu bestimmende Phasenverschiebung.
Für die Amplituden wird zu Beginn einmalig die der Generatorspannung abgelesen und im Anschluss die Amplituden der RC-Spannung
als Verhältnis dagegengemessen. Wichtig hierbei ist, dass die Auflösung der Spannung für beide Kanäle während aller Messungen
nicht verändert wird.

\subsubsection{Integratorspannung}
Um zu zeigen, dass ein RC-Kreis unter bestimmten Bedingungen auch als Integrator dienen kann, wird für die Sinus- und Dreieckspannung
eine Frequenz von $\approx \SI{5600}{\hertz}$ eingestellt und die Spannungen am Oszilloskop dargestellt. Wenn das Oszilloskop keine Screenshot-Funktion besitzt,
wird das Bild abfotografiert.
Für die Rechteckspannung empfiehlt sich eine Frequenz von etwa $\SI{650}{\hertz}$, um zu sehen, dass eine Dreieckspannung daraus entsteht, und somit die Ableitung
der Ursprungsspannung darstellt.