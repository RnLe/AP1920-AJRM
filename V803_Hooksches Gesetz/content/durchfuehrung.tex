\section{Beschreibung und Durchführung}
\label{sec:Durchführung}
An einem Stativ wird eine Spiralfeder aufgehängt, an deren unteren Ende ein Faden befestigt ist.
Dieser wird mit einem elektronischen Kraftmesser verbunden. 
Der Faden wird gerade soweit unter Spannung gebracht, dass kein Kraftaufwand dafür nötig ist, 
das Messgerät demnach keine Kraft misst.
Ein Lineal wird so angebracht, dass die Auslenkung der Feder in Zentimetern abgelesen werden kann. \\
Nun wird für zehn verschiedene Auslenkungen die benötigte Kraft am Kraftmesser abgelesen. 
Die Federkonstante D wird über den Quotienten $\frac{\symup{F}}{\symup{\triangle x}}$ berechnet. 