\section{Auswertung}
\label{sec:Auswertung}

\subsection{Arithmetisches Mittel und Messunsicherheit}

Das arithmetische Mittel einer Größe $x$, also das dem unbekannten wahren Wert am nächsten kommende Ergebnis aus einer Messung mit $N$ Messwerten, berechnet sich über 
\begin{equation}
    \bar{x}=\frac{1}{N} \sum_i x_i\,.
    \label{eqn:mittel}
\end{equation}
Die Messunsicherheit beläuft sich auf 
\begin{equation}
    s_x=\sqrt{\frac{1}{N-1}\sum_i (\bar{x}-x_i)^2}
    \label{eqn:falsch}
\end{equation}

\subsection{Wellenlängenbestimmung des Lasers}

Die Werte der ersten Messung sind in Tabelle \ref{tab:lambda} dargestellt. 
Hierbei sind ebenfalls die mit der Übersetzung skalierten Werte für den Weglängenunterschied aufgeführt, 
sowie die sich jeweils aus einem Wertepaar ergebenden Werte für die Wellenlänge. 

\begin{table}
    \centering
    \caption{Messwerte zur Bestimmung der Wellenlänge, sowie diese selbst.}
    \label{tab:lambda}
    \begin{tabular}{c c c c}
        \toprule
        $\Delta d_\text{mess}\,/\,\si{\milli\meter}$ & $\Delta d\,/\,\si{\milli\meter}$ & $z$ & $\lambda \,/\,\si{\nano\meter}$ \\
        \midrule
        3.72 & 0.741 & 2256 & 657 \\
        1.93 & 0.385 & 2133 & 361 \\
        4.75 & 0.947 & 3000 & 631 \\
        1.75 & 0.349 & 1135 & 615 \\
        3.00 & 0.598 & 1831 & 653 \\        
        \bottomrule
    \end{tabular}
\end{table}

Durch Mittelwertbildung und eine entsprechende Berechnung der Messunsicherheit mit \eqref{eqn:mittel} und \eqref{eqn:falsch} 
ergibt sich das experimentelle Ergebnis von 
\begin{equation}
    \lambda_\text{exp,1} = \SI{583+-126}{\nano\meter} \,. 
\end{equation}
Auffällig ist der zweite Messwert, der eine sehr viel geringere Wellenlänge ergibt. Wird dieser Wert nicht in die Berechnung miteinbezogen, 
beläuft sich das experimentelle Ergebnis auf
\begin{equation}
    \lambda_\text{exp,2} = \SI{639+-20}{\nano\meter} \,.
\end{equation}
Der theoretisch zu erwartende Wert ist 
\begin{equation}
    \lambda_\text{theo}=\SI{635}{\nano\meter}\,.
\end{equation}

\subsection{Bestimmung des Brechindexes}

Der Druckunterschied zur Umgebung beläuft sich auf etwa $\Delta p=p_0-p=\SI{0.8}{\bar}$, die Länge der Messzelle ist $b=\SI{50}{\milli\meter}$. 
Die gemessene Anzahl der Interferenzstreifen sowie der sich mit Gleichung \eqref{eqn:Brechi} daraus ergende Brechindex für die Umgebung sind 
in Tabelle \ref{tab:Brech} aufgeführt. 

\begin{table}
    \centering
    \caption{Die Messwerte zur Bestimmung des Brechindexes.}
    \label{tab:Brech}
    \begin{tabular}{c c}
        \toprule
        $z$ & $n$ \\
        \midrule
        33 & 1.00027 \\
        33 & 1.00027 \\
        32 & 1.00026 \\
        32 & 1.00026 \\
        33 & 1.00027 \\
        \bottomrule
    \end{tabular}
\end{table}

Der experimentelle Wert dieser Messung ergibt sich (wohlgemerkt unter Aufrundung des Messfehlers) zu 
\begin{equation}
    n_\text{exp}=\num{1.00026+-0.00001}\,.
\end{equation}