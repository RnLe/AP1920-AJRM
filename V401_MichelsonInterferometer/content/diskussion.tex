\section{Diskussion}
\label{sec:Diskussion}

Bei der Messung zur Bestimmung der Wellenlänge zeigt das zweite Messwertepaar eine auffällig hohe Abweichung. 
Da an sich die Anzahl der Messwerte vergleichsweise gering ist, führt dies unter Vergleich von \eqref{eqn:bloed} zu einer sehr 
hohen experimentellen Unsicherheit von $22\,\%$. 
Geht hingegen der missglückte Messwert nicht mit ein -- betrachte hierzu \eqref{eqn:besser} --, reduziert sich die relative Abweichung auf den sehr viel geringeren Wert von $3\,\%$. 
Vor allem mit Blick auf den theoretisch zu erwartenden Wert \eqref{eqn:lam_theo}, zeigt das Auslassen des einen Messwerts 
eine deutlich zufriedenstellerende Übereinstimmung. 
Zwar liegt die Wellenlänge von $\lambda_\text{theo}=\SI{635}{\milli\meter}$ in dem Fehlerintervall von \eqref{€qn:bloed}; 
die Größe des Fehlerintervalls bleibt jedoch. 

Unter Vergleich der Messwertpaare in Tabelle \ref{tab:lambda} wird deutlich, dass die Weglängendifferenz des zweiten 
Messwerts sehr viel geringer ist, als erwartet werden würde. 
Da das Zählwerk voll automatisch die Anzahl der Interferenzstreifen liefert, die schlichtweg an der Anzeige abgelesen werden können, 
legt dies nahe, dass der Fehler beim Ablesen des Weglängenunterschieds zu finden ist. 
Der Versuch muss unter Verdunkelung der Umgebung durchgeführt werden, was hinsichtlich dieser Beobachtung zu einem erschwerten 
Ablesen der Messskala mit Handy-Taschenlampen zur Folge hat. 
Gut vorstellbar ist also dies als eine mögliche Fehlerquelle für den ausfallenden Messwert. 
Abgesehen von diesem deckt sich das Messergebnis in \eqref{eqn:besser} sehr gut mit dem in \eqref{eqn:lam_theo}, 
das Fehlerintervall ist gering und das theoretische Ergebnis liegt in der statistischen experimentellen Messunsicherheit. 

Der experimentell bestimmte Messwert für den Brechindex in \eqref{eqn:n_exp} hat eine bemerkenswert geringe Messunsicherheit 
(relative Abweichung von $0.001\,\%$)
und nähert sich gut an den theoretischen Wert in \eqref{eqn:n_theo}. 
Dieser liegt jedoch ganz knapp außerhalb des Fehlerintervalls. 
Als möglichen Grund für die Abweichung lassen sich die Näherungen anführen, die bei der Herleitung der Formel für den 
Brechindex in \eqref{eqn:Brechi} angestellt werden. 
Nicht beachtet werden Temperaturdifferenzen bei Druckverringerung, sowie generell die Abhängigkeit eines Brechindexes 
von der Temperatur fließen nicht in die Berechnung mit ein. 
Zusätzlich ist die Idee vom idealen Gas zwar ein gutes Modell für die Umgebungsluft, jedoch auch nur eine Näherung. 

Ein leichtes Abweichen des Theoriewerts vom experimentellen Ergebnisbereich ist somit gut mit der Anzahl an nähernden Vorüberlegungen vereinbar. 
