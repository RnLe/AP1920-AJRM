\section{Auswertung}
\label{sec:Auswertung}

\subsection{Vorbereitung: Berechnung der Theorie-Werte}

Es werden die drei Elemente Neodymium, Gadolinium und Dysprosium untersucht, von denen jeweils zwei als dreifach positiv geladene 
Ione mit drei zweifach negativ geladenen Sauerstoff-Ionen ein Molekül bilden. 
Ebenfalls haben alle drei Elemente zwei 6s-Elektronen und eine bis zur 5p-Schale. Nur hinsichtlich der Anzahl # der 4f-Elektronen unterscheiden sie sich. 
Dies ist auch der Fall, wenn sie im ionisierten Zustand sind: 
Jeweils zwei 6s-Elektronen und ein 4f-Elektron verschwindet; die Elektronen bis zur 5p-Schale bleiben unverändert. 
Zur besseren Übersicht werden die Zwischenergebnisse in Tabelle \ref{tab:Theo} zusammengefasst. 

Auf der 4f-Schale können sich maximal 14 Elektronen befinden. Dementsprechend wird gemäß den Hund'schen Regeln \ref{sub:wauwau} bei einer Elektronenzahl von inklusive acht 
die Quantenzahl des Gesamtdrehimpuls durch Addition $L+S$ berechnet, bei einer geringeren Zahl durch Subtraktion $L-S$. 

Mithilfe der Quantenzahlen wird der Landé-Faktor gemäß Gleichung \eqref{eqn:lande} berechnet. 

    \begin{table}
        \centering
        \caption{}
        \label{tab:Theo}
        \begin{tabular}{l r c c c c}
            \toprule
            & & & 
            neutral & 
            \multicolumn{4}{c}{ionisiert} \\
            \cmidrule(lr){4} \cmidrule(lr){5-8}
            Element &
            # 4f & 
            # 4f &
            $S$ & 
            $L$ & 
            $J$ \\
            \midrule
            Nd &  4 & 3 & $3\cdot\frac{+1}{2}=\frac{3}{2}$                       & $3+2+1=6$             & $\frac{9}{2}$  \\
            Gd &  8 & 7 & $7\cdot\frac{+1}{2}=\frac{7}{2}$                       & $3+2+1+0=6$           & $\frac{5}{2}$  \\
            Dy & 10 & 9 & $7\cdot\frac{+1}{2}+2\cdot\frac{-1}{2}=\frac{5}{2}$    & $3+2+3+2+1+0-1-2-3=5$ & $\frac{15}{2}$ \\
            \bottomrule
        \end{tabular}
    \end{table}

    \begin{table}
        \centering
        \caption{}
        \label{tab:}
        \begin{tabular}{l S[table-format=2.2] S[table-format=1.2]}
            \toprule
            Element & 
            $m\,/\,\si{\gram}$ & 
            $\rho\,/\,\si{\gram\per\cubic\centi\meter}$ \\
            \midrule
            Nd & \num{9.0}   & \num{7.24} \\
            Gd & \num{14.08} & \num{7.40} \\
            Dy & \num{14.38} & \num{7.8}  \\
            \bottomrule
        \end{tabular}
    \end{table}