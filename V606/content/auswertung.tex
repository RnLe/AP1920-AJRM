\section{Auswertung}
\label{sec:Auswertung}

\subsection{Vorbereitung: Berechnung der Theorie-Werte}

Es werden die drei Elemente Neodymium, Gadolinium und Dysprosium untersucht, von denen jeweils zwei als dreifach positiv geladene 
Ione mit drei zweifach negativ geladenen Sauerstoff-Ionen ein Molekül bilden. 
Ebenfalls haben alle drei Elemente zwei 6s-Elektronen und eine bis zur 5p-Schale. Nur hinsichtlich der Anzahl \# der 4f-Elektronen unterscheiden sie sich. 
Dies ist auch der Fall, wenn sie im ionisierten Zustand sind: 
Jeweils zwei 6s-Elektronen und ein 4f-Elektron verschwindet; die Elektronen bis zur 5p-Schale bleiben unverändert. 
Zur besseren Übersicht werden die Zwischenergebnisse in Tabelle \ref{tab:Theo1} zusammengefasst. 

Auf der 4f-Schale können sich maximal 14 Elektronen befinden. Dementsprechend wird gemäß den Hund'schen Regeln \ref{sub:wauwau} bei einer Elektronenzahl von inklusive acht 
die Quantenzahl des Gesamtdrehimpuls durch Addition $L+S$ berechnet, bei einer geringeren Zahl durch Subtraktion $L-S$. 

Mithilfe der Quantenzahlen wird der Landé-Faktor gemäß Gleichung \eqref{eqn:lande} berechnet. 

    \begin{table}
        \centering
        \caption{Bestimmung der Quantenzahlen.}
        \label{tab:Theo1}
        \begin{tabular}{l r c c c c c}
            \toprule
             & 
            neutral & 
            \multicolumn{4}{c}{ionisiert} \\
            \cmidrule(lr){2-2} \cmidrule(lr){3-7}
            Element &
            \# 4f & 
            \# 4f &
            $S$ & 
            $L$ & 
            $J$ &
            $g_J$ \\
            \midrule
            Nd &  4 & 3 & $3\cdot\sfrac{+1}{2}=\sfrac{3}{2}$                       & 6 & $\sfrac{9}{2}$  & 0.727 \\
            Gd &  8 & 7 & $7\cdot\sfrac{+1}{2}=\sfrac{7}{2}$                       & 0 & $\sfrac{-7}{2}$ & 2.400 \\
            Dy & 10 & 9 & $7\cdot\sfrac{+1}{2}+2\cdot\sfrac{-1}{2}=\frac{5}{2}$    & 5 & $\sfrac{15}{2}$ & 1.333 \\
            \bottomrule
        \end{tabular}
    \end{table}

Zur Berechnung des theoretisch zu erwartenden Werts der Suszeptibilität $\chi$ wird ebenfalls die Anzahl der atomaren magnetischen 
Dipolmomente pro Volumeneinheit benötigt. 
Hierfür wird die molare Masse zu 
\begin{equation*}
    M_\text{Molekül}=3M_\text{O}+2M_\text{SE}
\end{equation*}
bestimmt mit der molaren Masse des Sauerstoff und des jeweiligen Seltenen-Erd-Metalls. 
Je Molekül gibt es also zwei magnetische atomare Dipolmomente, deren Größe mithilfe der in Tabelle \ref{tab:Theo1} vorgegeben werden.

Die Stoffmenge der vorliegenden Proben ist 
\begin{equation*}
    n_\text{Molekül}=\frac{m}{M_\text{Molekül}}
\end{equation*}
mit der Masse $m$. 
Sie gibt die Anzahl der Teilchen in Mol an, die in der Probe enthalten sind. 
Dementsprechend berechnet sich die Anzahl der Moleküle pro Volumen durch Division der Stoffmenge mit dem Volumen. 
Multipliziert man dies mit zwei, erigbt sich die Anzahl der Mole der positiv geladenen Seltenen-Erd-Ionen:
\begin{equation*}
    n_\text{Ione}=2\cdot\frac{n_\text{Molekül}}{V_\text{Molekül}}=2\cdot \frac{n_\text{Molekül}\rho}{m}
\end{equation*}
Die Dichte $\rho$ steht im Experiment bereits zur Verfügung und muss nicht extra aufgenommen werden. 
Die Anzahl $N$ der Teilchen ohne Einheiten pro Volumeneinheit wird durch Multiplikation der Avogadro-Konstante $\symup{N}_\text{A}$ erreicht.
\begin{equation*}
    N=n_\text{Ione}\cdot \symup{N}_\text{A}=\frac{2\symup{N}_\text{A}\rho}{M_\text{Molekül}}
\end{equation*}
Die entsprechenden Daten dazu werden in Tabelle \ref{tab:Theo2} dargestellt. 
Die Werte für die Molare Masse sind \cite[610]{kohlrausch} entnommen.
Die des Sauerstoffs ist $M_\text{O}=\SI{15.9994}{\gram\per\mole}$

    \begin{table}
        \centering
        \caption{}
        \label{tab:Theo2}
        \begin{tabular}{l c c S[table-format=2.2] S[table-format=1.2] c}
            \toprule
            Element & 
            $M_\text{SE}\,/\,\si{\gram\per\mole}$ &
            $M_\text{Mol}\,/\,\si{\gram\per\mole}$ &
            $m\,/\,\si{\gram}$ & 
            $\rho\,/\,\si{\gram\per\cubic\centi\meter}$ &
            $N\,/\,\SI{e22}{\per\cubic\centi\meter}$ \\
            \midrule
            Nd & 144.22 & 336.44 & 9.0   & 7.24 & 2.592 \\
            Gd & 157.25 & 362.50 & 14.08 & 7.40 & 2.459 \\
            Dy & 162.50 & 373.00 & 14.38 & 7.8  & 2.519 \\
            \bottomrule
        \end{tabular}
    \end{table}

Gemäß Gleichung \eqref{eqn:chii} ist die Suszeptibilität temperaturabhängig. 
Der Versuch ist bei Raumtemperatur und Sommerwetter durchgeführt worden. 
Die Temperatur wird deshalb mit ${T\approx(\num{273.15}+\num{23.85})\si{\kelvin}=\SI{297}{\kelvin}}$ abgeschätzt.
Daraus ergeben sich für die Suszeptibilität folgende Werte:
\begin{align*}
    \chi _\text{Nd}=\num{2.98104118e-09} \\
    \chi _\text{Gd}=\num{1.08873279e-08} \\
    \chi _\text{Dy}=\num{2.50790125e-08} 
\end{align*}