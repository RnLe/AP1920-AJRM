\section{Theorie}
\label{sec:Theorie}

\subsection{Verwendete Naturkonstanten im Überblick}

Die in Tabelle \ref{tab:NatKonst} zusammengestellten Naturkonstanten sind \cite{scipy} entnommen. 
Das Bohr'sche Magneton ist über 
\begin{equation*}
    \symup{\mu}_\text{B}=\frac{\symup{e\hbar}}{2\symup{h}}=\frac{\symup{eh}}{4\symup{h\pi}}
\end{equation*}
definiert.

    \begin{table}
        \centering
        \caption{Für das Protokoll benötigte Naturkonstanten auf einen Blick.}
        \label{tab:NatKonst}
        \begin{tabular}{l c c c}
            \toprule
            Naturkonstante &
            Formelzeichen &
            Wert &
            Einheit \\
            \midrule
            Elementarladung                 & $\symup{e}          $   & $\num{1.602176634e-19}$   & \si{\coulomb} \\
            Planck'sches Wirkungsquantum    & $\symup{h}            $ & $\num{6.62607015e-34}$    & \si{\joule\second} \\
            Boltzmann-Konstante             & $\symup{k}_\text{B}   $ & $\num{1.380649e-23}$      & \si{\joule\per\kelvin} \\
            Magnetische Permeabilität       & $\symup{\mu}_0        $ & $\num{1.2566370614e-06}$  & \si{\newton\per\ampere\squared} \\
            Masse eines Elektrons           & $\symup{m_e}          $ & $\num{9.1093837015e-31}$  & \si{\kilo\gram} \\
            Avogadro-Konstante              & $\symup{N}_\text{A}   $ & $\num{6.02214076e+23}$    & \si{\per\mole} \\
            Bohr'sches Magneton             & $\symup{\mu}_\text{B} $ & $\num{9.2740100784e-24}$  & \si{\joule\per\tesla} \\
            \bottomrule
        \end{tabular}
    \end{table}

\subsection{Berechnung der Suszeptibilität paramagnetischer Stoffe}

    Die magnetische Flussdichte $\vec{B}_0$ im Vakuum ist mit der magnetischen Feldstärke $\vec{H}$ durch ${\vec{B}_0=\symup{\mu}_0 \vec{H}}$
    gegeben. Diese Flussdichte unterscheidet sich von der, wenn ein Material sich im Vakuum befindet. 
    In einem solchen Fall gilt 
    \begin{equation*}
        \vec{B}_\text{M}= \vec{B}_0 + \vec{M} \,.
    \end{equation*}
    $\vec{M}$ stellt hierbei die Magnetisierung dar und lässt sich auf das durchschnittliche magnetische Moment der Elementarteilchen 
    zurückführen, welche multipliziert mit der Anzahl pro Volumeneinheit und der magnetischen Feldkonstante eben die Magnetisierung ergeben.
    Eine ebenfalls gültige Formel wird durch 
    \begin{equation*}
        \vec{M}=\symup{\mu}_0 \chi \vec{H}
    \end{equation*}
    ausgedrückt.
    Die Suszeptibilität $\chi$ ist an dieser Stelle die für einen Stoff charakteristische Größe. 

    Bei Diamagneten liegt der Wertebereich zwischen $-1$ und $0$, sie werden aus einem Bereich höherer Feldstärke rausgedrückt, 
    Paramagneten mit einer positiven Suszeptibilität werden hineingezogen. 

    Das atomare magnetische Moment, dessen Mittelung die Magnetisierung definiert, hängt eng mit den Drehimpulsen der einzelnen 
    Elektronen in der Atomhülle zusammen. 
    Der Drehimpuls des Atomkerns kann -- vor allem bei den in diesem Experiment zu untersuchendnen Stoffen -- vernachlässigt 
    werden; ausschlaggebend wird er vor allem dann, wenn ein starkes äußeres Magnetfeld vorhanden ist.

    Somit setzt sich der atomare Gesamtdrehimpuls $\vec{J}$ aus dem Bahndrehimpuls $\vec{L}$ und dem Eigendrehimpuls -- dem 
    Spin -- $\vec{S}$ zusammen:
    \begin{equation*}
        \vec{J}=\vec{S}+\vec{L} \,.
    \end{equation*}
    
    Durch Berücksichtigung der Quantisierung bestimmter Größen lässt sich 
    \begin{equation}   
        \chi = \frac{\symup{\mu}_0 \symup{\mu} _\text{B}^2 g_J^2 NJ(J+1)}{3\symup{k}T}
        \label{eqn:chii}
    \end{equation}
    für die Suszeptibilität herleiten. 
    Näherungen werden bei dem gyromagnetischen Verhältnis des freien Elektrons $g_\text{S}$ vorgenommen, welches in einigen Zwischenrechnungen auftaucht und 
    abgesehen von ein paar Promillen Abweichungen den Wert $2$ annimmt. 
    Ebenfalls lässt sich die sogenannte Brillouin-Funktion nähern, da im Experiment bei Zimmertemperatur und mit 
    Magnetfeldern, die nicht größer als $\SI{1}{\tesla}$ sind, gearbeitet wird.

    Der Landé-Faktor $g_J$ wird an dieser Stelle mithilfe 
    \begin{equation}
        g_J=\frac{3J(J+1)+s(s+1)-L(L+1)}{2J(J+1)}
        \label{eqn:lande}
    \end{equation}
    gegeben, wobei $S$, $L$ und $J$ die Quantenzahlen der bereits genannten Drehimpulse darstellen, die sich aus der Summe
    der Quantenzahlen der an der Magnetisierung beteiligten Elektronen ergeben -- die Drehimpulsquantenzahlen der restlichen 
    Elektronen summieren sich gerade zu Null. 

\subsection{Die Hund'schen Regeln und das Pauli-Prinzip hinsichtlich Seltener-Erd-Verbindungen}
\label{sub:wauwau}

    Der Landé-Faktor Seltener-Erd-Verbindungen lässt sich mithilfe der Hund'schen Regeln berechnen. 
    Sie legen fest, welche Quantenzahlen $J$, $L$ und $S$ in Gleichung \eqref{eqn:lande} eingesetzt werden müssen. 

    Seltene Erden haben eine voll besetzte Elektronenhülle bis zur 5p-Schale. Darauf aufbauend befinden sich je nach 
    Element und Ionisationszustand noch zwei 6s-ELektronen und weitere 4f-Elektronen in der Elektronenhülle. 
    Die 4f-Elektronen sind die für die Magnetisierung entscheidenden Teilchen -- sie legen die Quantenzahlen fest, 
    die der anderen Schalen kompensieren sich gegenseitig. 

    Die Hund'schen Regeln sind:
    \begin{enumerate}
        \item Der Eigendrehimpuls beziehungsweise der Spin $S$ ist maximal, woraus eine möglichst parallele Orientierung 
            der einzelnen Elektronenspins resultiert.    
        \item Der Bahndrehimpuls $L$ ist maximal, also ebenfalls nach Möglichkeit parallel. 
        \item Wenn eine Unterschale maximal zur Hälfte gefüllt ist, ergibt sich ${J=L-S}$, andernfalls ${J=L+S}$.
    \end{enumerate}
    Wichtig zu wissen ist hierbei: 
    \begin{itemize}
        \item Der Spin nimmt die Werte $\sfrac{1}{2}$ oder $\sfrac{-1}{2}$ an. 
        \item Hierbei seien die summierten Werte durch $S=\sum_i s_i$ und $L=\sum_i l_i$ gegeben. 
        \item Die ausschlaggebende Unterschale ist 4f; das bedeutet, dass die Hauptquantenzahl ${n=4}$ ist.
        \item Für die Quantenzahlen des Bahndrehimpuls gilt: $l_i \in [0,n-1] \Rightarrow l_i \in [0,3]$ .
    \end{itemize}

    Ein letzter wichtiger Aspekt für die Bestimmung der Quantenzahlen stellt das Pauli-Prinzip dar. 
    Es besagt, dass mindestens eine Quantenzahl der Elektronen innerhalb einer Schale paarweise verschieden sein muss. 
    
    Mithilfe dieser Gesetzmäßigkeiten können im Nachhinein die Theorie-Werte für die Suszeptibilität der entsprechenden Stoffe bestimmt werden. 

\subsection{Messung der Suszeptibilität}

    