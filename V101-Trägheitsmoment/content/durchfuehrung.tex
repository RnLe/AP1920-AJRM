\section{Durchführung}
\label{sec:Durchführung}

\subsection{Einleitung}
Ziel des Versuchs ist es, die Trägheitsmomente verschiedener Körper zu bestimmen. Hierfür wird ein Stativ verwendet, an dem eine Spiralfeder angebracht ist.
Die Spiralfeder ist nach oben beziehungsweise unten ausgerichtet. In der Mitte der Feder befindet sich eine zylinderförmige Halterung, welche nach oben gerichtet ist und sich frei bewegen kann.
Das äußere Ende ist an dem Stativ angebracht, welches U-förmig um die Spiralfeder herum führt. Auf dem Stativ befindet sich eine Scheibe mit einem Loch, an der man Winkel ablesen kann.
Durch das Loch führt die Halterung, welche mittig aus der Feder hervorgeht. Mit dieser Konfiguration können Objekte an der Halterung angebracht und Auslenkungen an 
der Lochscheibe abgelesen werden.

\subsection{Apparatekonstante}
Zu Beginn werden Messungen durchgeführt, um die Apparatekonstante $D$ zu bestimmen.
Dazu wird eine lange, leichte Stange horizontal an der Halterung befestigt. Daraufhin wird die Winkelscheibe so ausgerichtet, dass sich Auslenkungen genau ablesen lassen.
Mit einer Federwaage, welche Kräfte bis zu \SI{1}{\newton} misst, wird die Stange nun an einer anfangs beliebigen Stelle ausgelenkt. Dabei ist es wichtig, dass die Federwaage
orthogonal an der Stange zieht, und nicht abgewinkelt. So muss kein Winkel beim Betrag des Drehmoments $\vec{M}=\vec{r}\times\vec{F}$ berücksichtigt werden.  
Die Kraft wird in Abhängigkeit des Winkels aufgeschrieben und
%Es wird das Wertepaar des Winkels und der Kraft aufgeschrieben und 
die Messung insgesamt zehn mal durchgeführt.
Hierbei wird ausschließlich die Auslenkung beziehungsweise die Kraft, jedoch nicht der Abstand zur Drillachse variiert. Es werden also alle Messungen mit demselben Abstand durchgeführt.

\subsection{Eigenträgheit}
Zur Bestimmung des Trägheitheitsmomentes der Drillachse selbst werden zwei Gewichte verwendet, deren Masse zunächst gewogen wird.
Die Massen sind Hohlzylinder und besitzen eine Feststellschraube, um sie auf der Stange fixieren zu können.
Es wird an je einer Seite der Stange eine Masse angebracht. Wenn sie fixiert sind, wird der Abstand zur Drillachse notiert und die Stange anschließend sehr leicht,
um etwa \ang{10;;}-\ang{20;;} ausgelenkt. Nach dem Loslassen wird die Schwingungsdauer der Stange gemessen.
Dies wird insgesamt zehn mal unter Variation des Abstands der Massen zur Drehachse durchgeführt.

\subsection{Trägheit der Körper}
Für die Messungen der einzelnen Testkörper wird immer gleich vorgegangen. Der Körper wird an der Halterung befestigt und anschließend um \ang{10;;}-\ang{20;;} ausgelenkt.
Daraufhin wird die Schwingungsdauer gemessen. Dies wird fünf Mal wiederholt. Gemessen werden die Werte für zwei beliebige Körper und für eine menschliche 
Gliederpuppe aus Holz. Alle Testkörper werden gewogen und ausgemessen. Die Holzpuppe kann beliebig klein angenähert, jedoch mindestens in sechs Einzelteile aufgeteilt werden;
Kopf, zwei Arme, zwei Beine, Torso. Die Holzfigur wird in verschiedene Posen gebracht, zu der jeweils fünf Messwerte nach obiger Beschreibung aufgenommen werden.
Zudem wird die Geometrie der Posen vermessen und notiert, um im Anschluss das Trägheitsmoment bestimmen zu können.