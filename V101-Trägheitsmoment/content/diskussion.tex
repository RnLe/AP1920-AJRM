\section{Diskussion}
\label{sec:Diskussion}

Es fällt auf, dass das Trägheitsmoment der Drillachse vergleichsweise hoch ist.
Hierdurch ergeben sich schnell negative Momente für die einzelnen Körper.
Dafür jedoch ist der erwartete Wert des kleinen Zylinders ${I_\text{kl,geo} = \SI{7.74}{\kilo\gram\squared\centi\meter}}$
zufriedenstellend nahe an dem gemessenen Wert $\SI{6.44\pm0.55}{\kilo\gram\squared\centi\meter}$.

Gleiches gilt für den großen Zylinder mit ${I_\text{gr,geo} = \SI{30.75}{\kilo\gram\squared\centi\meter}}$ und dem
gemessenen Wert $\SI{24.24\pm2.32}{\kilo\gram\squared\centi\meter}$.

Auch bei der Puppe ergeben sich negative Massenträgheitsmomente nach Abziehen von $I_D$.
Mögliche Gründe sind systematische Fehler der Messungen. Da sich diese aus vielen einzelnen Prozessen zusammensetzen,
können Fehler beim Wiegen, der Längenmessungen, des Ablesens des Kraftmessers und der Auslenkungen, insbesondere aber beim Stoppen
der Zeit nicht ausgeschlossen werden. Gerade bei der Holzpuppe ergeben sich sehr kurze Periodendauern, wodurch relative Fehler zunehmen.
Außerdem ist die Holzpuppe nicht vollkommen starr und wackelt deutlich, sodass sie bei den Drehungen etwas nachgibt, anstatt träge zu sein.

Letztlich kann auch nicht ausgeschlossen werden, dass die Körper zu weit ausgelenkt werden. Da es sich bei der Theorie
um Kleinwinkelnäherungen handelt, können etwas größere Auslenkungen die Messergebnisse stark verfälschen und unbrauchbar, oder vielmehr unplausibel machen.

Um auf andere Weise einen Vergleich der Messwerte auf ihre Validität hin zu erhalten, kann das Verhältnis der 
experimentellen Werte dem der theoretischen Werte gegenübergestellt werden: 

\begin{gather}
    \frac{I_\text{geo,Pose1}}{I_\text{geo,Pose2}}=\SI{7.12}{\percent} \quad \quad \quad 
    \frac{I'_\text{P1,T}}{I'_\text{P2,T}}=\SI{19.25}{\percent}
\end{gather}

Auch an dieser Stelle ist zu erkennen, dass die Abweichungen sehr groß sind und diese mit hoher Wahrscheinlichkeit 
unter dem Einfluss systematischer Messfehler zustandekommen. Somit erhärtet sich an dieser Stelle die Annahme, dass es sich 
hierbei um keine mit der Theorie konsistenten Messwerte handeln kann. 