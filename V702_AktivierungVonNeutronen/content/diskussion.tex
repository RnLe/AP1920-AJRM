\section{Diskussion}
\label{sec:Diskussion}

Der Versuch liefert gute Näherungen an die tatsächlichen Halbwertszeiten der Isotope $^{52}V$ und $^{104}Rh$.
So liegen die Abweichungen für das Vanadium-Isotop bei etwa $13.1\%$ und für das Rhodium-Isotop bei $3.7\%$ und $7.6\%$ für je den langsamen und schnellen Zerfall.

\subsection{Vanadium}

\begin{table}
    \centering
    \caption{Abweichungen des Vanadium-Isotops.}
    \label{tab:abwVan}
    \begin{tabular}{c c c c}
        \toprule
         & Literaturwert $\tau_\text{V,L}\,/\,\si{\second}$\cite{perTable} & Alle Messwerte $\tau_\text{V,all}\,/\,\si{\second}$ & Angepasst $\tau_\text{V}\,/\,\si{\second}$ \\
        \midrule
         & $\SI{224.58}{\second}$ & $\SI{312\pm11}{\second}$ & $\SI{254\pm11}{\second}$ \\
        Abweichungen & - & $38.93\%$ & $13.1\%$ \\
        \bottomrule
    \end{tabular}
\end{table}

Das Vanadium-Isotop hat eine konstante Halbwertszeit. Wenn die meisten Isotope zerfallen sind, kann über die Zählapparatur jedoch nicht mehr unterschieden werden,
ob es sich um die Isotop-Zerfälle oder Zerfälle aus Höhenstrahlung und natürlicher Radioaktivität handelt. Der Nulleffekt bildet somit eine untere Grenze der Messgenauigkeit.
Aus diesem Grund ist es sinnvoll, nur etwa die erste Hälfte der Messwerte zu betrachten und eine lineare Regression für nur diese Werte durchzuführen.
Ein Vergleich mit dem Literaturwert zeigt, dass die neu gewonnene Näherung deutlich genauer ist und eine Abweichung um den Faktor drei reduziert.
Eine Gegenüberstellung ist in Tabelle \ref{tab:abwVan} zu sehen.

\subsection{Rhodium}

\begin{table}
    \centering
    \caption{Abweichungen des Rhodium-Isotops.}
    \label{tab:abwRh}
    \begin{tabular}{c c c c c}
        \toprule
        & \multicolumn{2}{c}{Literaturwerte\cite{RhIsotopes}} & & \\
        & $\tau_\text{sch, L}\,/\,\si{\second}$ & $\tau_\text{lan, L}\,/\,\si{\second}$ & $\tau_\text{sch}\,/\,\si{\second}$ & $\tau_\text{lan}\,/\,\si{\second}$ \\
        \midrule
        & $\SI{42.3}{\second}$ & $\SI{260.4}{\second}$ & $\SI{39.3\pm1.4}{\second}$ & $\SI{270\pm60}{\second}$ \\
        Abweichungen & - & - & $3.7\%$ & $7.6\%$ \\
        \bottomrule
    \end{tabular}
\end{table}

Der schnelle Zerfall des Rhodium-Isotops ist dem Literaturwert nur um $3.7\%$ verschieden und damit gut genähert. Die relative Unsicherheit des Wertes beträgt etwa $3.6\%$.
Der langsame Zerfall hingegen hat mit $7.6\%$ eine größere Abweichung dem Literaturwert gegenüber. Zudem ist die Messunsicherheit dieser Halbwertszeit mit $22\%$ vergleichsweise hoch.
Die Hauptursache für diese statistischen Unterschiede ist vor allem der Nulleffekt, der nach dem schnellen Zerfall einen großen Einfluss auf die zweite Messhälfte hat.
Das Messverhalten ist vergleichbar mit dem des Vanadium-Isotops. \\

Um die Messgenauigkeit zu verbessern können die Proben intensiver oder länger bestrahlt werden, sodass mehr Isotope gebildet werden. Hierdurch wird der statistische Einfluss
des Nulleffektes reduziert, da zum Ende der Messung noch wesentlich mehr Isotope zerfallen als Strahlung durch die Umgebung gemessen wird.
