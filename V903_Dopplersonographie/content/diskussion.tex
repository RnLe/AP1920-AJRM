\section{Diskussion}
\label{sec:Diskussion}

Der Versuch ist ohne Schwierigkeiten durchführbar und die verwendeten Gerätschaften vollständig funktionsfähig.
Auch die Bedienung des Computers, des Programms, der Pumpe und des Sonographen sind leicht verständlich und intuitiv.
Lediglich das Armphantom aus der Versuchsanleitung fehlt.

\subsection{Strömungsgeschwindigkeit}
Die in den Abbildungen \ref{fig:15}, \ref{fig:30} und \ref{fig:45} gegenübergestellten Werte entsprechen der Erwartung.
Die grünen Messpunkte zeigen die vom Sonographen maximal gemessene Frequenzverschiebung; die gelben zeigen einen Mittelwert dieser Verschiebung.
Es ist also plausibel, dass die grünen Messpunkte höher verschoben sind. Es ist ebenso verständlich, dass der Mittelwert ein stabileres Verhalten zeigt als die Maximalwerte,
welche nur obere Ausreißer betrachten. Ein stabiles Verhalten meint hier eine Linearität, welche von der Thoerie vorhergesagt wird, obgleich die Mittelwerte nicht perfekt linear, sondern
leicht parabelförmig sind - also eine größer werdende Steigung besitzen. Dadurch scheint die Annahme, dass die Umdrehungszahl proportional
zur Strömungsgeschwindigkeit ist, zunächst bestätigt zu sein.

Die oberen Ausreißer hingegen zeigen ein stärkeres parabelförmiges Verhalten. Grund hierfür können technische Eigenschaften der Pumpe sein.
Auch können geometrische Charakteristika wie etwa raue Innenwände für dämpfende Turbulenzen sorgen, die bei höheren Strömungsgeschwindigkeit Grenzschichten ausbilden und somit
mehr Raum für laminare Strömung schaffen. 

\subsection{Strömungsprofil}
Die Messungen der Streuintensität und der Strömungsgeschwindigkeit sind ähnlich. Es fällt auf, dass die Strömungsgeschwindigkeit einen Knick am Ende der Messung hat. Dies ist unter
anderem darauf zurückzuführen, dass die Eindringtiefe über den Rohrdurchmesser hinaus geht. Dadurch ist eine Betrachtung aller Messwerte nicht sinnvoll.

Die Werte entsprechen auch hier den Erwartungen aus der Theorie. Auffällig sind die erhöhten Geschwindigkeitswerte an den Innenseiten des Schlauches, also bei etwa 
$r \approx \SI{\pm 5}{\milli\meter}$. Eine Erklärung für dieses Phänomen ist die vorhin bereits erwähnte Rauheit der Innenwände, welche lokale Turbulenzen erzeugen kann, den laminaren
Fluss innerhalb der Röhre aber nicht stört. Anhand der Messwerte kann darauf geschlossen werden, dass diese turbulente Grenzschicht kleiner als $\SI{1}{\milli\meter}$ sein muss.

\subsection{Fazit}
Der Versuch ist hinreichend durchführbar. Die Messapparaturen sind genau genug, um die Theorie zu bestätigen und kleinere, beziehungsweise nebensächliche Phänomene darzustellen.
Es wird für zukünftige Teilnahmen an diesem Versuch empfohlen, auch die Mittelwerte (\textit{mean}) der Frequenzverschiebung aus dem Programm abzulesen, da diese eine größere 
experimentelle Aussagekraft besitzen.