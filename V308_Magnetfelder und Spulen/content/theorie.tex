\section{Theorie}
\label{sec:Theorie}
Theoretische Grundlagen für dieses Experiment sind das Biot--Savartsche Gesetz, der Hall-Effekt, elektromagnetsiche Induktion und Ferromagnetismus.

Magnetfelder bilden sich durch bewegte Ladungen, wie beispielsweise durch Strom. 
Eine Quantifizierung der Feldstärke $\symbf{H}$ wird durch das Biot--Savart-Gesetz 
\begin{equation}
\symbf{H}=\frac{I}{4\pi}\int_\Gamma \frac{\symup{d}\symbf{s}\times\symbf{r}}{r^3}
\label{eqn:biotsavart}
\end{equation}
gegeben, bei dem über die Leiterschleife $\Gamma$ integriert wird.
Für Vakuum und Materialien, deren magnetische Momente aufgrund der Wärmebewegung statistisch verteilt sind, gilt der 
Zusammenhang 
\begin{equation}
\symbf{B}=\symup{\mu_0}\mu_r\symbf{H}
\end{equation}
für die magnetische Flussdichte $\symbf{B}$. 
Dabei sind $\symup{\mu_0}=4\pi \cdot 10^(-7) \, \si{\volt\second\ampere\tothe{-1}\meter\tothe{-1}}$ die Permeabilitätskonstante 
und $\mu_r$ die materialabhängige relative Permeabilität.

Die Untersuchung des Stoffmagnetismus, wie sie in diesem Versuch unter anderem vorgenommen wird, erfordert die Unterteilung 
in Para-, Dia- und Ferromagneten. 
Die relative Permeabilität $\mu_r$ von sowohl Para-, als auch Diamagneten ist eine konstante Zahl. 
Das bedeutet im Umkehrschluss, dass sich das im Material durch die Stoffeigenschaften zusätzlich ausbildende Magnetfeld, die 
sogenannte Magnetisierung 
\begin{equation}
\symbf{M}=\frac{1}{\symup{\mu_0}}\symbf{B}_\text{Materie}-\symbf{H}_\text{Vakuum}=\symbf{H}_\text{Vakuum}(\symup{\mu_0}-1) \text{,} 
\label{eqn:Magnetisierung}
\end{equation}
parallel zum äußeren Feld $\symbf{H}$ ist. 
Beim Diamagneten ist die Parallelität vielmehr eine Antiparallelität; das heißt, $\symbf{M}$ hat eine das Feld
abschwächende Wirkung und der Magnet wird aus Bereichen hoher Feldstärke herausgestoßen. 
Daraus ergibt sich, dass $\symup{\mu_0}-1 < 0$, also $\symup{\mu_0}<1$ gelten muss. 

Genau entgegengesetzt ist es beim Paramagneten: Hier gilt echte Parallelität, also $\symup{\mu_0}-1 > 0$ beziehungsweise 
$\symup{\mu_0}> 1$, der Magnet hat eine das Feld verstärkende Wirkung und wird in Gebiete hoher Feldstärke hineingezogen.

Ferromagneten legen ein anderes Verhalten an den Tag: Die relative Permeabilität ist keine Konstante, sondern vielmehr 
eine komplizierte Funktion des angelegten Feldes $\symbf{H}_\text{Vakuum}$, die von der Geschichte des Materials abhängt. 
Anhand einer Hysteresekurve, bei der die Magnetisierung $\symbf{M}$ auf das äußere Magnetfeld $\symbf{H}_\text{Vakuum}$ 
aufgetragen wird, wird das Verhalten solcher Magneten im Folgenden erklärt. 
%HIER NOCH BILD VON HYSTERESE EINFÜGEN

Ferromagneten besitzen sogenannte \textit{Weiß'sche Bezirke}, in denen die Teilchen das gleiche magnetische Dipolmoment besitzen. 
Ist der Magnet unmagnetisiert, ist deren Ausrichtung statistisch verteilt und sie kompensieren sich, sodass $\symbf{M}=0$ gilt.
Wird nun ein äußeres Magnetfeld $\symbf{H}_\text{Vakuum}$ angelegt und gesteigert, richten sich immer mehr Weißsche Bezirke entlang des 
Feldes aus, wodurch sich diese vergrößern. 
Dies geschieht bis zu einer Obergrenze, die \textit{Sättigung} des Materials wird erreicht. 
Wird nun $\symbf{H}_\text{Vakuum}$ wieder verringert und ganz abgeschaltet, bleibt eine Restmagnetisierung übrig, 
die sogenannte \textit{Remanenz}. 
Erst bei Anlegen eines entgegengesetzten Magnetfelds verschwindet die Magnetisierung an einem gewissen Punkt. 
Dieser wird mit der \textit{Koerzitivfeldstärke} betitelt. 
Derselbe Durchlauf kann nun in anderer Richtung durchgespielt werden: $\symbf{H}_\text{Vakuum}$ wird im negativen 
maximiert, bis der Ferromagnet in entgegensetzter Richtung seine magnetisische Sättigung erreicht, danach wieder 
bis ins positive gesteigert, sodass der Magnet die Punkte der Remanenz und Koerzitivfeldstärke nur mit anderem Vorzeichen 
durchläuft. 

Gemessen werden die Magnetfelder mit einer Hall-Sonde, die auf folgender Funktionsweise beruht:
Die stromdurchflossene Messspitze wird in das entsprechende Magnetfeld gehalten. 
Da Magnetfelder auf bewegte Ladungen die Lorentzkraft $F_L$ ausüben, werden die Ladungsträger alle in die gleiche Richtung
abgelenkt. Dies passiert solange, bis sich die Lorentzkraft mit der elektrischen Kraft kompensiert, die sich gleichnamige 
Ladungen abstoßen lässt. Es hat sich eine stabile Ladungskonfiguration ausgebildet, zwischen der sich die \textit{Hall-Spannung} 
messen lässt. Daraus lässt sich dann das zu messende Magnetfeld berechnen. Die Hall-Sonde zeigt den sich ergebenden Wert 
für die magnetische Flussdichte $\symbf{B}$ an. 
Es gibt verschiedene Modelle von Sonden, transversale und longitudinale. Diese unterscheiden sich alleinig durch die 
Orientierung der Messspitze. Die Wahl des Modells hängt also nur von der Geometrie der Messung ab und hat keiner tiefergehende Bedeutung.

Ergänzend zu dem Biot-Savart'schen Gesetz seien hier noch einige Magnetfeldinstallationen vorgestellt, die für das Experiment 
von Bedeutung sind.

Mithilfe \eqref{eqn:biotsavart} ergibt sich für einen stromdurchflossenen Drahtring mit Radius $R$ auf seiner durch den Kreismittelpunkt 
gehenden Symmetrieachse (der Parameter $x$ sei im Kreismittelpunkt Null)
\begin{equation}
    \lvert\symbf{B}_\text{Ring}(x)\rvert = B_\text{Ring}(x) = \frac{\symup{\mu_0 I}}{2} \frac{R^2}{(R^2 + x^2)^{\frac{3}{2}}}.
    \label{eqn:ring}
\end{equation}
Werden zwei Drahtringe bei $x=\sfrac{d}{2}$ und $x=\sfrac{-d}{2}$ positioniert und in gleicher Richtung mit Strom durchflossen,
gilt gemäß des Superpositionsprinzips: 
\begin{equation}
    B_d(x)=B_\text{Ring}(x-\frac{d}{2}) + B_\text{Ring}(x+\frac{d}{2})
    \label{eqn:2ringe}
\end{equation}

Das B-Feld innerhalb einer langen Spule ist nahezu homogen, sofern die Länge $l$ sehr viel größer als der Radius $R$ ist. 
Die Magnetfeldlinien innerhalb verlaufen parallel zur Symmetrieachse und bilden außerhalb einen großen 
Bogen vom Ende zum Anfang der Spule, sodass geschlossene Feldlinien durch die Spule laufen.
Innerhalb der Spule gilt abgesehen von Randeffekten 
\begin{equation}
    B_\text{Sp} = \mu_r \symup{\mu_0} \frac{n}{l} I
    \label{eqn:langespule}
\end{equation}
für die Flussdichte mit der Windungszahl $n$ und dem Strom $I$. 
Wird die lange Spule zu einem Kreis gebogen, entsteht eine Toroidspule, außerhalb der das Magnetfeld Null ist und bei der 
die Randeffekte verschwinden. 
Für die Flussdichte gilt mit \eqref{eqn:langespule} und durch Ersetzung der Länge $l$ durch den Umfang des Toroiden des Radius $r_T$:
\begin{equation}
    B_T=\mu_r \symup{\mu_0} \frac{n}{2 \pi r_T} I
    \label{eqn:toroid}
\end{equation}
