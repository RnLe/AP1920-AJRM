\section{Auswertung}
\label{sec:Auswertung}
\subsection{Helmholtzspule}
Die $x$-Skala entspricht der an der Apparatur angebrachten Skala bei der Experimentdurchführung. 
Hierbei war durch $x=0$ die innenliegende Spulenkante der linken Spule gegeben. 
Zur Vereinfachung der Auswertung wurde folgende Verschiebung der Achse vorgenommen, sodass sich idealerweise die Mitte der 
beiden Spulen bei $y=0$ befindet:
\begin{equation*}
    y=x+\frac{b}{2} -\frac{d}{2}
\end{equation*}
Dabei entsprechen $b$ der Spulenbreite und $d$ dem jeweiligen Abstand der Messreihe.
\begin{table}
    \centering
    \caption{Daten der verwendeten Doppelspule und Grundeinstellungen.}
    \label{tab:HH}
    \begin{tabular}{c c c c}
        \toprule
        {Windungszahl je Spule $n$} & {Spulendurchmesser $2R$} & {Spulenbreite $b$} & {Strom $I$}\\
        \midrule
        100 &  $\SI{125}{\milli\m}$ & $\SI{33}{\milli\meter}$ & $\SI{3.03}{\ampere}$\\
        \bottomrule
    \end{tabular}
\end{table}

\begin{table}
    \centering
    \caption{tab:1. Messreihe mit einem Abstand von $d=R=\SI{62.5}{\milli\meter}$.}
    \label{tab:HH1} %Helmholtz 1
    \begin{tabular}{S[table-format=3.1] S[table-format=2.2] S[table-format=1.3]}
        \toprule
        {$x\,/\,\si{\milli\m}$} & {$y\,/\,\si{\milli\m}$} & {$B\,/\,\si{\milli\tesla}$} \\
        \midrule
        7    & -7.75 & 4.239 \\
        9    & -5.75 & 4.231 \\
        10.5 & -4.25 & 4.234 \\
        12   & -2.75 & 4.260 \\ 
        13.5 & -1.25 & 4.239 \\
        68.5 & 53.75 & 3.003 \\
        69   & 54.25 & 2.960 \\
        70   & 55.25 & 2.891 \\
        75   & 60.25 & 2.666 \\
        80   & 65.25 & 2.445 \\
        85   & 70.25 & 2.219 \\
        90   & 75.25 & 2.018 \\
        95   & 80.25 & 1.839 \\
        100  & 85.25 & 1.662 \\
        \bottomrule
    \end{tabular}
\end{table}

\begin{table}
    \centering
    \caption{tab:2. Messreihe mit einem Abstand von $d=\SI{104}{\milli\meter}$.}
    \label{tab:HH2} %Helmholtz 2
    \begin{tabular}{S[table-format=3.1] S[table-format=3.2] S[table-format=1.3]}
        \toprule
        {$x\,/\,\si{\milli\m}$} & {$y\,/\,\si{\milli\m}$} & {$B\,/\,\si{\milli\tesla}$} \\
        \midrule
        7     & -28.5 & 3.091 \\
        16    & -19.5 & 2.976 \\
        27    & -8.5  & 2.887 \\
        33.5  & -2    & 2.882 \\
        44    & 8.5   & 2.945 \\
        54.5  & 19    & 3.081 \\
        108.5 & 73    & 2.639 \\
        115   & 79.5  & 2.410 \\
        120   & 84.5  & 2.194 \\
        125   & 89.5  & 2.031 \\
        130   & 94.5  & 1.849 \\
        160   & 124.5 & 1.036 \\
        190   & 154.5 & 0.615 \\
        230   & 194.5 & 0.366 \\
        \bottomrule
    \end{tabular}
\end{table}

\begin{table}
    \centering
    \caption{tab:3. Messreihe mit einem Abstand von $d=\SI{130}{\milli\meter}$.}
    \label{tab:HH3} %Helmholtz 3
    \begin{tabular}{S[table-format=3.0] S[table-format=3.2] S[table-format=1.3]}
        \toprule
        {$x\,/\,\si{\milli\m}$} & {$y\,/\,\si{\milli\m}$} & {$B\,/\,\si{\milli\tesla}$} \\
        \midrule
        7   & -41.5 & 2.754 \\
        22  & -26.5 & 2.420 \\
        30  & -18.5 & 2.288 \\
        45  & -3.5  & 2.199 \\
        51  & 2.5   & 2.211 \\
        63  & 14.5  & 2.354 \\
        79  & 30.5  & 2.688 \\
        134 & 85.5  & 2.529 \\
        149 & 100.5 & 2.007 \\
        155 & 106.5 & 1.789 \\
        163 & 114.5 & 1.540 \\
        171 & 122.5 & 1.315 \\
        179 & 130.5 & 1.128 \\
        190 & 141.5 & 0.909 \\
        208 & 159.5 & 0.663 \\
        \bottomrule
    \end{tabular}
\end{table}

\subsection{Hysteresekurve}
    Gemessen wurden wie erläutert der Strom $I$ durch die Spule und das Magnetfeld im Luftspalt der Spule. 
    Um die Hysteresekurve zu zeichnen, muss die Magnetisierung $B_\text{Fe}$ vom Eisenkern auf das durch die Spule verursachte 
    Magnetfeld $H$ aufgetragen werden. 
    Dafür gelten die Zusammenhänge: %hier eventuell Referenzen zu den Formeln einfügen, wenn diese bereits in der Theorie erwähnt wurden, und dann mit gather* die Nummerierung wegmachen
    \begin{gather}
        H=\symup{\mu_0}\frac{n}{2\symup{\pi}r} I \\
        B_\text{mess} = H+B_\text{Fe}
    \end{gather}
    \begin{table}
        \centering
        \caption{tab:Messwerte der Hysteresekurve.}
        \label{tab:Hyst} 
        \begin{tabular}{S[table-format=1.2] S[table-format=1.3] S[table-format=3.0] S[table-format=3.3]}
            \toprule
            {$I\,/\,\mathrm{A}$} & {$H\,/\,\mathrm{mT}$} & {$B_\text{mess}\,/\,\mathrm{mT}$} & {$B_\text{Fe}\,/\,\mathrm{mT}$} \\
            \midrule
            0.0     & 0.        & 0     & 0.0       \\
            1.0     & 0.881     & 148   & 147.1     \\
            2.0     & 1.762     & 335   & 333.2     \\
            3.0     & 2.644     & 440   & 437.3     \\
            4.0     & 3.525     & 508   & 504.4     \\
            5.0     & 4.407     & 561   & 556.5     \\
            6.0     & 5.288     & 603   & 597.7     \\
            7.0     & 6.170     & 638   & 631.8     \\
            8.0     & 7.051     & 668   & 660.9     \\
            9.0     & 7.933     & 698   & 690.0     \\
            8.0     & 7.051     & 677   & 669.9     \\
            7.0     & 6.170     & 656   & 649.8     \\
            6.0     & 5.288     & 632   & 626.7     \\
            5.0     & 4.407     & 604   & 599.5     \\
            4.0     & 3.525     & 569   & 565.4     \\
            3.0     & 2.644     & 524   & 521.3     \\
            2.0     & 1.762     & 460   & 458.2     \\
            1.0     & 0.881     & 332   & 331.1     \\
            0.0     & 0.        & 128   & 128.0     \\
            -0.65   & -0.572    &  0    & 0.573    \\
            -1.0    & -0.881    & -71   & -70.11    \\
            -2.0    & -1.762    & -252  & -250.2    \\
            -3.0    & -2.644    & -390  & -387.3    \\
            -4.0    & -3.525    & -482  & -478.4    \\
            -5.0    & -4.407    & -546  & -541.5    \\
            -6.0    & -5.288    & -594  & -588.7    \\
            -7.0    & -6.170    & -634  & -627.8    \\
            -8.0    & -7.051    & -669  & -661.9    \\
            -9.0    & -7.933    & -698  & -690.0    \\
            -8.0    & -7.051    & -679  & -671.9    \\
            -7.0    & -6.170    & -658  & -651.8    \\
            -6.0    & -5.288    & -635  & -629.7    \\
            -5.0    & -4.407    & -607  & -602.5    \\
            -4.0    & -3.525    & -573  & -569.4    \\
            -3.0    & -2.644    & -529  & -526.3    \\
            -2.0    & -1.762    & -464  & -462.2    \\
            -1.0    & -0.881    & -339  & -338.1    \\
            0.0     & 0.        & -129  & -129.0    \\
            0.6     & 0.528     & -8    & -8.528    \\
            0.67    & 0.595     & 0     & -0.595   \\
            1.0     & 0.881     & 72    & 71.12     \\
            2.0     & 1.762     & 253   & 251.2     \\
            3.0     & 2.644     & 390   & 387.3     \\
            4.0     & 3.525     & 482   & 478.4     \\
            5.0     & 4.407     & 544   & 539.5     \\
            6.0     & 5.288     & 592   & 586.7     \\
            7.0     & 6.170     & 630   & 623.8     \\
            8.0     & 7.051     & 663   & 655.9     \\
            9.0     & 7.933     & 693   & 685.0     \\
            \bottomrule
        \end{tabular}
    \end{table}

\subsection{Spulen}
    \subsubsection{Kurze Spule}
    \begin{table}
    \centering
    \caption{tab:Messwerte der kurzen Spule.}
    \label{tab:kurzSp}
        \begin{tabular}{S[table-format=3.0] S[table-format=2.0] S[table-format=2.2]}
            \toprule
            {$x\,/\,\mathrm{mm}$} & {$y\,/\,\mathrm{mm}$} & {$B\,/\,\mathrm{mT}$}\\
            \midrule
            0   & -96   & 13.71 \\
            15  & -81   & 16.68 \\
            30  & -66   & 18.48 \\
            40  & -56   & 18.81 \\
            50  & -46   & 18.44 \\
            60  & -36   & 17.40 \\
            73  & -23   & 15.02 \\
            87  & -9    & 11.82 \\
            100 & 4     & 8.82  \\
            110 & 14    & 6.90  \\
            121 & 25    & 5.23  \\
            135 & 39    & 3.68  \\
            151 & 55    & 2.52  \\
            \bottomrule
        \end{tabular}
    \end{table}

    \subsubsection{Lange Spule}
    \begin{table}
    \centering
    \caption{tab:Messwerte der langen Spule.}
    \label{tab:langSp}
        \begin{tabular}{S[table-format=3.0] S[table-format=2.0] S[table-format=2.2]}
            \toprule
            {$x\,/\,\mathrm{mm}$} & {$y\,/\,\mathrm{mm}$} & {$B\,/\,\mathrm{mT}$}\\
            \midrule
            0   & -79   & 3.076 \\
            5   & -74   & 3.051 \\
            10  & -69   & 3.020 \\
            15  & -64   & 2.977 \\
            20  & -59   & 2.916 \\
            31  & -48   & 2.671 \\
            39  & -40   & 2.318 \\
            47  & -32   & 1.775 \\
            55  & -24   & 1.174 \\
            64  & -15   & 0.680 \\
            70  & -9    & 0.501 \\
            80  & 1     & 0.317 \\
            100 & 21    & 0.177 \\
            \bottomrule
        \end{tabular}
    \end{table}