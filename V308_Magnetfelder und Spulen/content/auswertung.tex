\section{Auswertung}
\label{sec:Auswertung}
\subsection{Helmholtzspule}
Die $x$-Skala entspricht der an der Apparatur angebrachten Skala bei der Experimentdurchführung. 
Hierbei war durch $x=0$ die innenliegende Spulenkante der linken Spule gegeben. 
Zur Vereinfachung der Auswertung wurde folgende Verschiebung der Achse vorgenommen, sodass sich idealerweise die Mitte der 
beiden Spulen bei $y=0$ befindet:
\begin{equation*}
    y=x+\frac{b}{2} -\frac{d}{2}
\end{equation*}
Dabei entsprechen $b$ der Spulenbreite und $d$ dem jeweiligen Abstand der Messreihe.
\begin{table}
    \centering
    \caption{Daten der verwendeten Doppelspule und Grundeinstellungen.}
    \label{tab:HH}
    \begin{tabular}{c c c c
        \toprule
        {Windungszahl je Spule $n$} & {Spulendurchmesser $2R$} & {Spulenbreite $b$} & {Strom $I$}\\
        \midrule
        100 &  $\SI{125}{\milli\m}$ & $\SI{33}{\milli\meter}$ & $\SI{3.03}{\ampere}$\\
        \bottomrule
    \end{tabular}
\end{table}

\begin{table}
    \centering
    \caption{tab:1. Messreihe mit einem Abstand von $d=R=\SI{62.5}{\milli\meter}$.}
    \label{tab:HH1} %Helmholtz 1
    \begin{tabular}{S[table-format=3.1] S[table-format=2.2] S[table-format=1.3]}
        \toprule
        {$x\,/\,\si{\milli\m}$} & {$y\,/\,\si{\milli\m}$} & {$B\,/\,\si{\milli\tesla}$} \\
        \midrule
        7    & -7.75 & 4.239 \\
        9    & -5.75 & 4.231 \\
        10.5 & -4.25 & 4.234 \\
        12   & -2.75 & 4.260 \\ 
        13.5 & -1.25 & 4.239 \\
        68.5 & 53.75 & 3.003 \\
        69   & 54.25 & 2.960 \\
        70   & 55.25 & 2.891 \\
        75   & 60.25 & 2.666 \\
        80   & 65.25 & 2.445 \\
        85   & 70.25 & 2.219 \\
        90   & 75.25 & 2.018 \\
        95   & 80.25 & 1.839 \\
        100  & 85.25 & 1.662 \\
        \bottomrule
    \end{tabular}
\end{table}

\begin{table}
    \centering
    \caption{tab:2. Messreihe mit einem Abstand von $d=\SI{104}{\milli\meter}$.}
    \label{tab:HH2} %Helmholtz 2
    \begin{tabular}{S[table-format=3.1] S[table-format=3.2] S[table-format=1.3]}
        \toprule
        {$x\,/\,\si{\milli\m}$} & {$y\,/\,\si{\milli\m}$} & {$B\,/\,\si{\milli\tesla}$} \\
        \midrule
        7     & -28.5 & 3.091 \\
        16    & -19.5 & 2.976 \\
        27    & -8.5  & 2.887 \\
        33.5  & -2    & 2.882 \\
        44    & 8.5   & 2.945 \\
        54.5  & 19    & 3.081 \\
        108.5 & 73    & 2.639 \\
        115   & 79.5  & 2.410 \\
        120   & 84.5  & 2.194 \\
        125   & 89.5  & 2.031 \\
        130   & 94.5  & 1.849 \\
        160   & 124.5 & 1.036 \\
        190   & 154.5 & 0.615 \\
        230   & 194.5 & 0.366 \\
        \bottomrule
    \end{tabular}
\end{table}

\begin{table}
    \centering
    \caption{tab:3. Messreihe mit einem Abstand von $d=\SI{130}{\milli\meter}$.}
    \label{tab:HH3} %Helmholtz 3
    \begin{tabular}{S[table-format=3.0] S[table-format=3.2] S[table-format=1.3]}
        \toprule
        {$x\,/\,\si{\milli\m}$} & {$y\,/\,\si{\milli\m}$} & {$B\,/\,\si{\milli\tesla}$} \\
        \midrule
        7   & -41.5 & 2.754 \\
        22  & -26.5 & 2.420 \\
        30  & -18.5 & 2.288 \\
        45  & -3.5  & 2.199 \\
        51  & 2.5   & 2.211 \\
        63  & 14.5  & 2.354 \\
        79  & 30.5  & 2.688 \\
        134 & 85.5  & 2.529 \\
        149 & 100.5 & 2.007 \\
        155 & 106.5 & 1.789 \\
        163 & 114.5 & 1.540 \\
        171 & 122.5 & 1.315 \\
        179 & 130.5 & 1.128 \\
        190 & 141.5 & 0.909 \\
        208 & 159.5 & 0.663 \\
        \bottomrule
    \end{tabular}
\end{table}