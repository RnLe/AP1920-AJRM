\section{Durchführung}
\label{sec:Durchführung}
\subsection{Übersicht}
Es werden insgesamt 4 Spulen untersucht. Ein Helmholtz-Spulenpaar, eine lange und eine kurze Spule, und eine Toroidspule mit Eisenkern.
Für die ersten drei Spulen(anordnungen) wird ein konstanter Strom eingestellt, um die erzeugten, magnetischen Felder auf Homogenität und Randeffekte zu untersuchen.
Für die Toroidspule wird der Strom variiert, um eine Hysteresekurve zu messen.

\subsection{Helmholtz-Spulenpaar}
Bei dem Helmholtz-Spulenpaar wird vor allem die physikalische Besonderheit untersucht, dass das Magnetfeld mittig zwischen den beiden Spulen und auf
ihrer gemeinsamen, rotationssymetrischen Achse, näherungsweise homogen ist. Dafür muss der Spulenabstand dem Radius der Spulen entsprechen.

Zunächst werden die Spulen auf der vorgegebenen Halterung ausgerichet, wobei eine Spule fest, und die andere frei beweglich ist. Die Apparatur besitzt zusätzlich zu der unteren Schiene, 
auf der eine Spule frei verschoben werden kann, eine weitere Schiene oberhalb der Spulen, an der ein kleiner Kopf befestigt ist, an dem wiederum die Hall-Sonde befestigt werden kann.
Beide Schienen besitzen ein eigenes, fest verbautes Lineal, sodass der Spulen- sowie der Sondenabstand genau eingestellt werden können.
Es werden 3 Spulenabstände eingestellt, und zu jeder Einstellung ungefähr 15 Messwerte aufgenommen, indem die Hall-Sonde verschoben wird. Dabei wird die Hall-Sonde für die eine Hälfte der Messwerte innerhalb,
für die andere Hälfte außerhalb des Spulenpaares gesetzt.
Für diese Anordnung wird die transversale Hall-Sonde verwendet. Das quadratische Plättchen an der Spitze der Sonde wird möglichst mittig und orthogonal auf die gemeinsame Symmetrieachse der Spulen platziert.
Wenn alles eingestellt und ausgerichtet ist, werden die Spulen in Reihe verkabelt. Erst dann wird die Stromquelle eingeschaltet, wobei zu beachten ist, dass alle Regler auf Null gesetzt sind.
Unter Berücksichtigung des maximal zulässigen Stroms wird die Spannung bzw. der Strom nun so weit erhöht, bis ein Wert unter dieser Grenze erreicht ist.
Anschließend wird das Messgerät für das Magnetische Feld eingeschaltet. Über den Knopf \texttt{Range} kann eingestellt werden, welche Stellen des Messwertes angezeigt werden sollen.
Nun werden die Messungen wie oben beschrieben vorgenommen.


\subsection{Kurze Spule}
Für die kurze Spule wird die longitudinale Hall-Sonde verwendet. Die Sonde wird an einem Stativ befestigt, an dem sich auch ein fest verbautes Lineal befindet.
Die Spule wird auf das Lineal gelegt und die Sonde mithilfe der Stellschrauben des Stativs so ausgerichtet, dass sie möglichst mittig in die Spule zeigt, also wieder auf der Symmetrieachse liegt.
Es werden ungefähr 12 Messungen aufgenommen, die sich darin unterscheiden, wie weit die Sonde in die Spule zeigt. Es werden auch Bereiche außerhalb der Spule gemessen, 
um Randeffekte zu untersuchen.
Das Anschließen an den Strom verhält sich ähnlich wie das des Helmholtz-Spulenpaares. Die kurze Spule wird an die ausgeschaltete Stromquelle angeschlossen. Daraufhin wird der Strom bei Null-geschalteten
Reglern eingeschaltet und dann so weit erhöht, bis der Strom unter der Maximalstromgrenze der Spule liegt.
Daraufhin werden die Messungen aufgenommen.

\subsection{Lange Spule}
Die Messungen an der langen Spule unterscheiden sich zum Aufbau der kurzen Spule nicht. Lediglich der einzustellende Maximalstrom ist unterschiedlich und genau zu beachten.
Ansonsten bleiben Durchführung, die verwendete Hall-Sonde und die Anzahl der Messwerte gleich.

\subsection{Toroidspule mit Eisenkern}
Der Aufbau der Toroidspule ist vollständig vorbereitet. Es müssen lediglich Stromquelle und Messgerät eingeschatet werden, wenn dies nicht schon ebenfalls vorbereitet wurde.
Außerdem muss die Spule mit zwei Kabeln an die Stromquelle verbunden werden.
Bei dem Aufbau ist eine transversale Hall-Sonde über der Spule befestigt, welche durch einen Spalt in Spule und Eisenkern führt und damit genau mittig ausgerichtet ist.
Für die Messung einer Hysteresekurve muss der Strom zunächst bis unter die Maximalstromgrenze erhöht werden, um die Sättigung des Eisenkerns zu erreichen.
Hier beginnt die Messung. Im Anschluss wird der Strom in Inkrementen gesenkt und gemessen. Welche größe die Inkremente haben, ist frei zu wählen, sollten jedoch in einem
plausiblen Bereich liegen. 1-Ampere-Schritte sind bei 10 Ampere Maximalstrom ausreichend. Der Strom wird also um jeweils 1 Ampere gesenkt, wobei zu beachten ist, dass die Stromänderung immer
nur in eine Richtung erfolgt! Dies ist von außerordentlicher Wichtigkeit, da ansonsten die Hysteresekurve verfälscht ist. Der Strom sollte also langsam gesenkt werden, sodass der Strom nicht
versehentlich um einen großeren Schritt gesenkt wird. Falls dies doch der Fall ist, notiert man diesen Messwert und macht den nächsten Schritt kleiner. Aber in keinem Fall wird der Strom erhöht.
Wenn der Strom 0 erreicht, so notiert man diesen Messwert. Anschließend vertauscht man die Verkabelung der Spule. Nun wird der Strom von 0 in 1-Ampere-Schritten auf den zuvor gewählten Maximalstrom erhöht.
Es sollte auch der Strom notiert werden, an dem das Magnetfeld 0 wird. Ist der Wert des Anfangstroms erreicht, wird der gesamte Prozess wiederholt.
Die Messwerte bei umgekehrter Verkabelung erhalten ein negatives Vorzeichen. Mit diesen Messwerten zeichnet sich eine Hysteresekurve ab.