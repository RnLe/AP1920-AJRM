\section{Auswertung}
\label{sec:Auswertung}

\begin{table}
    \centering
    \caption{Messwerte aller Schwingungsdauern.}
    \label{tab:messSchwing}
    \begin{tabular}{c c c c}
        \toprule
        {T$_{zyl, gross}\:/\:\si{\second}$} & {T$_{zyl, klein}\:/\:\si{\second}$} & {T$_{pose1}\:/\:\si{\second}$} & {T$_{pose2}\:/\:\si{\second}$} \\
        \midrule
        1.08 & 2.31 & 0.38 & 0.92 \\
        1.27 & 2.23 & 0.41 & 0.89 \\
        1.13 & 2.17 & 0.43 & 0.94 \\
        1.18 & 2.30 & 0.39 & 0.91 \\
        1.16 & 2.26 & 0.41 & 0.91 \\
        \bottomrule
    \end{tabular}
\end{table}

\begin{table}
    \centering
    \caption{Messwerte zur Bestimmung der Eigenträgheit.}
    \label{tab:messEigen}
    \begin{tabular}{c c c}
        \toprule
        {T$_{eigen}\:/\:\si{\second}$} & {a$_{m1}\:/\:\si{\centi\meter}$} & {a$_{m2}\:/\:\si{\centi\meter}$} \\
        \midrule
        2.50 & 4.5  & 5.5  \\
        2.93 & 6.5  & 7.5  \\
        3.21 & 8.5  & 9.5  \\
        3.83 & 10.5 & 11.5 \\
        4.16 & 12.5 & 13.5 \\
        4.70 & 14.5 & 15.5 \\
        5.27 & 16.5 & 17.5 \\
        5.79 & 18.5 & 19.5 \\
        6.27 & 20.5 & 21.5 \\
        6.78 & 22.5 & 23.5 \\
        \bottomrule
    \end{tabular}
\end{table}

\begin{table}
    \centering
    \caption{Messunsicherheiten aller Schwingungsdauern.}
    \label{tab:mittelSchwing}
    \begin{tabular}{l c}
        \toprule
        & {T$\:/\:\si{\second}$} \\     % werden Messgrößen kursiv geschrieben? (hier das T)
        \midrule
        Zylinder$_{gross}$  & 2.25$\pm$0.05 \\
        Zylinder$_{klein}$  & 1.16$\pm$0.06 \\
        Holzfigur Pose 1    & 0.40$\pm$0.02 \\
        Holzfigur Pose 2    & 0.91$\pm$0.02 \\
        \bottomrule
    \end{tabular}
\end{table}