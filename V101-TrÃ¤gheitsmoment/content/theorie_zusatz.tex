\subsection{Trägheitsmoment spezieller Körpers}
\label{sub:Trägheitsmoment}

\paragraph{Abgeschnitte Kugel}

Für die Auswertung bestimmter Körper sei hier noch ein etwas speziellerer Körper vorgestellt, mit seiner Masse und seinem 
Trägheitsmoment. 
Es handelt sich um eine Kugel mit Radius $R$, die oben und unten zu gleichen Anteilen \glqq abgeschnitten\grqq{} ist, sodass 
sie die Höhe ${h<R}$ besitzt. 
Die verwendeten Kugelkoordinaten lauten 
\begin{equation}
    (x,y,z)=(r\cos \varphi \sin \vartheta, r\sin \varphi \sin \vartheta, r\cos \vartheta) \,.
\end{equation}
Zur Berechnung des Volumens wird der Körper in zwei Zylinder und einen Kugelkörper, für dessen Polarwinkel 
${\arccos (\frac{h}{2R})\le\vartheta\le\symup{\pi}-\arccos{\frac{h}{2R}}}$ gilt, aufgeteilt. 
Die Zylinder haben eine Höhe von ${\sfrac{h}{2}}$ und einen Radius von ${a=\sqrt{R^2-\sfrac{h^2}{4}}}\,$. 
\begin{align}
    V_1&=\int_{r=0}^R \int_{\varphi=0}^{2\symup{\pi}} \int_{\vartheta=\arccos{\frac{h}{2R}}}^{\symup{\pi}-\arccos{\frac{h}{2R}}} 
        r^2\sin{\vartheta} \, \symup{d}\vartheta \, \symup{d}\varphi \, \symup{d}r \\
        &= (\frac{1}{3}R^3) (2\frac{h}{2R}) (2\symup{\pi}) \\
        &= \frac{2}{3}\symup{\pi}hR^2 \\
    V_2&= 2\cdot\frac{1}{3} X_\text{Grundfläche} Y_\text{Höhe} \\
        &= 2\cdot\frac{1}{3} \bigl(\symup{\pi} (R^2-\Bigl(\frac{h}{2}\Bigr)^2)\bigr) \frac{h}{2} \\
        &= \frac{1}{3}\symup{\pi}h(R^2-\frac{h^2}{4}) = \symup{\pi}h(\frac{R^2}{3}-\frac{h^2}{12}) \\
    V&=V_1+V_2=\symup{\pi}h(R^2-\frac{h^2}{12})
\end{align}
\begin{align}
    I_1&=\rho _0 \int_V r_{\perp}^2 \, \symup{d}V 
        \quad \quad \quad |\quad r_{\perp}^2=x^2+y^2=r^2\sin ^2\vartheta \\
        &= \rho _0  \int_{r=0}^R \int_{\varphi=0}^{2\symup{\pi}} \int_{\vartheta=\arccos{\frac{h}{2R}}}^{\symup{\pi}-\arccos{\frac{h}{2R}}} 
        r^4\sin{\vartheta}(1-\cos ^2\vartheta) \, \symup{d}\vartheta \, \symup{d}\varphi \, \symup{d}r \\
        &= \rho _0 \, 2\symup{\pi} \, \frac{1}{5} R^5 \bigl[-\cos{\vartheta}+\frac{1}{3}\cos ^2\vartheta\bigr]_{\arccos \frac{h}{2R}}^{\symup{\pi}-\arccos \frac{h}{2R}} \\
        &= \frac{m_1}{\frac{2\symup{\pi}}{3}hR^2} \frac{2\symup{\pi}}{5} R^5 \Bigl(\frac{h}{R} + \frac{h^3}{12R^3}\Bigr) \\
        &= \frac{3}{5}m_1(R^2+\frac{h^2}{12}) \\
    I_\text{Kegel}&=\frac{3}{10}mr^2 \\
    I_2&=\frac{3}{5}m_2(R^2-\frac{h^2}{4}) \\
\end{align}
\begin{align}
    m_1&=\frac{V_1}{V}m=\frac{\frac{2}{3}\symup{\pi}hR^2}{\symup{\pi}h(R^2-\frac{h^2}{12})}m=\frac{8R^2}{12R^2-h^2}m \\
    m_2&=\frac{V_2}{V}m=\frac{\frac{1}{3}\symup{\pi}h(R^2-\frac{h^2}{4})}{\symup{\pi}h(R^2-\frac{h^2}{12})}m=\frac{4R^2-h^2}{12R^2-h^2}m \\
    I_1&=\frac{3}{5\cdot12}(12R^2+h^2)\frac{8R^2}{12R^2-h^2}m=\frac{2}{5}\cdot \frac{12R^2+h^2}{12R^2-h^2}R^2m \\
    I_2&=\frac{3}{5\cdot4}(4R^2-h^2)\frac{4R^2-h^2}{12R^2-h^2}m=\frac{3}{20}\frac{(4R^2-h^2)^2}{12R^2-h^2}m \\
    I&=I_1+I_2=\frac{2}{5}\frac{12R^2+h^2}{12R^2-h^2}R^2m+\frac{3}{20}\frac{(4R^2-h^2)^2}{12R^2-h^2}m \\
        &= \frac{1}{12R^2-h^2}\frac{m}{20}\bigl(8\cdot(12R^2+h^2)R^2+3\cdot(4R^2-h^2)^2\bigr) 
\end{align}

\paragraph{Kegelstumpf}

Der Kegelstumpf habe die Höhe $h$ und die Radien $R_\text{max}$ und $R_\text{min}$. 
Für die folgenden Betrachtungen sei an dieser Stelle festgelegt, dass die Größen bezüglich des Kegels mit Radius ${r=R_\text{max}}$
und mit dem gleichen Neigungswinkel der Mantelfläche wie beim Kegelstumpf mit dem Index \glqq groß\grqq{} versehen werden. 
Dem Kegel mit ${r=R_\text{min}}$ und ebenfalls gleichem Neigungswinkel sei der Index \glqq diff\grqq{} zugehörig. 

Mithilfe der Kongruenzsätze lassen sich die Volumina von beiden Kegeln berechnen, sowie die Massen dieser:

\begin{align}
    h_\text{groß}&= \frac{h}{R_\text{max}-R_\text{min}} R_\text{max} \\
    h_\text{diff}&= \frac{h}{R_\text{max}-R_\text{min}} R_\text{max}-h \\
    V_\text{groß}&= \frac{1}{3} (\symup{\pi} R_\text{max}^2) (\frac{h}{R_\text{max}-R_\text{min}} R_\text{max}) \\
    V_\text{diff}&= \frac{1}{3} (\symup{\pi} R_\text{min}^2) (\frac{h}{R_\text{max}-R_\text{min}} R_\text{max}-h) \\
    V_\text{Stumpf}&= V_\text{groß}-V_\text{diff} \\
    m_\text{groß}&= \frac{V_\text{groß}}{V_\text{Stumpf}}m_\text{Stumpf} \\
    m_\text{diff}&= \frac{V_\text{diff}}{V_\text{Stumpf}}m_\text{Stumpf}
\end{align}

Werden diese Formeln nun in die folgenden Gleichungen eingesetzt, kann auf die Gleichung des Trägheitsmoments eines 
Kegelstumpfs geschlussfolgert werden. 

\begin{align}
    I_\text{Kegel}&=\frac{3}{10}mR^2 \\
    I_\text{groß}&=\frac{3}{10}m_\text{groß}R_\text{max}^2 \\
    I_\text{diff}&=\frac{3}{10}m_\text{diff}R_\text{min}^2 \\
    I_\text{Stumpf}&=I_\text{groß}-I_\text{diff}
\end{align}
