\section{Durchführung}
\label{sec:Durchführung}

\subsection{Einleitung}
Ziel des Versuchs ist es, die Trägheitsmomente verschiedener Körper zu bestimmen. Hierfür wird ein Stativ verwendet, an dem eine Spiralfeder angebracht ist.
Die Spiralfeder ist nach oben beziehungsweise unten ausgerichtet. In der Mitte der Feder befindet sich eine zylinderförmige Halterung, welche nach oben gerichtet ist und sich frei bewegen kann.
Das äußere Ende ist an dem Stativ angebracht, welches U-förmig um die Spiralfeder herum führt. Auf dem Stativ befindet sich eine Lochscheibe, an der man Winkel ablesen kann.
Durch das Loch führt die Halterung, welche mittig aus der Feder hervorgeht. Mit dieser Konfiguration können Objekte an der Halterung angebracht und Auslenkungen an 
der Lochscheibe abgelesen werden.

\subsection{Apparatekonstante}
Zu Beginn werden Messungen durchgeführt, um die Apparatekonstantezu bestimmen.
Dazu wird eine lange, leichte Stange orthogonal an der Halterung befestigt. Daraufhin wird die Winkelscheibe genau ausgerichtet, sodass sich Auslenkungen genau ablesen lassen.
Mit einer Federwaage, welche Kräfte bis zu \SI{1}{\newton} misst, wird die Stange nun an einer anfangs beliebigen Stelle ausgelenkt. Dabei ist es wichtig, dass die Federwaage
orthogonal an der Stange zieht, und nicht abgewinkelt. Es wird das Wertepaar des Winkels und der Kraft aufgeschrieben und die Messungen insgesamt zehn mal durchgeführt.
Variiert werden die Auslenkung beziehungsweise die Kraft, jedoch nicht der Abstand zur Drillachse. Es werden also alle Messungen mit demselben Abstand durchgeführt.

\subsection{Eigenträgheit}
Zur Bestimmung des Trägheitheitsmomentes der Drillachse selbst werden zwei Gewichte verwendet, dessen Masse zunächst gewogen wird.
Die Massen sind Hohlkegel und besitzen eine Feststellschraube, um sie auf der Stange fixieren zu können.
Es wird an je einer Seite der Stange eine Masse angebracht. Wenn sie fixiert sind wird der Abstand zur Drillachse notiert und die Stange anschließend sehr leicht,
um etwa \ang{10;;}-\ang{20;;} ausgelenkt. Nach dem Loslassen wird die Schwingungsdauer der Stange gemessen.
Dies wird insgesamt zehn mal durchgeführt, wobei der Abstand der Massen zur Drillachse variiert wird.

\subsection{Trägheit der Körper}
\subsubsection{Großer Zylinder}
