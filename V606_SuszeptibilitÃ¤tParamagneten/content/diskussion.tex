\section{Diskussion}
\label{sec:Diskussion}

Die Bestimmung der Durchlassfrequenz ist recht eindeutig, da die umliegenden Werte bei $\nu=\SI{35.0}{\kilo\hertz}$ 
deutlich unter dem Maximum liegen. 
Das Wechseln der Messbereiche bei dem verwendeten Voltmeter schlägt sich in dem abrupten leichten Abfallen oder Steigen 
der Spannung nieder, welche im Prinzip im Bereich vor dem Maximum und im Bereich nach dem Maximum monoton steigen beziehungsweise 
fallen sollte. 
Durch Wechsel des Messbereichs wird in dem Voltmeter ein anderer Innenwiderstand geschaltet, und sollten diese nicht 
so fein abgestimmt sein, wie theoretisch zu erwarten, hat das die beschriebenen Variationen zur Folge. 
Diese Beobachtung zeigt, dass die mit dem Voltmeter gemessenen Werte bereits ein gewisses Maß an Unsicherheit haben. 
Die Durchlassfrequenz kann dennoch als valide angenommen werden, da das Maximum und direkt umliegende Werte 
in ein und demselben Messbereich aufgenommen sind. 

Auffällig bei der Auswertung der Suszeptibilitäten ist die Abweichungen von mehreren Größenordnungen von den Theoriewerten in \eqref{eqn:chi1} bis \eqref{eqn:chi3}. 
Wird sie mithilfe der Spannung berechnet, reduziert sie sich um den Faktor $\num{e-1}$ bis $\num{e-2}$; 
über die Widerstände vergrößern sie sich um den Faktor $\num{e2}$. 
Es ist recht offensichtlich, dass hier ein systematischer Messfehler vorliegt und dies keine statistisch zu erklärende Streuung ist.

Aufgrund der Annahme, dass $\Delta L\ll L$ ist, ist zu erwarten, dass bei der Messung ohne Füllung die Brückenwiderstände 
in etwa gleich sind. Mit Blick auf \ref{tab:Messwerte} bestätigt sich dies, die Widerstände weichen um $3.2\%$ von dem 
idealen Wert $\SI{2.5}{\ohm}$ ab. 
Dementsprechend muss die Erhöhung des Widerstands $R_3$ bei Einführen von Materie -- und damit $\Delta R$ -- fehlerbehaftet und dadurch größer als erwartet sein. 
Da die Spulen in dem Schaltkasten \ref{fig:Bruecke} verborgen sind, kann von außen nicht beurteilt werden, ob möglicherweise 
sichtbare Schäden an den Spulen oder an dem Regelwiderstand zu sehen sind. 
Die Spulen sind wie beschrieben sehr empfindlich und könnten vielleicht bei vorherigen Verwendungen beschädigt worden sein. 
Da die Auswirkungen auf die Messwerte von den genauen Schäden abhängig sind, können an dieser Stelle nur schwerlich Rückschlüsse 
auf ausgleichende Fehlerrechnungen gezogen werden. 

Ebenso verhält es sich an dieser Stelle mit den Werten, die mithilfe der Brückenspannung berechnet werden. 
