%\addbibresource{content/quellen.bib}
\section{Vorbereitungsaufgaben}
\label{sec:Vorbereitungsaufgaben}
Im Vorfeld sollten die Dichte $\rho$, die spezifische Wärme $c$ und die Wärmeleitfähigkeit $\kappa$ von Aluminium, Messing
und Edelstahl recherchiert werden. 
\begin{table}
    \centering
    \caption{Literaturwerte (Umgebungstemperatur $\SI{20}{\celsius}$)}
    \label{tab:Literaturwerte}
    \begin{tabular}{S | S[table-format=1.3] S[table-format=1.3] S[table-format=3.3]}
        \toprule
        {Stoff} & {$\rho [\si{\gram\per\cubic\centi\meter}] $} & {$c [\si{\kilo\joule\per{\kilo\gram\kelvin}}]$} 
        & {$\kappa [\si{\watt\per{\meter\kelvin}}]$} \\
        \midrule
        %Aluminium \cite{taschenbuch} & 2.70 & 0.920 & 221 \\
        %Messing \cite{formelsammlung} & 8.6 &  0.375 & 112 \\
        %Edelstahl \cite{taschenbuch} & 7.84 & 0.460 & 46 \\
        %Wasser \cite{taschenbuch} & 0.998 & 4.19 & 0.600 \\ 
        \bottomrule
    \end{tabular}
\end{table}