\section{Messdaten}
\label{sec:Messdaten}
\begin{table}
    \centering
    \caption{Messreihe 1 - Statische Methode}
    \label{tab:data1}
    \begin{tabular}{S[table-format=3.1, round-mode=places, round-precision=1] S[table-format=2.2] S[table-format=2.2] S[table-format=2.2] S[table-format=2.2] S[table-format=2.2] S[table-format=2.2] S[table-format=2.2] S[table-format=2.2]}
        \toprule
         & \multicolumn{2}{c}{Messing(breit)} & \multicolumn{2}{c}{Messing(schmal)} & \multicolumn{2}{c}{Aluminium} & \multicolumn{2}{c}{Edelstahl} \\
        \cmidrule(lr){2-3}\cmidrule(lr){4-5}\cmidrule(lr){6-7}\cmidrule(lr){8-9}
        {$t[\si{\second}]$} & {$T_{1, \text{fern}}[\si{\celsius}]$} & {$T_{2, \text{nah}}[\si{\celsius}]$} & {$T_{3, \text{nah}}[\si{\celsius}]$} & {$T_{4, \text{fern}}[\si{\celsius}]$} & {$T_{5, \text{fern}}[\si{\celsius}]$} & {$T_{6, \text{nah}}[\si{\celsius}]$} & {$T_{7, \text{nah}}[\si{\celsius}]$} & {$T_{8, \text{fern}}[\si{\celsius}]$}  \\
        \midrule
        0.0   & 23.64 & 24.10 &	24.27 &	23.69 &	23.75 &	24.72 &	23.71 &	23.68 \\
        5.0   & 23.71 & 24.90 &	25.21 &	23.79 &	24.18 &	26.03 &	23.88 &	23.69 \\
        10.0  & 23.88 & 25.79 &	26.27 &	23.99 &	24.79 &	27.31 &	24.22 &	23.68 \\
       $\vdots$ & $\vdots$ & $\vdots$ & $\vdots$ & $\vdots$ & $\vdots$ & $\vdots$ & $\vdots$ & $\vdots$ \\
        690.0 & 46.36 & 49.27 & 47.95 & 44.53 & 50.04 & 51.60 & 44.98 & 35.64 \\	
        \bottomrule
    \end{tabular}
\end{table}

\begin{table}
    \centering
    \caption{Messreihe 2 - Dynamische Methode}
    \label{tab:data2}
    \begin{tabular}{S[table-format=3.1, round-mode=places, round-precision=1] S[table-format=2.2] S[table-format=2.2] S[table-format=2.2] S[table-format=2.2] S[table-format=2.2] S[table-format=2.2] S[table-format=2.2] S[table-format=2.2]}
        \toprule
        & \multicolumn{2}{c}{Messing(breit)} & \multicolumn{2}{c}{Messing(schmal)} & \multicolumn{2}{c}{Aluminium} & \multicolumn{2}{c}{Edelstahl} \\
        \cmidrule(lr){2-3}\cmidrule(lr){4-5}\cmidrule(lr){6-7}\cmidrule(lr){8-9}
        {$t$} & {$T_{1, \text{fern}}$} & {$T_{2, \text{nah}}$} & {$T_{3, \text{nah}}$} & {$T_{4, \text{fern}}$} & {$T_{5, \text{fern}}$} & {$T_{6, \text{nah}}$} & {$T_{7, \text{nah}}$} & {$T_{8, \text{fern}}$} \\
        \midrule
        0.000 & 33.08 &	36.21 &	36.46 &	32.47 &	34.62 &	37.16 &	33.62 &	29.54 \\
        0.500 & 33.10 &	36.25 &	36.48 &	32.50 &	34.66 &	37.19 &	33.65 &	29.55 \\
        1.000 & 33.12 &	36.27 &	36.51 &	32.52 &	34.69 &	37.25 &	33.68 &	29.54 \\
        $\vdots$ & $\vdots$ & $\vdots$ & $\vdots$ & $\vdots$ & $\vdots$ & $\vdots$ & $\vdots$ & $\vdots$ \\
        882.00 & 65.16 & 65.67 & 62.65 & 61.61 & 67.34 & 65.75 & 62.62 & 50.17 \\
        \bottomrule
    \end{tabular}
\end{table}

\begin{table}
    \centering
    \caption{Messreihe 3 - Dynamische Methode - Angström}
    \label{tab:data3}
    \begin{tabular}{S[table-format=2.2] S[table-format=2.2] S[table-format=2.2] S[table-format=2.2] S[table-format=2.2] S[table-format=2.2] S[table-format=2.2] S[table-format=2.2] S[table-format=3.1, round-mode=places, round-precision=1]}
        \toprule
        \multicolumn{2}{c}{Messing(breit)} & \multicolumn{2}{c}{Messing(schmal)} & \multicolumn{2}{c}{Aluminium} & \multicolumn{2}{c}{Edelstahl} \\
        \cmidrule(lr){1-2}\cmidrule(lr){3-4}\cmidrule(lr){5-6}\cmidrule(lr){7-8}
        {$T_{1, \text{fern}}$} & {$T_{2, \text{nah}}$} & {$T_{3, \text{nah}}$} & {$T_{4, \text{fern}}$} & {$T_{5, \text{fern}}$} & {$T_{6, \text{nah}}$} & {$T_{7, \text{nah}}$} & {$T_{8, \text{fern}}$} & {$t$} \\
        \midrule
        29.48 &	28.85 &	28.20 &	28.60 &	28.13 &	26.75 &	28.24 &	27.16 &	0.0000 \\
        29.43 &	28.54 &	27.85 &	28.55 &	27.93 &	26.51 &	28.16 &	27.16 &	2.0000 \\
        29.37 &	28.51 &	27.85 &	28.49 &	27.76 &	27.05 &	28.06 &	27.15 &	4.0000 \\
        $\vdots$ & $\vdots$ & $\vdots$ & $\vdots$ & $\vdots$ & $\vdots$ & $\vdots$ & $\vdots$ & $\vdots$ \\
        59.88 &	58.92 &	56.03 &	56.24 &	58.68 &	57.55 &	55.66 &	50.00 &	1400.0000 \\        
        \bottomrule
    \end{tabular}
\end{table}

Für die Auswertung wird ebenfalls der Abstand $\increment x$ zwischen den beiden Temperatursensoren einer Probe benötigt. 
Hier wird der gemessene Wert von $\increment x = \SI{3.0}{\centi\meter}$ verwendet. 