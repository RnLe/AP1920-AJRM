\section{Diskussion}
\label{sec:Diskussion}
Für die drei Metalle Aluminium, Messing und Edelstahl wurden in diesem Versuch mithilfe der Angström-Methode 
die jeweiligen Wärmeleitfähigkeiten $\kappa$ bestimmt.
Für Messing wurde ein Wert von $\kappa = \SI{75 \pm 22}{\watt\per\meter\per\kelvin}$ ermittelt, der Literaturwert liegt 
hingegen mit $\SI{112}{\watt\per\meter\per\kelvin}$ nicht im berechneten Fehlerintervall. 
Dies könnte darin begründet sein, dass Messing an sich kein reines Element ist, sondern teilweise auch stark schwankende 
Mischverhältnisse verschiedener Reinmetalle enthalten kann. 
Somit ist Messing an sich nicht eindeutig identifizierbar und weist je nach Mischverhältnis verschiedene 
Materialeigenschaften auf. 
Dies bezieht sich ebenfalls auf die Wärmeleitfähigkeit: Diese hängt von den verschiedenen, in dem verwendeten 
Messing enthaltenen Metallen ab, die nicht angegeben waren. So bezieht sich der Literaturwert vermutlich auf eine andere
Messingzusammensetzung, als in dem Experiment verwendet wurde. 

Aluminium hat gemäß der Auswertung eine Wärmeleitfähigkeit von \\
$\kappa = \SI{206 \pm 33}{\watt\per\meter\per\kelvin}$.
Der Literaturwert von $\SI{221}{\watt\per\meter\per\kelvin}$ liegt somit im berechneten Fehlerintervall.

Edelstahl besitzt wiederum laut Rechnung eine Wärmeleitfähigkeit von \\
$\kappa = \SI{11.0 \pm 1.7}{\watt\per\meter\per\kelvin}$, 
was im Widerspruch zum Literaturwert $\SI{46}{\watt\per\meter\per\kelvin}$ steht. 
Wieder ist hier die nicht eindeutige Definition des Stoffs Edelstahls als Ursache möglich. 
Edelstahl ist eine besonders reine Form des Stahls, der zu großen Teilen aus Eisen besteht. 
Jedoch sorgen auch hier verschiedene Legierungen für unterschiedliche Stoffzusammensetzungen, die die Wärmeleitfähigkeit 
von Produkt zu Produkt schwanken lassen können. 

Eine weitere Beobachtung ist die um einiges stärkere Wärmeleitung des Aluminiums: Es erreicht insgesamt eine höhere Temperatur
als die anderen Metalle. Messing, aber vor allem Edelstahl bestehen meist aus edleren Metallen, die weniger freie Elektronen 
aufweisen als Unedlere, wie Aluminium eins ist. Durch weniger freie Elektronen kann tendenziell weniger thermische 
Energie transportiert werden, wie bereits in Kapitel \texttt{1 Theorie} erwähnt wird. 
