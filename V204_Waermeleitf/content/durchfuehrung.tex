\section{Durchführung}
\label{sec:Durchführung}
Ziel des Experiments ist es, die Wärmeleitung exemplarisch an Aluminium, Messing und Edelstahl zu untersuchen. 
\subsection{Versuchsaufbau}
Auf einer Messplatine sind die verschiedenen Metallstäbe befestigt: 
Jeweils ein Aluminium- und ein Edelstahlstab und zwei Messingstäbe, die sich allein durch ihre Querschnittsfläche unterscheiden.
Es gibt acht Temperaturmessstellen, je zwei pro Stab, die sich in einem Abstand von $\SI{3}{\centi\meter}$ voneinander befinden. 
Je ein Ende wird durch das mittig platzierte Peltier-Element erwärmt oder gekühlt. 
Die Platine wird über einen Temperatur-Array mit dem Xplorer GLX verbunden. 
Das Temperatur-Array ist dafür da, die jeweiligen Temperatursensoren zu identifizieren, sodass die gemessenen Werte den richtigen Sensoren zugeordnet werden.
Die Messwerte werden über den Xplorer GLX, der mit dem Temperatur-Array verbunden ist, aufgenommen und gespeichert. 
Das Peltier-Element wird durch einen Spannungsquelle betrieben, bei der die Spannung reguliert werden kann. 
Eine Wärmeisolierung sorgt dafür, dass der Wärmeaustausch der Metalle mit der Umgebung möglichst gering bleibt.
\subsection{Durchführung}
Es werden insgesamt drei Messreihen aufgenommen: Die Erste mit der statischen, die beiden Folgenden mit der dynamischen Methode.
