\section{Auswertung}
\label{sec:Auswertung}
Gemäß \cite{Versuchsanleitung} wird für die Referenzbauteile, wenn nicht anders erwähnt, ein Fehler von $\pm{0.2\%}$ angenommen. 
Das verwendete Potentiometer hat eine Toleranz von $\pm{0.5\%}$ -- hiermit sind meist die Widerstände $R_3$ und $R_4$ eingestellt worden.
Die Fehlerrechnungen sind jeweils mit iPython 7.8.0 mit dem Paket \textit{uncertainties} durchgeführt worden.

Aus einigen Messungen resultieren mehrere Messergebnisse mit jeweiliger Unsicherheit. Um ein Endresultat der Größe aus der 
den Messungen zu erhalten, werden diese mittels 
\begin{gather}
    \langle x \rangle = \frac{1}{N} \sum_{i=1}^N x_i \\
    \increment x = \sqrt{\frac{1}{N-1} \sum_{i=1}^N (x_i - \langle x \rangle)^2}
\end{gather}
zu einem endgültigen Ergebnis zusammengefasst.

\FloatBarrier
\subsection{Messung mit der Wheatstone'schen Brücke}
\FloatBarrier
    \begin{table}
        \centering
        \caption{Messdaten für die Wheatstone'sche Brückenschaltung.}
        \label{tab:wheat}
        \begin{tabular}{c S[table-format=4.1] @{${}\pm{}$} S[table-format=1.1] S[table-format=3.0] @{${}\pm{}$} S[table-format=1.0] S[table-format=3.0] @{${}\pm{}$} S[table-format=1.0]}
            \toprule
            {$R_\text{x}$} & \multicolumn{2}{c}{$R_2 \:/\: \si{\ohm}$} & \multicolumn{2}{c}{$R_3 \:/\: \si{\ohm}$} & \multicolumn{2}{c}{$R_4 \:/\: \si{\ohm}$} \\
            \midrule
            Wert 12 & 332.0 & 0.7   & 542 & 3 & 458 & 2 \\
                    & 500.0 & 1.0   & 440 & 2 & 560 & 3 \\
                    & 1000  & 2     & 282 & 1 & 718 & 4 \\
            Wert 11 & 332.0 & 0.7   & 598 & 3 & 402 & 2 \\
                    & 500.0 & 1.0   & 497 & 2 & 503 & 3 \\
                    & 1000  & 2     & 330 & 2 & 670 & 3 \\
            \bottomrule
        \end{tabular}
    \end{table}
    Laut der Abgleichbedingung \ref{eqn:wheatstone}
    ergeben sich für die Werte 12 und 11 folgende Ergebnisse:
    \begin{table}
        \centering
        \caption{Messergebnisse der Wheatstone-Brücke.}
        \label{tab:resultwheat}
        \begin{tabular}{c c S[table-format=3.0] @{${}\pm{}$} S[table-format=1.0]}
            \toprule
            {Widerstand} & {Messung Nr.} & \multicolumn{2}{c}{$R_x \:/\: \si{\ohm}$} \\
            \midrule
            Wert 12 & 1 & 393 & 3 \\
                    & 2 & 393 & 3 \\
                    & 3 & 393 & 3 \\
            Endresultat&   & 393 & 2 \\
            Wert 11 & 1 & 494 & 4 \\
                    & 2 & 494 & 4 \\
                    & 3 & 493 & 4 \\
            Endresultat&   & 494 & 2 \\
            \bottomrule
        \end{tabular}
    \end{table}
\FloatBarrier

\subsection{Messung mit der Kapazitätsmessbrücke}
\FloatBarrier
    Die erste Kapazität (Wert 1) wurde mit einer Frequenz von $f=\SI{40}{\kilo\hertz}$, die zweite (Wert~3) mit $f=\SI{36}{\kilo\hertz}$ gemessen.
    \begin{table}
        \centering
        \caption{Messdaten für die Kapazitätsmessbrückenschaltung.}
        \label{tab:kapzmess}
        \begin{tabular}{c S[table-format=3.0] @{${}\pm{}$} S[table-format=1.0] S[table-format=3.0] @{${}\pm{}$} S[table-format=1.0] S[table-format=3.0] @{${}\pm{}$} S[table-format=1.0]}
            \toprule
            {$C_\text{x}$} & \multicolumn{2}{c}{$C_2 \:/\: \si{\nano\farad}$} & \multicolumn{2}{c}{$R_3 \:/\: \si{\ohm}$} & \multicolumn{2}{c}{$R_4 \:/\: \si{\ohm}$} \\
            \midrule
            Wert 1  & 450   & 1 & 408 & 1 & 592 & 1 \\
                    & 597   & 1 & 487 & 1 & 513 & 1 \\
                    & 992   & 2 & 611 & 1 & 389 & 1 \\
            Wert 3  & 597   & 1 & 595 & 1 & 405 & 1 \\
                    & 992   & 2 & 711 & 1 & 289 & 1 \\
                    & 750   & 2 & 643 & 1 & 377 & 1 \\
            \bottomrule
        \end{tabular}
    \end{table}
    \FloatBarrier
    Aus der Abgleichbedingung
    \begin{equation}
        C_x = C_2 \frac{R_4}{R_3}
    \end{equation}
    resultieren die in \ref{tab:kapz} dargestellten Werte.
     \begin{table}
        \centering
        \caption{Messergebnisse der Kapazitätsmessbrücke.}
        \label{tab:kapz}
        \begin{tabular}{c c S[table-format=3.0] @{${}\pm{}$} S[table-format=1.0]}
            \toprule
            {Kapazität} & {Messung Nr.} & \multicolumn{2}{c}{$C_x \:/\: \si{\nano\farad}$} \\
            \midrule
            Wert 1  & 1 & 653 & 2 \\
                    & 2 & 629 & 2 \\  
                    & 3 & 632 & 2 \\ 
            Endresultat& & 638 & 4 \\ 
            Wert 3  & 1 & 406 & 1 \\  
                    & 2 & 403 & 2 \\ 
                    & 3 & 440 & 2 \\ 
            Endresultat&   & 416 & 17 \\ 
            \bottomrule 
        \end{tabular}
    \end{table}
    Da sich bei der Durchführung dieser Messung mit einer RC-Kombination Schwierigkeiten auftaten, muss an dieser Stelle 
    auf eine Auswertung diesbezüglich verzichtet werden. 
    Die Praktikumsaufsicht zeigte sich damit einverstanden.

\subsection{Messung mit der Induktivitätsmess- und der Maxwell-Brücke}
\FloatBarrier
    Bei Messung der verlustbehafteten Induktivität (Wert 19) mit der Induktivitätsmessbrücke wurde eine Frequenz 
    von ${f=\SI{100}{\hertz}}$ benutzt, 
    bei der mit der Maxwell-Brücke ${f=\SI{40.2}{\kilo\hertz}}$.
    Für die Widerstände $R_3$ und $R_4$ wird bei der Maxwell-Brücke eine Toleranz von $\pm{3\%}$ angenommen \cite{Versuchsanleitung}.  
    \begin{table}
        \centering
        \caption{Messdaten zur Ermittlung der Induktivität mit Wert 19.}
        \label{tab:indumess}
        \begin{tabular}{c 
                        c        %S[table-format=2.1] 
                        c        %S[table-format=3.0] 
                        c        %S[table-format=3.1] 
                        c        %S[table-format=3.0]  
                        c}        %S[table-format=3.0]}
            \toprule
            {Messmethode} 
            & {$L_2 \:/\: \si{\milli\henry}$} 
            & {$C_4 \:/\: \si{\nano\farad}$} 
            & {$R_2 \:/\: \si{\ohm}$} 
            & {$R_3 \:/\: \si{\ohm}$} 
            & {$R_4 \:/\: \si{\ohm}$} \\
            \midrule
            Ind.-Br.    & $14.6\pm{0.03}$ & & $332.0\pm{0.7}$ & $255\pm{1}$ & $745\pm{4}$ \\
                        & $14.6\pm{0.03}$ & & $500.0\pm{1.0}$ & $185\pm{1}$ & $815\pm{4}$ \\
                        & $14.6\pm{0.03}$ & & $664  \pm{1  }$ & $145\pm{1}$ & $855\pm{4}$ \\
            Maxwell     & & $750\pm{2}$     & $664\pm{1}$     & $57\pm{2}$  & $270\pm{8}$ \\
            \bottomrule
        \end{tabular}
    \end{table}
    
    Mithilfe der Abgleichbedingungen für die Induktivitätsmessbrücke 
    \begin{align}
        R_x=R_2 \frac{R_3}{R_4}
        &L_x=L_2 \frac{R_3}{R_4}
    \end{align}
    und denen für die Maxwell-Brücke 
    \begin{align}
        R_x=R_2 \frac{R_3}{R_4}
        &L_x=C_4 R_2 R_3
    \end{align}
    ergibt sich für die verlustbehaftete Spule: 
    \begin{table}
        \centering
        \caption{Innenwiderstand und Induktivität der verwendeten Spule.}
        \label{tab:R_L_Spule}
        \begin{tabular}{c S[table-format=3.0] @{${}\pm{}$} S[table-format=1.1] S[table-format=2.2] @{${}\pm{}$} S[table-format=1.2]}
            \toprule
            {Messung Nr.} & \multicolumn{2}{c}{$R_x \:/\: \si{\ohm}$} & \multicolumn{2}{c}{$L_x \:/\: \si{\milli\henry}$} \\
            \midrule
            1           & 114 & 1 & 5.00 & 0.04 \\
            2           & 113 & 1 & 3.31 & 0.03 \\
            3           & 113 & 1 & 2.48 & 0.02 \\
            Endresultat & 113 & 0.6& 3.60 & 0.80 \\
            4 (Maxwell) & 140 & 6 & 28.4 & 1.0  \\ %Was soll das?? Warum ist der Wert einfach mal um eine Zehnerpotenz größer?? Hab alles nochmal nachgerechnet und keine Fehler gefunden...
            \bottomrule
        \end{tabular}
    \end{table}

\subsection{Frequenzabhängigkeit der Brückenspannung einer Wien-Robinson-Brücke} 
%ich finde nur was zu dem Erfinder Wien... über Robinson habe ich nichts rausfinden können. Keine Ahnung, welchen Bindestrich man dann verwenden soll
\FloatBarrier
Bei dieser Messung wurden ein Kondensator mit einer Kapazität von $C=\SI{660}{\nano\farad}$ und zwei ohmsche Widerstände 
$R=\SI{400}{\ohm}$ und $R'=\SI{500.0}{\ohm}$ verwendet und eine konstante, frequenzunabhängige Speisespannung von
$U_\text{Sp}=\SI{4.0}{\volt}$ gemessen.
Der Messwert der Frequenz der minimalen Brückenspannung ist ${f_0=\SI{613.0}{\hertz}}$. 
Berechnet man diese mit den gegebenen Referenzwerten der Bauteile, erhält man einen Wert von 
\begin{equation}
    f_{0\text{,rechn}}=\frac{1}{2 \symup{\pi} RC}=\SI{603}{\hertz} .
\end{equation}
\begin{table}
    \centering
    \caption{Messwerte der Wien-Robinson-Brücke.}
    \label{tab:wien}
    \begin{tabular}{S[table-format=3.1] S[table-format=1.4] S[table-format=3.0] S[table-format=1.4]}
        \toprule
        {$f \:/\: \si{\hertz}$} & {$\Omega = \sfrac{f}{f_0}$} & {$U_\text{Br} \:/\: \si{\milli\volt}$} & {$\sfrac{U_\text{Br}}{U_\text{Sp}}$} \\
        \midrule
        213.0   & 0.3475 & 460    & 0.12   \\
        263.0   & 0.4290 & 360    & 0.09   \\
        313.0   & 0.5106 & 290    & 0.073  \\
        363.0   & 0.5922 & 230    & 0.058  \\
        413.0   & 0.6737 & 175    & 0.044  \\
        463.0   & 0.7553 & 122    & 0.031  \\
        513.0   & 0.8369 & 80     & 0.020  \\
        563.0   & 0.9184 & 38     & 0.0095 \\
        573.0   & 0.9347 & 32     & 0.0080 \\
        583.0   & 0.9511 & 24     & 0.0060 \\
        593.0   & 0.9674 & 16     & 0.0040 \\
        603.0   & 0.9837 & 10     & 0.0025 \\
        613.0   & 1.000  & 6      & 0.0015 \\
        623.0   & 1.016  & 10     & 0.0025 \\
        633.0   & 1.033  & 20     & 0.0050 \\
        643.0   & 1.049  & 30     & 0.0075 \\
        653.0   & 1.065  & 36     & 0.0090 \\
        663.0   & 1.082  & 40     & 0.010  \\
        713.0   & 1.163  & 67     & 0.017  \\
        763.0   & 1.245  & 95     & 0.024  \\
        813.0   & 1.326  & 125    & 0.031  \\
        863.0   & 1.408  & 148    & 0.037  \\
        913.0   & 1.489  & 170    & 0.043  \\
        963.0   & 1.571  & 190    & 0.048  \\
        1013.0  & 1.653  & 215    & 0.054  \\
    \end{tabular}
\end{table}
\FloatBarrier
Die Messwerte sowie die erwartete Kurve 
% HIER: Bitte noch Referenz zur Theorie mit der Formel zu der erwarteten Kurve einfügen
sind, wie in der Versuchsanleitung beschrieben, in \ref{fig:wien} aufgetragen.
\begin{figure}
    \centering
    \includegraphics[width=0.75\textwidth]{content/plot_wien.pdf}
    \caption{Messkurve der Wien-Robinson-Brücke.}
    \label{fig:wien}
\end{figure}

\subsection{Klirrfaktor-Messung}
\label{klirr_sub}
\FloatBarrier
Zur Berechnung des Klirrfaktors $k$ des Generators wird vereinfachend angenommen, dass ausschließlich Oberwellen der 
zweiten Oberwelle ausschlaggebend sind. 
So vereinfacht sich die Berechnung von $n$ Oberwellen zu 
\begin{equation}
    k=\frac{\sqrt{U_2^2 + U_3^2 + \dotsc + U_n^2}}{U} = \frac{U_2}{U}.
\end{equation}
Für die Spannung $U$ gilt $U=U_\text{Sp}=\SI{4.0}{\volt}=\text{const}$. 
Die Amplitude der zweiten Oberwelle ($\Omega = \sfrac{f_2}{f_1} = 2$) berechnet sich über
\begin{equation}
    U_2=3 \, U_\text{Br} \cdot \frac{\sqrt{(\Omega ^2 -1)^2 + 9\Omega ^2}}{\lvert \Omega ^2 -1 \rvert}
        =3 \cdot \SI{6}{\milli\volt} \cdot \frac{\sqrt{9 + 36}}{3}=3 \sqrt{5} \cdot \SI{6}{\milli\volt} \approx \SI{40}{\milli\volt} .
\end{equation}
Hierbei wird der minimale Wert für die Brückenspannung verwendet, da bei dieser ausschließlich Oberwellen, in dem Fall also die zweite Oberwelle, für eine Spannung sorgen. 
Mit $U=U_\text{Sp}=\SI{4.0}{\volt}$ ergibt sich der Klirrfaktor:
\begin{equation}
    k=\frac{U_2}{U_\text{Sp}}=0.010
\end{equation}

