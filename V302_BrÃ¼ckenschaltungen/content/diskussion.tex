\section{Diskussion}
\label{sec:Diskussion}

Bei Betrachtung der Ergebnisse fällt relativ schnell ins Auge, dass sich für die 
ohmschen Widerstände ein vergleichsweise eindeutiger Wert bestimmen lässt, wohingegen die kapazitiven und induktiven 
Widerstände größere Abweichungen aufweisen. 
Grund dafür könnte die Wahl der Frequenz sein. 
Für eine optimale Messung befinden sich die Wirk- und Blindwiderstände in der gleichen Größenordnung. 
Die Blindwiderstände bei der Kapazitätsmessbrücke mit der Kapazität ${C_2\sim \num{5e-7}}$ liegen mit Frequenzen 
${f \sim 10^4}$ in einer Größenordnung von ${\sim \num{e1}}$. Die ohmschen Widerstände sind in etwa ${\propto \num{5e3}}$. 
Durch die etwas höhere Wahl der Frequenz als der Referenzierten ist der Einfluss der Streukapazitäten in den 
Verdrahtungen größer. Da bei jedem Messvorgang unterschiedliche Kapazitäten und Widerstände verwendet worden sind, ist 
dieser Einfluss jeweils unterschiedlich ausschlaggebend und führt zu teils sehr verschiedenen Abweichungen vom Mittelwert. 
Zudem ist die der Kapazitätsmessbrücke zugrunde liegende Annahme, dass es sich um einen ohmschen verlustfreien Kondensator 
handelt, was zu Differenzen zwischen erwartetem und gemessenen Wert führt. 

Die Messung der verlustbehafteten Induktivitäten zeigt große Abweichungen, insbesondere im Vergleich der Induktivitätsmessbrücke 
und der Maxwell-Brücke. 
Ebenfalls liegen hier die Blindwiderstände in einer um einen Faktor ${10^2}$ verschobenen Größenordnung im Vergleich zu 
der der Wirkwiderstände. Daraus resultieren, wie erläutert, größere Auswirkungen der induktiven und kapazitiven Widerstände 
in den Leitern, die die gemessenen Werte verfälschen können. 
Die Abweichung, die der aus der Maxwell-Brücke berechnete Wert für die Induktivität aufweist, lässt sich hiermit jedoch kaum
erklären. 
Nahe liegt die Vermutung, dass es sich um einen systematischen Messfehler handelt, der sich im Nachhinein nur schwerlich 
verifizieren lässt -- zumal diese Messung nur einmal, ohne Wiederholung durchgeführt worden ist. 
Die konsequente Streuung der Messergebnisse der Induktivitätsmessbrücke um einen Mittelwert deutet darauf hin, dass der Messfehler 
bei der Maxwell-Brücke zu finden ist. 
Wirklich sicher kann dies jedoch nicht festgemacht werden, da systematische Messfehler die Werte der Induktivitätsmessbrücke
ebenfalls im gleichen Maße verfälschen könnten. 

Bei der Messung mit der Wien-Robinson-Brücke zur Untersuchung der Frequenzabhängigkeit fällt anhand Abbildung \ref{fig:wien} 
auf, dass die Messwerte mit etwa gleicher Abweichung unterhalb der zu erwartenden Kurve des Spannungsverhältnisses liegen. 
Dies deutet auf einen systematischen Fehler hin, der vermutlich in nicht perfekten, verlustfreien Kabeln begründet ist. 
Die Speisespannung wird direkt am Generator gemessen, wohingegen die Brückenspannung erst mitten im Stromkreis abgegriffen 
wird. Nicht beachtet werden dabei ohmsche Verluste über die Kabel, über die der Strom bis zur Messstelle erst noch fließt.

Die Differenz der Nullfrequenz vom Messwert ist relativ gering und lässt sich gut über durch statistische Messunsicherheiten 
bedingte Abweichungen erklären.

Die in \ref{klirr_sub} durchgeführte Klirrfaktor-Berechnung bestätigt die Vermutung, dass sich ein idealer Sinusgenerator 
von einem realen durch den Gehalt an Oberwellen unterscheidet. 
Weil ${k \neq 0}$, sind beim Minimum der Brückenspannung noch Wellen messbar, was beim idealen Sinusgenerator nicht der Fall wäre.
\clearpage
