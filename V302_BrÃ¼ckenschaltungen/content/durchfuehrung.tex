\section{Durchführung}
\label{sec:Durchführung}
\subsection{Wheatstone}
Für die Wheatstone'sche Brückenschaltung werden alle Kabel gemäß der Abbildung verbunden. Die Speisespannung wird daraufhin an der Spannungsquelle maximal eingestellt.
Die Brückenspannung wird mit Hilfe eines Oszilloskops dargestellt. Nun wird an dem Potentiometer so weit gedreht, sodass die Brückenspannung Null wird.
Der eingetellte Widerstand wird abgelesen und der Vorgang mit Austauschen des Widerstands $R_2$ insgesamt 3 mal durchgeführt.

\subsection{Kapazitätsmessbrücke}
Es werden wieder die Bauelemente nach der Abbildung verschaltet. Nun jedoch müssen beim Abgleichen die zwei Potentiometer individuell variiert werden, sodass nicht nur Spannung, 
sondern auch die Phase verschwindet. Wird dies erreicht, wird der Messwert abgelesen und alles ebenfalls 3 mal durchgeführt.

\subsection{Induktionsmessbrücke}
Die Messung erfolgt vollkommen analog zur Kapazitätsmessbrücke.

\subsection{Wien-Robinson-Brücke}
Es wird erneut die Schaltung nach Plan aufgebaut. Für diese Messungen jedoch wird ein anderes Gerät verwendet, bei dem die eingestellte Frequenz digital abgelesen und fein eingestellt werden kann.
Es wird zunächst der Bereich gesucht, an dem die Brückenspannung und -phase verschwindet. Anschließend werden 10 Werte nah um die eingestellte Frequenz abgetastet, und danach weitere 14 Werte in einem
größeren Frequenzbereich um $\omega_0$ aufgenommen.