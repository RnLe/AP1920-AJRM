\section{Auswertung}
\label{sec:Auswertung}
Gemäß \cite{Versuchsanleitung} wird für die Referenzbauteile ein Fehler von $\pm{0.2\%}$ angenommen. 
Das verwendete Potentiometer hat eine Toleranz von $\pm{0.5\%}$.
Die Fehlerrechnungen sind jeweils mit iPython 7.8.0 mit dem Paket \textit{uncertainties} durchgeführt worden.
\subsection{Messung mit der Wheatstone'schen Brücke}
    \begin{table}
        \centering
        \caption{Messdaten für die Wheatstone'sche Brückenschaltung.}
        \label{tab:wheat}
        \begin{tabular}{c S[table-format=4.1] @{${}\pm{}$} S[table-format=1.1] S[table-format=3.0] @{${}\pm{}$} S[table-format=1.0] S[table-format=3.0] @{${}\pm{}$} S[table-format=1.0]}
            \toprule
            {$R_\text{x}$} & \multicolumn{2}{c}{$R_2 \:/\: \si{\ohm}$} & \multicolumn{2}{c}{$R_3 \:/\: \si{\ohm}$} & \multicolumn{2}{c}{$R_4 \:/\: \si{\ohm}$} \\
            \midrule
            Wert 12 & 332.0 & 0.7   & 542 & 3 & 458 & 2 \\
                    & 500.0 & 1.0   & 440 & 2 & 560 & 3 \\
                    & 1000  & 2     & 282 & 1 & 718 & 4 \\
            Wert 11 & 332.0 & 0.7   & 598 & 3 & 402 & 2 \\
                    & 500.0 & 1.0   & 497 & 2 & 503 & 3 \\
                    & 1000  & 2     & 330 & 2 & 670 & 3 \\
            \bottomrule
        \end{tabular}
    \end{table}
    Laut der Abgleichbedingung
    \begin{equation}
        R_x = R_2 \frac{R_3}{R_4}
    \end{equation}
    ergeben sich für die Werte 12 und 11 folgende Ergebnisse:
    \begin{table}
        \centering
        \caption{Messergebnisse der Wheatstone-Brücke.}
        \label{tab:resultwheat}
        \begin{tabular}{c c S[table-format=3.0] @{${}\pm{}$} S[table-format=1.0]}
            \toprule
            {Widerstand} & {Messung Nr.} & \multicolumn{2}{c}{$R_x \:/\: \si{\ohm}$} \\
            \midrule
            Wert 12 & 1 & 393 & 3 \\
                    & 2 & 393 & 3 \\
                    & 3 & 393 & 3 \\
            Wert 11 & 1 & 494 & 4 \\
                    & 2 & 494 & 4 \\
                    & 3 & 493 & 4 \\
            \bottomrule
        \end{tabular}
    \end{table}
 
\subsection{Messung mit der Kapazitätsmessbrücke}
    Die erste Kapazität (Wert 1) wurde mit einer Frequenz von $f=\SI{40}{\kilo\hertz}$, die zweite (Wert~3) mit $f=\SI{36}{\kilo\hertz}$ gemessen.
    \begin{table}
        \centering
        \caption{Messdaten für die Kapazitätsmessbrückenschaltung.}
        \label{tab:kapzmess}
        \begin{tabular}{c S[table-format=3.0] @{${}\pm{}$} S[table-format=1.0] S[table-format=3.0] @{${}\pm{}$} S[table-format=1.0] S[table-format=3.0] @{${}\pm{}$} S[table-format=1.0]}
            \toprule
            {$C_\text{x}$} & \multicolumn{2}{c}{$C_2 \:/\: \si{\nano\farad}$} & \multicolumn{2}{c}{$R_3 \:/\: \si{\ohm}$} & \multicolumn{2}{c}{$R_4 \:/\: \si{\ohm}$} \\
            \midrule
            Wert 1  & 450   & 1 & 408 & 1 & 592 & 1 \\
                    & 597   & 1 & 487 & 1 & 513 & 1 \\
                    & 992   & 2 & 611 & 1 & 389 & 1 \\
            Wert 3  & 597   & 1 & 595 & 1 & 405 & 1 \\
                    & 992   & 2 & 711 & 1 & 289 & 1 \\
                    & 750   & 2 & 643 & 1 & 377 & 1 \\
            \bottomrule
        \end{tabular}
    \end{table}
    Aus der Abgleichbedingung
    \begin{equation}
        C_x = C_2 \frac{R_4}{R_3}
    \end{equation}
    resultierende die in \ref{tab:kapz} dargestellten Werte.
     \begin{table}
        \centering
        \caption{Messergebnisse der Kapazitätsmessbrücke.}
        \label{tab:kapz}
        \begin{tabular}{c c S[table-format=3.0] @{${}\pm{}$} S[table-format=1.0]}
            \toprule
            {Kapazität} & {Messung Nr.} & \multicolumn{2}{c}{$C_x \:/\: \si{\nano\farad}$} \\
            \midrule
            Wert 1  & 1 & 653 & 2 \\
                    & 2 & 629 & 2 \\  
                    & 3 & 632 & 2 \\  
            Wert 3  & 1 & 406 & 1 \\  
                    & 2 & 403 & 2 \\ 
                    & 3 & 440 & 2 \\  
            \bottomrule 
        \end{tabular}
    \end{table}

\subsection{Messung mit der Induktivitätsmess- und der Maxwell-Brücke}
    Bei Messung der verlustbehafteten Induktivität (Wert 19) mit der Induktivitätsmessbrücke wurde eine Frequenz 
    von $f=\SI{100}{\hertz}$ benutzt, 
    bei der mit der Maxwell-Brücke $f=\SI{40.2}{\kilo\hertz}$.
    Für die Widerstände $R_3$ und $R_4$ wird bei der Maxwell-Brücke eine Toleranz von $\pm{3\%}$ angenommen.  
    \begin{table}
        \centering
        \caption{Messdaten zur Ermittlung der Induktivität mit Wert 19.}
        \label{tab:indumess}
        \begin{tabular}{c 
                        S[table-format=2.1] @{${}\pm{}$} S[table-format=1.2] 
                        S[table-format=3.0] @{${}\pm{}$} S[table-format=1.0] 
                        S[table-format=3.1] @{${}\pm{}$} S[table-format=1.1] 
                        S[table-format=3.0] @{${}\pm{}$} S[table-format=1.0]  
                        S[table-format=3.0] @{${}\pm{}$} S[table-format=1.0]}
            \toprule
            {Messmethode} 
            & \multicolumn{2}{c}{$L_2 \:/\: \si{\milli\henry}$} 
            & \multicolumn{2}{c}{$C_4 \:/\: \si{\nano\farad}$} 
            & \multicolumn{2}{c}{$R_2 \:/\: \si{\ohm}$} 
            & \multicolumn{2}{c}{$R_3 \:/\: \si{\ohm}$} 
            & \multicolumn{2}{c}{$R_4 \:/\: \si{\ohm}$} \\
            \midrule
            Ind.-Br.    & 14.6 & 0.03 & & & 332.0   & 0.7   & 255 & 1   & 745 & 4 \\
                        & 14.6 & 0.03 & & & 500.0   & 1.0   & 185 & 1   & 815 & 4 \\
                        & 14.6 & 0.03 & & & 664     & 1     & 145 & 1   & 855 & 4 \\
            Maxwell     & & & 750 & 2     & 664     & 1     & 57  & 2   & 270 & 8 \\
            \bottomrule
        \end{tabular}
    \end{table}
    Mithilfe der Abgleichbedingungen für die Induktivitätsmessbrücke 
    \begin{align}
        R_x=R_2 \frac{R_3}{R_4}
        &L_x=L_2 \frac{R_3}{R_4}
    \end{align}
    und denen für die Maxwell-Brücke 
    \begin{align}
        R_x=R_2 \frac{R_3}{R_4}
        &L_x=C_4 R_2 R_3
    \end{align}
    ergibt sich für die verlustbehaftete Spule: 
    \begin{table}
        \centering
        \caption{Innenwiderstand und Induktivität der verwendeten Spule.}
        \label{tab:R_L_Spule}
        \begin{tabular}{c S[table-format=3.0] @{${}\pm{}$} S[table-format=1.0] S[table-format=2.2] @{${}\pm{}$} S[table-format=1.2]}
            \toprule
            {Messung Nr.} & \multicolumn{2}{c}{$R_x \:/\: \si{\ohm}$} & \multicolumn{2}{c}{$L_x \:/\: \si{\milli\henry}$} \\
            \midrule
            1           & 114 & 1 & 5.00 & 0.04 \\
            2           & 113 & 1 & 3.31 & 0.03 \\
            3           & 113 & 1 & 2.48 & 0.02 \\
            4 (Maxwell) & 140 & 6 & 28.4 & 1.0  \\ %Was soll das?? Warum ist der Wer einfach mal um eine Zehnerpotenz größer??
            \bottomrule
        \end{tabular}
    \end{table}
