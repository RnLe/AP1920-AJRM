\section{Diskussion}
\label{sec:Diskussion}
%Auffallend sind auf den ersten Blick die starken Abweichungen des von dem Dulong--Petit-Gesetz gegebenen Werts $C_V = 3 \symup{R} = \SI{24.9435}{\joule\per\mol\per\kelvin}$
%und den aus dem Experiment berechneten Werten, sowie die von den experimentellen Werten untereinander. 
%Diese liegen im Mittel bei $\bigl< C_V\bigr> = \SI{94.784e3}{\joule\per\mol\per\kelvin} $. 
Auffallend ist auf den ersten Blick die starke Abweichungen des von dem Dulong--Petit-Gesetz gegebenen Werts $C_V = 3 \symup{R} = \SI{24.9435}{\joule\per\mol\per\kelvin}$
und dem aus dem Experiment berechneten Wert für Zinn. 
Deutlich außerhalb des berechneten Fehlerintervalls liegt 3 $3 \symup{R}$. 

Aufgrund dessen, dass alle Messwerte dieses Materials in der gleichen Größenordnung eine Abweichung von dem 
theoretischen Wert aufweisen, könnte davon ausgegangen werden, dass u.a. systematische Fehler vorliegen, die in der 
Nichtberücksichtigung von weiteren Einflüssen den Messwert in gleicher Form verfälschen. 
Da bei den Messungen von Kupfer und Aluminium keine solche Abweichungen vorliegen, müssen anderweitige Fehlerquellen 
beziehungsweise Aussagen über die Theorie hinter dem Experiment in Erwägung gezogen werden.

%Aufgrund dessen, dass alle Messwerte dieses Materials in der gleichen Größenordnung eine Abweichung von dem theoretischen Wert aufweisen, 
%kann davon ausgegangen werden, dass u.a. systematische Fehler vorliegen, die in der Nichtberücksichtigung von 
%weiteren Einflüssen den Messwert in gleicher Form verfälschen. 
%Solche Fehler sind oft schwierig zu quantifizieren und ihre Auswirkungen werden unterschätzt. 
%Deswegen folgt hier nur eine qualitative Abschätzung dieser Art der Messfehler.
%
Das Thermometer hat bei Messung der Mischtemperatur teilweise unerwartet die Anzeige geändert, ohne 
dass die Ursache dafür ersichtlich war. 
Ob also bei jeder Messung die \glqq richtige\grqq{} Mischtemperatur gemessen wurde, sei dahingestellt. %deutsche Anführungszeichen
Abweichungen von einigen Dezikelvin sind hier allemal möglich. 
Dies würde zumindest beim Faktor $(T'_\text{m} - T_\text{w})^{-1}$, der in \eqref{eqn:c_k} verwendet wird, eine prozentual 
betrachtet signifikante Schwankung der berechneten Werte zur Folge haben, da $T'_\text{m}$ und $T_\text{w}$ relativ nah beieinander liegen.

%Zur Berechnung der Molwärme wurde angenommen, dass die gesamte Wärmemenge des erhitzten Körpers vom Dewar-Gefäß und dem 
%darin enthaltenen Wasser absorbiert wurde. 
%Vernachlässigt wurden dabei jegliche Austauschprozesse, die sehr wahrscheinlich beim Transport des Metalls vom 
%Wasserbad ins Dewar-Gefäß und über den lose auf das offene Dewar-Gefäß gelegten Plastikdeckel stattfanden. 
%Nimmt man nun an, dass über Genanntes die Wärmemenge $Q_\text{xy}$ verloren gegangen ist, sieht man schnell, dass die 
%spezifische Wärmekapazität $c_k$, welche proportional zu $C_V$ ist, einen größeren Wert annimmt:
%\begin{gather*}
%    Q_1 = c_\text{k}m_\text{k}(T_\text{k}-c_\text{m}) \\
%    Q_\text{mess}=Q_2 + Q_\text{xy}=(c_\text{w}m_\text{w} + c_\text{g}m_\text{g})(T_\text{m}-T_\text{w}) + Q_\text{xy} \\
%    Q_1 = Q_\text{mess} 
%    \Leftrightarrow c_\text{k,mess} = \frac{Q_2+Q_\text{xy}}{m_\text{k}(T_\text{k}-T_\text{m})} > \frac{Q_2}{m_\text{k}(T_\text{k}-T_\text{m})} = c_\text{k,theor.} \\
%    \Rightarrow C_\text{V,mess} > C_\text{V,theor.}
%\end{gather*}
%Eine dermaßen große Abweichung durch genannte Ursachen ist nicht abwegig, da der Plastikdeckel beim Einstellen der 
%Mischtemperatur relativ warm war, was einen Wärmeaustritt indiziert. 
%
Auch bezüglich statistischer Fehler muss angemerkt werden, dass die Anzahl an hier vorgenommenen Messungen (zwei 
Metalle dreimal gemessen, ein Metall nur einmal) relativ gering ist und somit ebenfalls eine größere Abweichung provoziert wurde.

%Darüber hinaus hat das Thermometer bei Messung der Mischtemperatur teilweise unerwartet die Anzeige geändert, ohne 
%dass die Ursache dafür ersichtlich war. 
%Ob also bei jeder Messung die \glqq richtige\grqq{} Mischtemperatur gemessen wurde, sei dahingestellt. %deutsche Anführungszeichen
%Abweichungen von einigen Dezikelvin sind hier allemal möglich. 
%
%Hier noch Bezug nehmen zu Vergleich: klassisch vs quantenmechanisch
%Da, wie erläutert, vergleichsweise große Abweichungen von den erwarteten Ergebnissen durch systematische Messfehler 
%entstanden sind, die zusätzlich sich nur schwerlich quantifizieren lassen können, kann kaum eine stichhaltige 
%Aussage darüber gemacht werden, wie konsistent die Ansätze der klassischen Physik und der Quantenmechanik im Einzelnen sind. 
%Jeglich festzustellen bleibt, dass die aus den Messwerten berechneten molaren Wärmekapazitäten sich in derselben Größenordnung 
%wie $3 \symup{R}$ befinden. 
Dem gegenüber steht die Tatsache, dass diese Abweichungen ausschließlich bei Zinn zu finden sind. 
Das von Aluminium gegebene Fehlerintervall würde das Dulong--Petit-Gesetz bestätigen, ebenso der berechnete Wert von 
Kupfer liegt nahe $3\symup{R}$.
 %hier noch einen plausiblen Grund finden... ich finde momentan keinen. 

Eine weitere Erklärung ist eine mögliche Unreinheit des Materials. Außerdem gibt es viele gängige Zinnlegierungen.
In der Regel gibt es eine Zinnmarke, der hier jedoch fehlt. Diese gibt die Art des Zinns, bzw. der Legierung an.
Da sich bei einer Legierung, oder einer großen Unreinheit, die materialabhängige spezifische Wärme ändern würde, ändere sich damit auch 
die errechnete Gaskonstante. Die Unreinheit festzustellen ist für uns jedoch nicht umsetzbar.