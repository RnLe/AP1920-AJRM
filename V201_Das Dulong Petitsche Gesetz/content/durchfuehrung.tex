\section{Durchführung}
\label{sec:Durchführung}
\subsection{Ziel}
Ziel der Durchführung ist es, Messdaten über die Mischtemperatur aufgeheizter Messkörper in vorher abgewogenem Wasser zu bestimmen.
Darüber können wir eine Aussage treffen, welche spezifischen Wärmen die einzelnen Materialien haben, um die Ergebnisse dann mit der Theorie abzugleichen.

\subsection{Aufbau}
Für den Versuch wird ein Dewar-Gefäß benötigt, welches durch seine hohe Isolierung wenig Wärme aufnimmt und sich damit gut eignet. 
Um das Wasser abzumessen und um die Messkörper aufzuheizen, werden zwei Bechergläser verwendet.
Des Weiteren werden eine Waage, ein digitales Thermometer und eine Heizplatte verwendet.
Es sollte zudem genügend Wasser mit möglichst gleicher Temperatur zur Verfügung stehen - und auch ein Waschbecken, um das (z.T. heiße) Wasser abzuschütten.

\subsection{Durchführung}
Zunächst werden die Bechergläser gewogen, um daraufhin das Wasser abwiegen zu können. Dann werden die Messkörper gewogen.
Als nächstes wird die spezifische Wärme des Dewar-Gefäßes bestimmt, indem wir zwei ungefähr gleich große Anteile Wasser nehmen, und das Dewar-Gefäß
mit einem Anteil etwas weniger als die Hälfte füllen. Das Wasser sollte Raumtemperatur haben, was auch nachgemessen und aufgeschrieben wird.
Der andere Anteil wird auf ungefähr $70-80\si{\celsius}$ erhitzt, und nach dem Erhitzen ebenfalls mit dem Thermometer gemessen.
Das heiße Wasser wird sofort nach der Messung in das Dewar-Gefäß geschüttet und am besten etwas umgerührt.
Es stellt sich sehr schnell eine Mischtemperatur zwischen dem Wasser mit Raumtemperatur und dem heißen Wasser ein.
Diese Mischtemperatur wird gemessen und notiert. Sie kann gemessen werden, wenn die Temperatur des Wassers über ungefähr 10 Sekunden konstant bleibt.

Danach wird das Wasser weggeschüttet und neues Wasser in das Dewar-Gefäß gefüllt, damit dieses wieder auf Raumtemperatur abkühlt.
Nun wird ein Messkörperer hitzt, indem es in ein mit Wasser gefülltes Becherglas gehängt wird. Die Messkörper hängen an einem Deckel, der auf
die Messbecher passt, sodass sie frei im Wasser hängen.

Das Becherglas wird auf die Heizplatte gestellt und erhitzt. Nach einigen Minuten ist der Körper etwa über $70\si{\celsius}$ heiß.
Wir messen die Temperatur des Wassers, welches um den Körper erhitzt wird und setzen die Wassertemperatur mit der Körpertemperatur gleich.
Wir messen auch die Temperatur des Wassers in dem Dewar-Gefäß, welches wir vorher mit Hilfe des anderen Becherglases und der Waage gewogen haben.
Der heiße Körper wird nun in das Wasser im Dewar-Gefäß gehängt und am besten leicht bewegt, um das Vermischen zu beschleunigen.
Es stellt sich wieder eine Mischtemperatur ein, die wir messen und notieren. Der Messkörper und das Wasser haben nach dem Mischen eine Temperatur,
die nur knapp über der Raumtemperatur liegt, sodass weder das Dewar-Gefäß noch der Körper groß abgekühlt werden müssen.
Es empfiehlt sich dennoch bei dem Entleeren des Dewar-Gefäßes einmal mit Wasser durchzuspülen, um die Temperatur des Gefäßes anzugleichen.
Das heiße Wasser aus dem Becherglas, in dem der Körper erhitzt wurde, wird ebenfalls weggeschüttet.
Nun können wird das Becherglas wieder füllen, den Messkörper eintauchen und auf die Heizplatte stellen.
Wir wiegen wieder Wasser mit Hilfe des anderen Becherglases ab und füllen das Dewar-Gefäß.
Wir wiederholen diesen Vorgang drei mal und tauschen dann den Messkörper und damit das zu erhitzende Material.
Insgesamt untersuchen wir drei Materialien, wobei wir zwei der Messkörper je drei mal erhitzen bzw. messen, und den Vorgang für ein Material nur 
einmal anwenden, womit sich sieben Durchläufe ergeben, wenn man das erste Erhitzen und Mischen des Wassers ohne Messkörper nicht mit einrechnet.

Der Versuch ist schnell wieder aufgeräumt, da die Körper nur zurückgestellt und die Gefäße nur getrocknet werden müssen.