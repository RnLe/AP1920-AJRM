\section{Durchführung}
\label{sec:Durchführung}
\subsection{Ziel}
Ziel der Durchführung ist es, Messdaten über die Mischtemperatur aufgeheizter Messkörper in vorher abgewogenem Wasser zu bestimmen.
Mit Hilfe dieser Methode können wir eine Aussage darüber treffen, welche spezifischen Wärmen die Materialien besitzen, um die Ergebnisse im Anschluss mit der Theorie abzugleichen. %dann-->im Anschluss

\subsection{Aufbau}
Für den Versuch wird ein Dewar-Gefäß benötigt, welches durch seine hohe Isolierung wenig Wärme aufnimmt oder abgibt und sich damit für die Messungen gut eignet. %Gefäß nimmt nicht nur wenig wärme, es gibt auch wenig ab (isolierend)
Um das Wasser abzumessen und um die Messkörper aufzuheizen, werden zwei Bechergläser verwendet.
Des Weiteren werden eine Waage, ein digitales Thermometer und eine Heizplatte verwendet.
Es sollte zudem genügend Wasser mit möglichst gleicher Temperatur zur Verfügung stehen -- und auch ein Waschbecken, um das (z.T. heiße) Wasser wegzuschütten. %Gedankenstrich mit zwei Strichen --

\subsection{Durchführung}
Zunächst werden die Bechergläser gewogen, um daraufhin das Wasser abwiegen zu können. Dann werden die Messkörper gewogen.
Als nächstes wird die spezifische Wärme des Dewar-Gefäßes bestimmt, indem wir zwei ungefähr gleich große Anteile Wasser nehmen, und das Dewar-Gefäß %vom Gefühl her würde ich sagen, dass wir die Wir-Form in wissenschaftlichen Texten meiden sollten... 100% weiß ich es nicht und wollte das jetzt nicht einfach so ändern, vielleicht hast du dir ja was dabei gedacht.
mit einem Anteil etwas weniger als die Hälfte füllen. Das Wasser sollte Raumtemperatur haben, was auch nachgemessen und aufgeschrieben wird.
Der andere Anteil wird auf ungefähr $70-80 \, \si{\celsius}$ erhitzt, und nach dem Erhitzen ebenfalls mit dem Thermometer gemessen. %zwischen Zahl und Einheit ein Abstand mit \,
Das heiße Wasser wird sofort nach der Messung in das Dewar-Gefäß geschüttet und am besten etwas umgerührt.
Es stellt sich zeitnah eine Mischtemperatur zwischen dem Wasser mit Raumtemperatur und dem heißen Wasser ein.
Diese Mischtemperatur wird gemessen und notiert. Sie kann gemessen werden, wenn die Temperatur des Wassers über ungefähr 10 Sekunden konstant bleibt.

Danach wird das Wasser weggeschüttet und neues Wasser in das Dewar-Gefäß gefüllt, damit dieses wieder auf Raumtemperatur abkühlt.
Nun wird ein Messkörperer hitzt, indem es in ein mit Wasser gefülltes Becherglas gehängt wird. Die Messkörper sind an einem Plastikdeckel befestigt, der auf %Deckel-->Plastikdeckel: find ich wichtig, z.B. auch für die Auswertung. Über das Plastik geht ja vielmehr Wärme verloren als über das gesamte Dewar-Gefäß; sie hängen an einem Deckel-->sie sind befestigt: du hast zweimal in demselben Satz hängen benutzt
die Messbecher passt, sodass sie frei im Wasser hängen.

Das Becherglas wird auf die Heizplatte gestellt und erhitzt. 
Nach einigen Minuten ist der Körper etwa über $70 \, \si{\celsius}$ heiß. %Abstand zwischen Zahl und Einheit \, oder direkt SI{}{} benutzen
Gemessen wird freilich die Temperatur des Wassers, welches um den Körper erhitzt wird. 
Diese kann allerdings mit der Körpertemperatur gleichgesetzt werden, da von einer identischen Ausgangstemperatur der beiden Stoffe (Wasser und Metall) ausgegangen wird. %hier habe ich wenig umformuliert wegen einiger Wortdopplungen... nichts Schlimmes, hab auch das meiste deiner wortwahl so gelassen :)
Wir messen darüber hinaus die Temperatur des Wassers in dem Dewar-Gefäß, welches wir vorher mit Hilfe des anderen Becherglases und der Waage gewogen haben. %Wir(s.o.)(stattdessen vielleicht Passiv oder man-Form oder Nominalisierung)
Der heiße Körper wird nun in das Wasser im Dewar-Gefäß gehängt und am besten leicht bewegt, um den Wärmetransport zu beschleunigen.
Es stellt sich eine Mischtemperatur ein, die wir messen und notieren. Der Messkörper und das Wasser haben nach dem Mischen eine Temperatur, %wir
die nur knapp über der Raumtemperatur liegt, sodass weder das Dewar-Gefäß noch der Körper groß abgekühlt werden müssen.
Es empfiehlt sich dennoch bei dem Entleeren des Dewar-Gefäßes einmal mit Wasser durchzuspülen, um die Temperatur des Gefäßes anzugleichen.
Das heiße Wasser aus dem Becherglas, in dem der Körper erhitzt wurde, wird ebenfalls weggeschüttet.
Nun können wir das Becherglas wieder füllen, den Messkörper eintauchen und auf die Heizplatte stellen. %wir
Wir wiegen erneut Wasser mit Hilfe des anderen Becherglases ab und füllen das Dewar-Gefäß. %wortdopplung: wieder--> erneut
Wir wiederholen diesen Vorgang dreimal und tauschen den Messkörper und damit das zu erhitzende Material. %wir; drei mal-->dreimal (vgl. Duden ;) )
Insgesamt untersuchen wir drei Metalle, von denen wir zwei der Messkörper je dreimal erhitzen bzw. messen, und den Vorgang für ein Material nur %hier ist dir ein kleiner Bandwurmsatz unterlaufen... etwas auseinander gefriemelt jetzt
einmal anwenden. Daraus ergeben sich sieben Durchläufe, wenn man das erste Erhitzen und Mischen des Wassers ohne Messkörper nicht mit einrechnet.

Der Versuch ist schnell wieder aufgeräumt, da lediglich die Körper zurückgestellt und die Gefäße getrocknet werden müssen.
\pagebreak