\section{Durchführung}
\label{sec:Durchführung}
\subsection{Ziel}
Ziel der Durchführung ist es, Messdaten über die Mischtemperatur aufgeheizter Messkörper in vorher abgewogenem Wasser zu bestimmen.
Mit Hilfe dieser Methode kann eine Aussage darüber getroffen werden, welche spezifischen Wärmen die Materialien besitzen, um die Ergebnisse im Anschluss mit der Theorie abzugleichen. %dann-->im Anschluss

\subsection{Aufbau}
Für den Versuch wird ein Dewar-Gefäß benötigt, welches durch seine hohe Isolierung wenig Wärme aufnimmt oder abgibt und sich damit für die Messungen gut eignet. %Gefäß nimmt nicht nur wenig wärme, es gibt auch wenig ab (isolierend)
Um das Wasser abzumessen und um die Messkörper aufzuheizen, werden zwei Bechergläser verwendet.
Des Weiteren werden eine Waage, ein digitales Thermometer und eine Heizplatte genutzt.

\subsection{Durchführung}
Zunächst werden die Bechergläser gewogen, um daraufhin das Wasser abwiegen zu können. Dann werden die Messkörper gewogen.
Als nächstes wird die spezifische Wärme des Dewar-Gefäßes bestimmt, indem zwei ungefähr gleich große Anteile Wasser genommen werden, und das Dewar-Gefäß %vom Gefühl her würde ich sagen, dass wir die Wir-Form in wissenschaftlichen Texten meiden sollten... 100% weiß ich es nicht und wollte das jetzt nicht einfach so ändern, vielleicht hast du dir ja was dabei gedacht.
mit dem ersten Anteil etwas weniger als die Hälfte gefüllt wird. Das Wasser sollte idealerweise Raumtemperatur haben, was auch nachgemessen und aufgeschrieben wird.
Der andere Anteil wird auf ungefähr $70-80 \, \si{\celsius}$ erhitzt, und nach dem Erhitzen ebenfalls mit dem Thermometer gemessen. %zwischen Zahl und Einheit ein Abstand mit \,
Das heiße Wasser wird sofort nach der Messung in das Dewar-Gefäß geschüttet und etwas umgerührt.
Es stellt sich zeitnah eine Mischtemperatur zwischen dem Wasser mit Raumtemperatur und dem heißen Wasser ein.
Diese Mischtemperatur wird gemessen und notiert. Sie wird gemessen, wenn die Temperatur des Wassers über ungefähr 10 Sekunden konstant bleibt.

Danach wird das Wasser weggeschüttet und neues Wasser in das Dewar-Gefäß gefüllt, damit dieses wieder auf Raumtemperatur abkühlt.
Nun wird ein Messkörperer erhitzt, indem er in ein mit Wasser gefülltes Becherglas gehängt wird. Die Messkörper sind an einem Plastikdeckel befestigt, der auf %Deckel-->Plastikdeckel: find ich wichtig, z.B. auch für die Auswertung. Über das Plastik geht ja vielmehr Wärme verloren als über das gesamte Dewar-Gefäß; sie hängen an einem Deckel-->sie sind befestigt: du hast zweimal in demselben Satz hängen benutzt
die Messbecher passt, sodass sie frei im Wasser hängen.

Das Becherglas wird auf die Heizplatte gestellt und erhitzt. 
Gemessen wird die Temperatur des Wassers, welches um den Körper erhitzt wird. 
Diese kann mit der Körpertemperatur gleichgesetzt werden, da von einer identischen Ausgangstemperatur der beiden Stoffe (Wasser und Metall) ausgegangen wird. %hier habe ich wenig umformuliert wegen einiger Wortdopplungen... nichts Schlimmes, hab auch das meiste deiner wortwahl so gelassen :)
Darüber hinaus wird die Temperatur des Wassers in dem Dewar-Gefäß gemessen, welches vorher mit Hilfe des anderen Becherglases und der Waage gewogen wird. %Wir(s.o.)(stattdessen vielleicht Passiv oder man-Form oder Nominalisierung)
Der heiße Körper wird nun in das Wasser im Dewar-Gefäß gehängt und leicht bewegt, um den Wärmetransport zu beschleunigen.
Es stellt sich eine zu protokollierende Mischtemperatur ein. Der Messkörper und das Wasser haben nach dem Mischen eine Temperatur, %wir
die nur knapp über der Raumtemperatur liegt, sodass es nicht nötig ist, das Dewar-Gefäß noch den Körper groß abzukühlen.
Beim Entleeren des Dewar-Gefäßes wird einmal mit Wasser durchgespült, um die Temperatur des Gefäßes anzugleichen.
Das heiße Wasser aus dem Becherglas, in dem der Körper erhitzt wurde, wird ebenfalls weggeschüttet.
Nun wird das Becherglas wieder gefüllt, der Messkörper eingetaucht und auf die Heizplatte gestellt. %wir
Erneut wird Wasser mit Hilfe des anderen Becherglases abgewogen und in das Dewar-Gefäß gefüllt. %wortdopplung: wieder--> erneut
Nach dreimaliger Wiederholung dessen wird der Messkörper getauscht und damit das zu erhitzende Material. %wir; drei mal-->dreimal (vgl. Duden ;) )
Insgesamt werden drei Metalle untersucht, von denen zwei je dreimal erhitzt bzw. gemessen werden. Dieser Vorgang wird für ein beliebiges Material nur einmal angewendet. %hier ist dir ein kleiner Bandwurmsatz unterlaufen... etwas auseinander gefriemelt jetzt
Daraus ergeben sich sieben Durchläufe, wenn das erste Erhitzen und Mischen des Wassers ohne Messkörper nicht miteingerechnet wird.

\pagebreak