\section{Theorie}%\footnote{Unter Verwendung der Quellen \cite{demtroeder}, \cite{Versuchsanleitung}, \cite{gerthsen}.}}
\label{sec:Theorie}
\subsection{Zielsetzung}
    Ziel des Experiments ist es, zu untersuchen, inwiefern der klassische Ansatz der Physik genügt, Oszillationen von 
    Atomen und Molekülen im Festkörper zu untersuchen und zu beschreiben. 
    Eine davon abweichende, alternative Perspektive bietet die Quantenmechanik. 
    Um diesem näher auf den Grund zu gehen, wird im Folgenden die Molwärme $C$ einiger Metalle experimentell bestimmt und 
    mit den Werten jeweils aus der klassischen und quantenmechanischen Physik verglichen. 
\subsection{Ansatz der klassischen Physik}
    Die Molwärme $C$, auch bekannt unter der Atomwärme oder der molekularen Wärmekapazität, ist die auf die 
    Stoffmenge bezogene Wärmekapazität. 
    Sie gibt demnach die benötigte Wärmemenge $\increment Q$ an, um ein Mol eines Stoffes um ein $\SI{1}{\kelvin}$ zu erwärmen.
    Jene hängt davon ab, ob die Erwärmung bei konstantem Volumen $V$ oder bei konstant gehaltenem Druck $p$ durchgeführt 
    wird. An dem Formelzeichen der Größe wird das durch einen entsprechenden Index kenntlich gemacht: 
    \begin{align*}
        C_{p} &= \biggl(\frac{\partial Q}{\partial T}\biggr)_{\text{p}} &
        C_{V} &= \biggl(\frac{\partial Q}{\partial T}\biggr)_{\text{V}} \\
    \end{align*}

    Das aus der klassischen Physik abgeleitete Dulong--Petitsche Gesetz besagt nun, dass die spezifische Wärmekapazität bei 
    konstantem Volumen $C_\text{V}$ jeden Materials konstant $3 \, \symup{R}$ beträgt, mit $\symup{R}$ als der Allgemeinen Gaskonstante. 
    Die über einen langen Zeitraum gemittelte innere Energie $\bigl< u \bigr>$ eines Atoms im Festkörper setzt 
    sich aus der gemittelten potentiellen $\bigl< E_\text{pot} \bigr>$ und der gemittelten kinetischen Energie $\bigl< E_\text{kin} \bigr>$ zusammen. 
    Durch Berechnung der Gitter- und Trägheitskräfte, die das Atom beziehungsweise das Molekül in einer Gleichgewichtslage 
    halten, und der zugehörigen verrichteten Arbeit lässt sich die Beziehung 
    \begin{equation}
        \bigl< E_\text{pot} \bigr> = \bigl< E_\text{kin} \bigr>
    \end{equation} 
    herleiten. 
    Gemäß des Gleichverteilungssatz -- auch unter dem Äquipartitionstheorem bekannt -- $\bigl< E_\text{kin} \bigr> = \sfrac{f}{2} \, \symup{k} T$ 
    ist ein weiterer Zusammenhang gegeben. 
    Dieser besagt, dass jedes Molekül im thermischen Gleichgewicht pro Freiheitsgrad $f$ genannte konstante mittlere Bewegungsenergie innehat.
    Hierbei bezeichnet $\symup{k} = 1.3806505 \cdot 10^{-23} \, \si{\joule\per\kelvin} $ die Boltzmann-Konstante.
    Im Festkörper besitzt jedes Teilchen $3$ Freiheitsgrade. 
    Daraus geht
    \begin{equation}
        \bigl< u \bigr> = \bigl< E_\text{kin} \bigr> + \bigl< E_\text{pot} \bigr> = 2 \, \bigl< E_\text{kin} \bigr> = 3 \, \symup{k} T
    \end{equation}
    hervor. 
    In einem Mol eines Stoffes sind $\symup{N_\text{A}} = 6.0221415 \cdot 10^{23} \, \si{\mol\tothe{-1}}$ Teilchen enthalten, 
    weshalb für die Gesamtenergie eines Mols 
    \begin{equation}
        \bigl< U \bigr> = 3 \, \symup{N_\text{A}} \, \symup{k} \, T = 3 \, \symup{R} \, T
        \label{eq:formel}
    \end{equation}
    gilt. 
    
    Wird anschließend noch der Erste Hauptsatz der Thermodymanik auf die Erwärmung bei konstantem Volumen angewendet, erhält man
    \begin{equation}
        \mathrm{d} U = \mathrm{d} Q +  \mathrm{d} W
    \end{equation}
    mit
    \begin{align*} 
        U &\text{: innere Energie,} & Q &\text{: Wärmemenge,} & W &\text{: mechanische Arbeit,} & \rho &\text{: Dichte.}
    \end{align*}
    \begin{align}
         &  &\mathrm{d} W &= - \, \rho \mathrm{V} \, &\stackrel{V = \text{const}}{=} 0 &    &   &   &
        \Rightarrow &   &\mathrm{d} U &= \mathrm{d} Q &    \\
        \Rightarrow &   &C_{\text{V}} &= \biggl(\frac{\partial Q}{\partial T}\biggr)_{\text{V}} &\stackrel{\eqref{eq:formel}}{=} 3 \symup{R}
        &   &   &   &   &   &   &   &
    \end{align}
    
    Dieser eindeutig aus dem klassischen Ansatz resultierende Wert ist jedoch nicht mit der experimentellen Realität in 
    dieser hergeleiteten Absolutheit vereinbar: Festkörper hoher Temperatur erreichen in etwa diesen Wert, bei niedrigeren Temperaturen 
    gibt es indes starke Abweichungen. 
    Die Molwärme vieler Stoffe wird bei diesen Temperaturen beliebig klein. 
    Die Quantentheorie zeigt an dieser Stelle einen abweichenden Ansatz auf, der dies erklären könnte.
\subsection{Ansatz der Quantenmechanik}
    Grundannahme ist hier, dass die Energie nicht beliebig aufgenommen werden kann, sondern nur in festen, diskreten Quanten. 
    Die Boltzmann-Verteilung gibt die Wahrscheinlichkeit an, dass ein Teilchen eine Energie zwischen 
    $E$ und $(E + \mathrm{d}E)$ aufweist. 
    Integriert man nun die möglichen diskreten Energiewerte multipliziert mit ihrer Wahrscheinlichkeit und addiert diese, lässt 
    sich daraus auf eine bei weitem komplexere Formel und nicht einfach auf einen konstanten Wert für die innere Energie 
    eines Mols schließen:
    \begin{equation}
        \bigl< U_{\text{Qu}} \bigr> = \frac{3 \, \symup{N_{\text{A}}} \, \symup{\hbar} \, \omega}{\exp(\frac{\symup{\hbar} \omega}{\symup{k} T}) - 1}
    \end{equation}
    Diese Funktion geht für $T \to 0$ gegen Null und für große Temperaturen gegen den durch das Dulong--Petitsch-Gesetz 
    vorbestimmten Wert von $3 \, \symup{R} \, T$, genauer gesagt, wenn $\symup{k} \, T \gg \symup{\hbar} \, \omega$ gültig ist. 
    Zu beachten sei hierbei noch die Proportionalität $\omega \propto \sfrac{1}{\sqrt{m}}$.
    
    Demnach sind nur bei hohen Temperaturen das Dulong-Petitsche Gesetz und die Quantenmechanik konform zueinander. 

    Angemerkt sei noch, dass bei dieser Art der Berechnung der inneren Energie die verkomplizierende Modellaussage vernachlässigt wurde, 
    dass die Frequenzen $\omega$ ebenso nicht für alle Teilchen gleich sind, sondern eine Darstellung durch eine 
    kontinuierliche Verteilung besitzen. 
    Dies sollte für dieses Experiment nicht von Relevanz sein, aufgrund der Tatsache, dass dieser Aspekt ausschließlich 
    bei niedrigen Temperaturen in ausschlaggebenden Änderungen resultiert.