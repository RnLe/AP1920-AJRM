\section{Theorie}
\label{sec:Theorie}
\subsection{Zielsetzung}
    Ziel dieses Experiments ist es, zu untersuchen, inwiefern der klassische Ansatz der Physik genügt, Oszillationen von 
    Atomen und Molekülen im Festkörper zu untersuchen und beschreiben. 
    Eine davon abweichende, alternative Perspektive bietet die Quantenmechanik. 
    Um diesem näher auf den Grund zu gehen, wird im Folgenden die Molwärme $C$ einiger Metalle experimentell bestimmt und 
    mit den Werten jeweils aus der klassischen und quantenmechanischen Physik verglichen. 
\subsection{Ansatz der klassischen Physik}
    Die Molwärme $C$, auch bekannt unter der Atomwärme oder der molekularen Wärmekapazität, ist die auf die 
    Stoffmenge bezogene Wärmekapazität. 
    Sie ist 
    Sie gibt demnach die benötigte Wärmemenge $\increment Q$ an, um ein Mol eines Stoffes um ein $\SI{1}{\kelvin}$ zu erwärmen.
    Zu unterscheiden sind hierbei die Werte, die bei konstantem Volumen $V$ oder bei konstant gehaltenem Druck $p$ gemessen 
    werden. Dies wird stets durch einen entsprechenden Index kenntlich gemacht: 
    \begin{align*}
        C_{\text{p}} &= \biggl(\frac{\partial Q}{\partial T}\biggr)_{\text{p}} &
        C_{\text{V}} &= \biggl(\frac{\partial Q}{\partial T}\biggr)_{\text{V}} \\
    \end{align*}
