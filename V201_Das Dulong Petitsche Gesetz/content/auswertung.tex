\section{Auswertung}
\label{sec:Auswertung}
\subsection{Berechnen der spezifischen Wärme und der Molwärme}
Wir bestimmen zunächst die spezifische Wärme $c_gm_g$ des Dewar-Gefäßes, welche für die weiteren Rechnungen benötigt wird.
Über die Gleichung \eqref{eqn:dewar}
lässt sich mit den aufgenommenen Messdaten der Wert zu
\begin{equation} %Vorschlag: (das mache ich auch manchmal handschriftlich) die einheiten hinter den Bruch schreiben, einmal, dann haben wir faktisch nichts falsches, ohne Einheiten, da stehen, aber es dürfte auch übersichtlich/nicht allzu hässlich sein...  
    c_gm_g = \frac{4.18\times300.16(74.5-46.4)-4.18\times293.26(46.4-22)}{46.4-22} \, \si{\joule\per\kelvin} \approx \SI{219.1e3}{\joule\per\kelvin} %da war noch eine Masse einheit unter dem Bruchstrich, wo m.E. keine hingehört... passiert ;) 
\end{equation} %bei der rechnung hast du nicht beachtet, dass für c_w ein wert mit der einheit gramm und nicht kg genommen wird... dementsprechend erhöht sich das Ergebnis nochmal um den Faktor 10^3... hab es extra nochmal nachgerechnet
berechnen.
% Hier sei für uns einmal gegenübergestellt, wie man die Einheiten mit in die Gleichung beziehen kann.
% Eine Einheit für jede Zahl finde ich zu unübersichtlich - entsprechend würde ich die Einheit nur zum Ergebnis schreiben.
% Alles Weitere kann man ja aus den Tabellen ablesen.
% \begin{equation}
%     c_gm_g = \frac{4.18[\si{\joule\per\gram\per\kelvin}]\times300.16[\si{\g}](74.5[\si{\kelvin}]-46.4[\si{\kelvin}])-4.18[\si{\joule\per\gram\per\kelvin}]\times293.26[\si{\g}](46.4[\si{\kelvin}]-22[\si{\kelvin}])}{46.4[\si{\kelvin}]-22[\si{\kelvin}]} \approx 219.1[\si{\joule\per\gram\per\kelvin}].
% \end{equation}
Für $c_w$ nehmen wir den Wert $\SI{4.18}{\joule\per\g\per\kelvin}$ an\cite{Versuchsanleitung}. %wir
Die spezifische Wärme der Messkörper errechnet sich durch Gleichung \eqref{eqn:c_k}.
Zur Berechnung der molekularen Wärmekapazität bei konstantem Druck muss noch folgende Skalierung vorgenommen werden:
\begin{equation}
    C_{k,p}=c_k \cdot \frac{m_k}{n_k}=c_k \cdot M_k 
\end{equation}
$n_k$ bezeichnet an dieser Stelle die Stoffmenge. Die molare Masse $M$ entnehme man aus Tabelle \ref{tab:eigenschaften}. 
Da wir alle notwendigen Werte in Tabelle \ref{tab:values} aufgenommen haben, können wir $c_k$ für die drei Messkörper berechnen. %wir
\begin{table}
    \centering
    \caption{Spezifische Wärme und Wärmekapazität bei konstantem Druck der Messkörper.}
    \label{tab:c_k-Werte}
    \begin{tabular}{c c | S[table-format=4.2] S[table-format=3.1]}
        \toprule
        Stoff & Messung & $c_k[\si{\joule\per\g\per\kelvin}]$ & $C_{k,p}[\num{e3}\si{\joule\per\mol\per\kelvin}]$\\
        \midrule
        Zinn        & a) &   951.05 & 112.9   \\
                    & b) &   955.40 & 113.4   \\
                    & c) &   784.22 & 93.1   \\
        Aluminium   & a) &  3662.44 & 98.9   \\
                    & b) &  2721.43 & 73.5   \\
                    & c) &  2543.45 & 68.7   \\
        Kupfer      & a) &  1654.00 & 103.0   \\
        \bottomrule
    \end{tabular}
\end{table}

Mithilfe der in \cite{Versuchsanleitung} gegebenen Daten der Materialien lässt sich über den Zusammenhang \eqref{eqn:molwaerme}
die Molwärme aus den Messungen berechnen. 
Der Vollständigkeit halber seien hier ebenfalls die Werte für die hier verwendeten Stoffe aufgeführt.
\begin{table}
    \centering
    \caption{Eigenschaften der verwendeten Materialien.}
    \label{tab:eigenschaften}
    \begin{tabular}{c S[table-format=1.2] S[table-format=3.1] S[table-format=2.2] S[table-format=2.1] S[table-format=3.0]}
        \toprule
        Stoff & $\rho[\si{\gram\per\centi\meter\tothe{3}}]$ & $M[\si{\g\per\mol}]$ & $V_{0}[\num{e-6}\si{\meter\tothe{3}\mol\tothe{-1}}]$ & $\alpha[\num{e-6}\si{\kelvin\tothe{-1}}]$ & $\kappa[\num{e9}\si{\newton\per\meter\squared}]$ \\
        \midrule
        Zinn        &7.28   &118.7  &16.3   &27.0   &55     \\
        Aluminium   &2.70   &27.0   &10.0   &23.5   &75     \\
        Kupfer      &8.96   &63.5   &7.09   &16.8   &136    \\
        \bottomrule
    \end{tabular}
\end{table}

\begin{table}
    \centering
    \caption{Molwärme der Messkörper.}
    \label{tab:C_k-Werte}
    \begin{tabular}{c c | S[table-format=4.2]}
        \toprule
        Stoff & Messung & $C_{k,V}[\num{e3}\si{\joule\per\mol\per\kelvin}]$ \\
        \midrule
        Zinn        & a) & 112.898 \\
                    & b) & 113.398 \\
                    & c) &  93.098 \\
        Aluminium   & a) &  98.899 \\
                    & b) &  73.499 \\
                    & c) &  68.699 \\
        Kupfer      & a) & 102.999 \\
        \bottomrule
    \end{tabular}
\end{table}

\subsection{Vorbereitung zur Diskussion}
Um den Abgleich mit Literaturwerten und eine Diskussion zu ermöglichen, werten wir die Daten weiter aus.\\ %wir
Allgemeine Gaskonstante $\symup{R} = \SI{8.3145}{\joule\per\mol\per\kelvin}$ \cite{taschenbuch}.\\ %Konstanten mit symup aufrecht schreiben
\begin{equation*}
C_V = 3 \symup{R} = \SI{24.9435}{\joule\per\mol\per\kelvin}
\end{equation*}
