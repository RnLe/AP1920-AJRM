\section{Auswertung}
\label{sec:Auswertung}
\subsection{Berechnen der spezifischen Wärme}
Wir bestimmen zunächst die spezifische Wärme $c_gm_g$ des Dewar-Gefäßes, welche für die weiteren Rechnungen benötigt wird.
Über die Gleichung
\begin{equation}
    c_gm_g = \frac{c_wm_y(T_y-T'_m)-c_wm_x(T'_m-T_x)}{T'_m-T_x}
\end{equation}
lässt sich mit den aufgenommenen Messdaten der Wert zu
\begin{equation} %Vorschlag: (das mache ich auch manchmal handschriftlich) die einheiten hinter den Bruch schreiben, einmal, dann haben wir faktisch nichts falsches, ohne Einheiten, da stehen, aber es dürfte auch übersichtlich/nicht allzu hässlich sein...  
    c_gm_g = \frac{4.18\times300.16(74.5-46.4)-4.18\times293.26(46.4-22)}{46.4-22} \, \si{\joule\per\kelvin} \approx \SI{219.1}{\joule\per\kelvin} %da war noch eine Masse einheit unter dem Bruchstrich, wo m.E. keine hingehört... passiert ;) 
\end{equation}
berechnen.
% Hier sei für uns einmal gegenübergestellt, wie man die Einheiten mit in die Gleichung beziehen kann.
% Eine Einheit für jede Zahl finde ich zu unübersichtlich - entsprechend würde ich die Einheit nur zum Ergebnis schreiben.
% Alles Weitere kann man ja aus den Tabellen ablesen.
% \begin{equation}
%     c_gm_g = \frac{4.18[\si{\joule\per\gram\per\kelvin}]\times300.16[\si{\g}](74.5[\si{\kelvin}]-46.4[\si{\kelvin}])-4.18[\si{\joule\per\gram\per\kelvin}]\times293.26[\si{\g}](46.4[\si{\kelvin}]-22[\si{\kelvin}])}{46.4[\si{\kelvin}]-22[\si{\kelvin}]} \approx 219.1[\si{\joule\per\gram\per\kelvin}].
% \end{equation}
Für $c_w$ nehmen wir den Wert $\SI{4.18}{\joule\per\g\per\kelvin}$ an\cite{Versuchsanleitung}. %wir
Die spezifische Wärme der Messkörper errechnet sich durch
\begin{equation}
    c_k = \frac{(c_wm_w + c_gm_g)(T_m - T_w)}{m_k(T_k - T_m)}.
\end{equation}
Da wir alle notwendigen Werte in Tabelle \ref{tab:values} aufgenommen haben, können wir $c_k$ für die drei Messkörper berechnen. %wir
\begin{table}
    \centering
    \caption{Spezifische Wärmen der Messkörper.}
    \label{tab:c_k-Werte}
    \begin{tabular}{c c | S[table-format=4.2]}
        \toprule
        Stoff & Messung & $c_k[\si{\joule\per\g\per\kelvin}]$ \\
        \midrule
        Zinn        & a) &   951.05 \\
                    & b) &   955.40 \\
                    & c) &   784.22 \\
        Aluminium   & a) &  3662.44 \\
                    & b) &  2721.43 \\
                    & c) &  2543.45 \\
        Kupfer      & a) &  1654.00 \\
        \bottomrule
    \end{tabular}
\end{table}

\subsection{Vorbereitung zur Diskussion}
Um den Abgleich mit Literaturwerten und eine Diskussion zu ermöglichen, werten wir die Daten weiter aus.\\ %wir
Allgemeine Gaskonstante $\symup{R} = \SI{8.3145}{\joule\per\mol\per\kelvin}$ \cite{taschenbuch}. %Konstanten mit symup aufrecht schreiben