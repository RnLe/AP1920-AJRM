\section{Auswertung}
\label{sec:Auswertung}
\subsection{Messdaten}
\begin{table}
    \centering
    \caption{Massen der Messgegenstände.}
    \label{tab:masses}
    \begin{tabular}{l S[table-format=3.2]}
        \toprule
        {Objekt} & {Masse $m$ in $\si{\gram}$} \\
        \midrule
        kleines Becherglas  & 146.18 \\
        großes Becherglas   & 203.48 \\
        Dewar-Gefäß         & 850.99 \\
        \bottomrule
    \end{tabular}  
    \vspace{1cm}
    \caption{Messwerte zur Bestimmung von $c_g m_g$.}
    \label{tab:cgmg}
    \begin{tabular}{S S S S S}
        \toprule
        $m_x [\si{\gram}] $ & $m_y [\si{\gram}] $ & $T_x [\si{\celsius}] $ & $T_y [\si{\celsius}] $ & $T'_m [\si{\celsius}] $ \\
        \midrule
        293.26 & 300.16 & 22.0 & 74.5 & 46.4 \\
        \bottomrule
    \end{tabular}   
    \vspace{1cm}
    \caption{Messreihen zu Zinn, Aluminium und Kupfer.}
    \label{tab:values}
    \begin{tabular}{r c S[table-format=3.0] S[table-format=2.1] S S S}
        \toprule
        {k} & {Material} & {$m_\text{K} [\si{\gram}]$} & {$T_\text{k} [\si{\celsius}] $} & {$m_\text{w} [\si{\gram}] $} & {$T_\text{w} [\si{\celsius}] $} & { $T'_\text{m} [\si{\celsius}] $} \\
        \midrule
        1.a)      & Zinn      & 203   & 83.3  & 642.76  & 21.8  & 22.9  \\
          b)      &           &       & 83.1  & 645.94  & 21.6  & 22.7  \\
          c)      &           &       & 75.8  & 659.46  & 21.3  & 22.3  \\
        2.a)      & Aluminium & 153   & 87.2  & 654.63  & 21.0  & 24.0  \\
          b)      &           &       & 82.8  & 649.87  & 21.3  & 23.7  \\
          c)      &           &       & 75.1  & 669.27  & 21.0  & 23.5  \\
        3.a)      & Kupfer    & 237   & 89.2  & 654.82  & 20.9  & 22.9  \\
        \bottomrule
    \end{tabular}    
\end{table}

\subsection{Berechnen der spezifischen Wärme und der Molwärme}
Wir bestimmen zunächst die spezifische Wärme $c_gm_g$ des Dewar-Gefäßes, welche für die weiteren Rechnungen benötigt wird.
Über \eqref{eqn:dewar}
lässt sich mit den aufgenommenen Messdaten der Wert zu
\begin{equation} %Vorschlag: (das mache ich auch manchmal handschriftlich) die einheiten hinter den Bruch schreiben, einmal, dann haben wir faktisch nichts falsches, ohne Einheiten, da stehen, aber es dürfte auch übersichtlich/nicht allzu hässlich sein...  
    c_gm_g = \frac{4.18\times300.16(74.5-46.4)-4.18\times293.26(46.4-22)}{46.4-22} \, \si{\joule\per\kelvin} \approx \SI{219.1e3}{\joule\per\kelvin} %da war noch eine Masse einheit unter dem Bruchstrich, wo m.E. keine hingehört... passiert ;) 
\end{equation} %bei der rechnung hast du nicht beachtet, dass für c_w ein wert mit der einheit gramm und nicht kg genommen wird... dementsprechend erhöht sich das Ergebnis nochmal um den Faktor 10^3... hab es extra nochmal nachgerechnet
berechnen.
% Hier sei für uns einmal gegenübergestellt, wie man die Einheiten mit in die Gleichung beziehen kann.
% Eine Einheit für jede Zahl finde ich zu unübersichtlich - entsprechend würde ich die Einheit nur zum Ergebnis schreiben.
% Alles Weitere kann man ja aus den Tabellen ablesen.
% \begin{equation}
%     c_gm_g = \frac{4.18[\si{\joule\per\gram\per\kelvin}]\times300.16[\si{\g}](74.5[\si{\kelvin}]-46.4[\si{\kelvin}])-4.18[\si{\joule\per\gram\per\kelvin}]\times293.26[\si{\g}](46.4[\si{\kelvin}]-22[\si{\kelvin}])}{46.4[\si{\kelvin}]-22[\si{\kelvin}]} \approx 219.1[\si{\joule\per\gram\per\kelvin}].
% \end{equation}
Für $c_w$ wird der Wert $\SI{4.18}{\joule\per\g\per\kelvin}$ angenommen\cite{Versuchsanleitung}. %wir
Die spezifische Wärme der Messkörper errechnet sich durch \eqref{eqn:c_k}.
Zur Berechnung der molekularen Wärmekapazität bei konstantem Druck muss noch folgende Skalierung vorgenommen werden:
\begin{equation}
    C_{k,p}=c_k \cdot \frac{m_k}{n_k}=c_k \cdot M_k 
\end{equation}
$n_k$ bezeichnet an dieser Stelle die Stoffmenge. Die molare Masse $M$ kann aus Tabelle \ref{tab:eigenschaften} entnommen werden. 
Da alle notwendigen Werte in Tabelle \ref{tab:values} aufgenommen worden sind, kann $c_k$ für die drei Messkörper berechnet werden. %wir
\begin{table}
    \centering
    \caption{Spezifische Wärme und Wärmekapazität bei konstantem Druck der Messkörper.}
    \label{tab:c_k-Werte}
    \begin{tabular}{c c | S[table-format=4.2] S[table-format=3.1]}
        \toprule
        Stoff & Messung & $c_k[\si{\joule\per\g\per\kelvin}]$ & $C_{k,p}[\si{\joule\per\mol\per\kelvin}]$\\
        \midrule
        Zinn        & a) &   0.26 & 30.9   \\
                    & b) &   0.26 & 30.9   \\
                    & c) &   0.27 & 32.0   \\
        Aluminium   & a) &   0.92 & 24.8   \\
                    & b) &   0.78 & 21.1   \\
                    & c) &   0.96 & 25.9   \\
        Kupfer      & a) &   0.38 & 24.1   \\
        \bottomrule
    \end{tabular}
\end{table}

Mithilfe der in \cite{Versuchsanleitung} gegebenen Daten der Materialien lässt sich über den Zusammenhang \eqref{eqn:molwaerme}
die Molwärme aus den Messungen berechnen. 
Der Vollständigkeit halber seien hier ebenfalls die Werte für die hier verwendeten Stoffe aufgeführt.
\begin{table}
    \centering
    \caption{Eigenschaften der verwendeten Materialien.}
    \label{tab:eigenschaften}
    \begin{tabular}{c S[table-format=1.2] S[table-format=3.1] S[table-format=2.2] S[table-format=2.1] S[table-format=3.0]}
        \toprule
        Stoff & $\rho[\si{\gram\per\centi\meter\tothe{3}}]$ & $M[\si{\g\per\mol}]$ & $V_{0}[\num{e-6}\si{\meter\tothe{3}\mol\tothe{-1}}]$ & $\alpha[\num{e-6}\si{\kelvin\tothe{-1}}]$ & $\kappa[\num{e9}\si{\newton\per\meter\squared}]$ \\
        \midrule
        Zinn        &7.28   &118.7  &16.3   &27.0   &55     \\
        Aluminium   &2.70   &27.0   &10.0   &23.5   &75     \\
        Kupfer      &8.96   &63.5   &7.09   &16.8   &136    \\
        \bottomrule
    \end{tabular}
\end{table}

\begin{table}
    \centering
    \caption{Molwärme der Messkörper.}
    \label{tab:C_k-Werte}
    \begin{tabular}{c c | S[table-format=4.2]}
        \toprule
        Stoff & Messung & $C_{k,V}[\si{\joule\per\mol\per\kelvin}]$ \\
        \midrule
        Zinn        & a) & 30.9 \\
                    & b) & 30.9 \\
                    & c) & 32.0 \\
        Aluminium   & a) & 24.8 \\
                    & b) & 21.1 \\
                    & c) & 25.9 \\
        Kupfer      & a) & 24.1 \\
        \bottomrule
    \end{tabular}
\end{table}

\subsection{Fehlerrechnung}
Hier wird nun der Mittelwert $\bar{C_{k,V}}$ und der mittlere Fehler $\increment C_{k,V}$ der $N$ Messwerte für die Molwärme anhand 
\begin{gather}
\bar{C_{k,V}}=\frac{1}{N} \sum_{i=1}^N C_{k,V,i} \\
\increment C_{k,V} = \sqrt{\frac{1}{N-1}\sum_{i=1}^N (C_{k,V,i}-\bar{C_{k,V,i}})^2 }
\end{gather}
berechnet.
Aus offensichtlichen Gründen ergibt dies nur für mehrere Messwerte Sinn, weshalb die Molwärme von Kupfer an dieser Stelle nicht miteinbezogen wird.
\begin{gather*}
C_{k,V}=\bar{C_{k,V}} \pm \increment C_{k,V} \\
C_{k,V,\text{Zinn}} = \SI{31.3(6)e3}{\joule\per\mol\per\kelvin} \\
C_{k,V,\text{Alu}} = \SI{23.9(25)e3}{\joule\per\mol\per\kelvin} %+-2.5 soll angezeigt werden-->überprüfen!!
\end{gather*}