\section{Auswertung}
\label{sec:Auswertung}

\subsection{Vorbereitung}
Bei der Vorbereitung wird für den verwendeten Kondensator und die Spule eine Eigenfrequenz von 
$f_{\text{mess}}=\SI{33.0}{\kilo\hertz}$ bei einer Phasendifferenz von $\SI{90}{\degree}$ gemessen. 
Die Referenzwerte der Bauteile lauten $C=\SI{0.8015}{\nano\farad}$ und $L=\SI{32.351}{\milli\henry}$. 
Zudem weist die Spule eine Kapazität von $C_\text{Sp}=\SI{0.037}{\nano\farad}$ auf.
Der Kondensator mit regelbarer Kapazität wird so eingestellt, dass der zweite Schwingkreis die gleiche Eigenfrequenz besitzt.

Die Anzahl der Maxima/Minima innerhalb einer Schwebeperiode sind in der folgenden Tabelle \ref{tab:schwing_maxima} der regelbaren Kapazität $C_K$ gegenübergestellt.

\begin{table}
    \centering
    \caption{Anzahl Maxima der Schwebung.}
    \label{tab:schwing_maxima}
    \begin{tabular}{c c}
        \toprule
        {$C_K \:/\: \si{\nano\farad}$} & Schwingungsmaxima \\
        \midrule
        4.7  & \\ 
        6.8  & \\ 
        8.2  & \\ 
        10.0 & \\ 
        12.0 & \\ 
        \bottomrule
    \end{tabular}
\end{table}



\begin{table}
    \centering
    \caption{Resonanzfrequenzen verschiedener Kondensatoren.}
    \label{tab:resonanz}
    \begin{tabular}{c c c c}
        \toprule
        $f_- \:/\: \si{\kilo\hertz}$ & $f_+ \:/\: \si{\kilo\hertz}$ & $C_K \:/\: \si{\nano\farad}$ & $Amplitudenspannung \:/\: \si{\milli\volt}$ \\
        \midrule
        33.1 & 81.3 & 1.0  & 1830 \\
        33.1 & 61.1 & 2.2  & 1960 \\
        33.1 & 57.1 & 2.7  & 1883 \\
        33.1 & 48.7 & 4.7  & 2050 \\
        33.1 & 44.7 & 6.8  & 2160 \\
        33.1 & 42.8 & 8.2  & 1830 \\
        33.1 & 41.4 & 10.0 & 2030 \\
        33.1 & 40.2 & 12.0 & 2000 \\
        \bottomrule
    \end{tabular}
\end{table}

