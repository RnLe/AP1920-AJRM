\section{Diskussion}
\label{sec:Diskussion}

\subsection{Versuchsaufbau}
Der Aufbau des gesamten Versuchs ist eine wackelige Angelegenheit, weil der Laser und der Sensor sich höhentechnisch nicht aufeinander abstimmen lassen.
Stattdessen muss der Laser nach oben \textit{gekippt} und mit beispielsweise einem Stück Papier in der Halterung eingeklemmt werden.
Die Konstruktion ist daher anfällig für Stöße am Tisch oder der Führungsschiene selbst.

\subsection{Zuverlässigkeit der Vorhersagen und Messungen}
Die Versuchsanleitung hat einen kleinen Fehler in Gleichung (6), in der das $A_0$ fehlt. Wird dieser variable Faktor in der Ausgleichsrechnung nicht berücksichtigt,
funktioniert die komplette Rechnung nicht.
Erst ein Rückschluss über die Fourier-Transformierte $g(\xi)$ deckt diesen Fehler auf.
Das Auswerten über \texttt{curve\_fit} ist \textbf{extrem} fehleranfällig, da es viele Versuchsparameter gibt, die nicht exakt bestimmt sind.
Es \textbf{müssen} geeignete Startwerte übergeben werden, da ansonsten stark abweichende Parameterwerte gefunden werden.
Auch die Parameter wie etwa $c$ für den Dunkelstrom $I_d$ oder $x_0$ für das Intensitätsmaximum sollten nicht fest eingetragen werden, da die Covarianz ansonsten gegen
unrealistische Werte für $b$ konvergiert. Daher sind die Werte variabel gehalten, werden in der Funktion selbst aber mit Startwerten versehen. \\

Zudem ist der Dunkelstrom $I_d$ nicht genau bestimmbar, da die Wolkenlage starken Einfluss auf diese Messung hat. Der Versuch wurde zu Sommerbeginn in der Uhrzeit zwischen
14 und 17 Uhr durchgeführt und fand an einer nordseitigen Fensterfront statt.
Die Abweichungen des gemessenen Dunkelstroms $I_d = (\SI{0.63}{\micro\ampere}, \SI{0.7}{\micro\ampere}) $ mit dem errechneten $(c_1, c_2) = (\SI{0.38\pm0.12}{\micro\ampere}, \SI{0.81\pm0.01}{\micro\ampere})$
ist also auf die Wetterlage zurückzuführen und liegt mit Abweichungen von $0.5 < \dfrac{I_d}{c_i} < 2$ in einem sehr plausiblen Bereich.
Letztlich ist dieser Faktor lediglich eine Verschiebung des y-Achsen-Abschnitts und hat darüber hinaus keinen Einfluss auf die Ausgleichsrechnung beziehungsweise -kurve.
Dennoch sei erwähnt, dass die eigene Sitzposition oder das Vorbeilaufen an dem Versuch einen nennenswerten Einfluss auf den Dunkelstrom hat und ein Trend in den Messdaten sichtbar werden kann.
Von einem Umsetzen während der Messung oder einer Drehung der gesamten Führungsschiene ist daher abzuraten.

\subsection{Vergleich mit der Herstellerangabe}
Die Abweichungen der ermittelten Spaltenbreiten gegenüber den Herstellerangaben ist interessanterweise ein nahezu ganzzahliger Faktor mit relativen Abweichungen von
$(b_{par, 1}, b_{par, 2}) \approx (0.28\%, 0.5\%)$. Innerhalb dieser Abweichung lassen sich die Werte nähern zu $b_{par, 1} = \SI{0.02495 \pm 0.00007}{\milli\meter} \approx \SI{0.025}{\milli\meter}$ und
$b_{par, 2} = \SI{0.0752 \pm 0.0004}{\milli\meter} \approx \SI{0.075}{\milli\meter}$.
Auffällig ist, dass die Breite des Doppelspalts $b_{par, 2}$ ziemlich genau die tatsächliche Breite des Einzelspalts $b_1$ ist.
Eine Verwechslung der Messdaten wurde jedoch mehrfach geprüft und ausgeschlossen.
Darüber hinaus ist der ganzzahlige Zusammenhang 
\begin{equation*}
    \dfrac{b_{par, 1}}{b_1} = \dfrac{\SI{0.025}{\milli\meter}}{\SI{0.075}{\milli\meter}} = 3
\end{equation*}
und
\begin{equation*}
    \dfrac{b_{par, 2}}{b_2} = \dfrac{\SI{0.075}{\milli\meter}}{\SI{0.15}{\milli\meter}} = 2
\end{equation*}
sehr verdächtig. Da aber auch hier Faktoren wie falsches Ablesen der Daten oder unvollständige Überträge der Gleichungen in ein Python-Programm ausgeschlossen wurden,
werden diese Abweichungen als Zufall angenommen.
Genau genommen entstehen diese Abweichungen sehr wahrscheinlich durch Messungenauigkeiten und ungeeignete Näherungen innerhalb der Theorie selbst. \\

\subsection{Eignung der Theorie}
Die für die Rechnung verwendete Theorie der Fraunhofer-Beugung ist nur geeignet, wenn die Abstrahlwinkel auf der Spaltebene annähernd parallel sind. (s. Abb. \ref{fig:fresnelFraunhofer})
Denkbar ist, dass der Abstand $d$ mit $\SI{626.1}{\milli\meter}$ nicht groß genug ist, um eine Annäherung über $\sin{\varphi}$ zu ermöglichen.
Der in der Versuchsanleitung empfohlene Mindestabstand von $\SI{100}{\centi\meter} - \SI{120}{\centi\meter}$ kann auf der Führungsschiene nicht realisiert werden.
Mit $d = \SI{62.6}{\centi\meter}$ ist dieser Wert weit unterschritten. Zudem wurde keine Sammellinse verwendet, um diesen Abstandsverlust zu negieren.


\subsection{Fazit}
Der Versuch ist hinreichend durchführbar und die Ergebnisse bestätigen die klassischen Modelle der geometrischen Optik in sehr guter Näherung.
Trotz der veränderten Wolkenlage entsprechen die Messwerte der Erwartung.
Lediglich die Rahmenbedingung sind genau zu bestimmen und an manchen Stellen anzupassen. Die Form der gemessenen Beugungsfigur selbst jedoch besitzt keine sichtbaren Anomalien.
Für den Aufbau selbst würden vernünftige Stative für den Laser eine nennenswerte Verbesserung sein. Außerdem würde eine Sammellinse wesentlich stabilere Messungen ermöglichen.
Eine Schutzklappe beziehungsweise -vorrichtung für den Laser, oder ausdrückliche Sicherheitshinweise sind ratsam, da der Laser vorheizen muss, was bei Unwissen zu der Intuition führt,
den Kopf zu senken und in den Laser zu schauen. Bei einem Laser der Klasse 3 sollte das vermieden werden.