\section{Diskussion}
\label{sec:Diskussion}
Die Gesetze der geometrischen Optik lassen sich in diesem Versuch in sehr guter Näherung bestätigen. Die Theorie sagt das Verhalten der Strahlengänge und Messwerte genauestens vorher.

\subsection{Durchführbarkeit}
Der Versuch lässt sich problemlos durchführen. Die optischen Elemente lassen sich gut positionieren und das erzeugte Bild ist mit dem Auge bei entsprechenden Lichtverhältnissen
hinreichend scharfstellbar. Ledliglich die Chromatische Abberation, also farbliche Linsenfehler, sind schwierig einzustellen. Auch sind die Unterschiede von verschiedenfarbigem Licht
so klein, dass im $\sfrac{1}{10}\:\si{\milli\meter}$-Bereich gearbeitet werden müsste, was mit der bloßen Verschiebung der Reiter per Hand nicht umzusetzen ist.
Zudem können die Bilder bei solch einer Auflösung auch nicht scharfgestellt werden.

Darüber hinaus jedoch ist die Genauigkeit der Messapparatur für alle verwendeten Methoden vollkommen ausreichend.

\subsection{Genauigkeit der Ergebnisse}
Die relativen Messunsicherheiten sind für die Brennweiten mit meist unter $1\%$ Abweichung sehr gering und gut bestimmt. So hat die Brennweite der \textbf{ersten Methode} nur eine relative
Unsicherheit von $0.4\%$. Abbildung \ref{fig:fokus} zeigt als gemeinsamen Schnittpunkt aller Geraden von $g_i$ nach $b_i$ die Brennweite der Sammellinse an. Der Unterschied 
zur Herstellerangabe ist mit $\Delta f = \SI{1.31}{\centi\meter}$ deutlich größer als die Messunsicherheit. Die Abweichung kann verschiedene systematische Ursachen haben, wie etwa
Skalierungsfehler der Messschiene, ein kleiner Fehler bei der Positionsbestimmung des Gegenstandes oder eine tatsächliche Abweichung der Brennweite. \\

In der \textbf{Methode nach Bessel} kann von den Messwerten die Genauigkeit des Scharfstellens abgeleitet werden. Für beide Fokuspunkte gilt $b_1=g_2$ und $b_2=g_1$.
In Tabelle \ref{tab:mess2} sind diese Werte gegenübergestellt. Es gibt jedoch kleine Unterschiede in der Abstandsbestimmung. Anhand dieser Varianz kann auf die Messgenauigkeit geschlossen
werden. Werden die Differenzen $d_1 = |b_1 - g_2|$ und $d_2 = |b_2 - g_1|$ gebildet, kann daraus bestimmt werden, auf wie viel $\si{\centi\meter}$ genau die Versuchsteilnehmer den schärfsten
Punkt am Schirm einstellen können. In diesem Falle ist die Einstellgenauigkeit $\bar{d} \approx \SI{3.6}{\milli\meter}$.

Die Brennweite ist auch bei dieser Methode mit einer Abweichung von $0.5\%$ sehr genau bestimmt. Zudem beträgt die Differenz zur Herstellerangabe nur $\Delta f = \SI{1}{\milli\meter}$,
was einer relativen Unsicherheit von $1\%$ entspricht. \\

Die \textbf{Methode nach Abbe} ist in Summe etwas ungenauer, was an den verwendeten Linsen liegen kann. Die Kombination aus Zerstreuungs- und Sammellinse ergibt einen multiplikativen Faktor des Fehlers
für Abweichungen der Realbrennweite vom Herstellerwert. Dieser Fehler skaliert zusätzlich linear mit dem Abstand der beiden Linsen voneinander (je weiter ein abweichender
Strahlengang von der Streulinse verläuft, bis er von der Sammellinse eingefangen und fokussiert wird, desto größer wird die Abweichung).

Nichtsdestotrotz sind die berechneten Werte plausibel und hinreichend genau, um die Theorie zu bestätigen. Zwar kann die Gültigkeit der Hauptebenenpositionen $h$ und $h'$ nicht geprüft werden,
dafür kann jedoch wieder die Brennweite verglichen werden. Mit einer Gesamtabweichung von $\Delta f \approx \SI{1.2}{\centi\meter}$ ist die Abweichung der Linsenkombination
nur um $6\%$ verschieden. Durch die Abhängigkeit von mehreren Größen sind die Messunsicherheiten höher als bei den anderen Methoden. Insbesondere die Bestimmung der
Hauptebenenpositionen ist deutlich ungenauer mit relativen Werten von $h_{rel} \approx 35\%$ und $h_{rel}' \approx 9.8\%$. In den Abbildungen \ref{fig:lin1} und \ref{fig:lin2}
sind die Messwerte und die entsprechenden Ausgleichsgeraden aufgezeichnet. Sie zeigen keine großen Abweichungen oder Ausreißer.

\subsection{Fazit}
Der Versuch ist ein qualitativer Erfolg. Die einzelnen Theorien lassen sich gezielt überprüfen und ohne Ausnahmen bestätigen.
Die Messgenauigkeit ist technisch bedingt zwar eingeschränkt, für die hier verwendeten Methoden aber vollkommen ausreichend.
Abweichungen lassen sich durch systematische Fehler erklären, die im einzelnen aber aufwändig zu bestimmen sind. In Bezug auf die Realwerte jedoch ist die Messgenauigkeit sehr hoch.