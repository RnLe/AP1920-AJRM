\section{Theorie}
\label{sec:Theorie}

\subsection{Radioaktiver Zerfall durch Kernrekationen mit Neutronen}

Zur Herstelellung instabiler und damit radioaktiver Nuklide werden Atome eines stabilen Elements mit Neutronen beschossen. 
Der Vorteil von Neutronen ist, dass sie nicht geladen sind und somit nicht die Coulomb-Barriere überwinden müssen, die nur 
für geladene Teilchen einen bemerkenswerten Unterschied macht. 
Der Kern nimmt bei Reaktion das Neutron auf und die kinetische Energie wird fast vollständig auf alle vorhandenen Nukelonen 
verteilt. Der Kern befindet sich damit im angeregten Zustand -- für eine Zeitspanne von etwa $\SI{e-16}{\second}$. 
Dann geht er nämlich unter Emission eines $\gamma$-Quants wieder in seinen Grundzustand über und der vorher stabile 
Atomkern hat sich in einen instabilen umgewandelt durch das gestiegene Verhältnis von Neutronenzahl zur Kernladungszahl. 
An dieser Stelle beginnt der radioaktive Zerfall, der hier untersucht werden soll. 
Ein Neutron des Nukleon wandelt sich unter Emission eines Elektrons und eines Elektron-Antineutrinos in ein Proton um, 
ein Kern einer um eins höheren Ordnungszahl entsteht. 
Die Reaktion sieht im Allgemeinen so
\begin{equation*}
    \ce{^m_zA + ^1_0n -> ^{m+1}_zA* -> ^{m+1}_zA -> ^{m+1}_{z+1}A + e- + \bar{\symup{\nu}}_e}
\end{equation*}
aus, wobei das Sternchen * den angeregten Zustand des Kerns indiziert.
Überschüssige Energie beziehungsweise Massedifferenzen werden in Bewegungsenergie der Kerne und die Bildung Antineutrinos $\bar{\symup{\nu}}_e}$ 
gemäß der Masse-Energie-Beziehung nach Einstein $\Delta E=\Delta m c^2$ ineinander umgewandelt.

\subsection{Wirkungsquerschnitt}

Der Wirkungsquerschnitt $\sigma$ ist ein Maß dafür, wie wahrscheinlich ein Stoß von zwei Teilchen ist. 
Eine andere Umschreibung für den Wirkungsquerschnitt wäre, dass sie die Fläche darstellt, die ein Nukleon haben müsste, 
damit der Beschuss eines Neutrons auf die Fläche jedes Mal zu einem Einfang führt. 
Ist die Geschwindigkeit des kollidierenden Teilchens eher groß, können Interferenz- und Streueffekte vernachlässigt werden
und der Wirkungsquerschnitt kann anschaulich als die Fläche angesehen werden, die anteilmäßig von den Nukleonen eingenommen wird. 
Je zahlreicher und größer die Nukleonen sind, umso mehr nehmen sie prozentual an Fläche und somit an Stoßfläche ein. 
Ist hingegen die Geschwindigkeit eher gering, sorgen Streueffekte des Teilchens an den Nukleonen dafür, dass die Wahrscheinlichkeit 
eines Stoßes sehr viel größer wird, als es die geometrische Betrachtung glauben machen würde. 

Diese Unterscheidung nach der Geschwindigkeit $v$ kann auf die Wellenlänge $\lambda$ mithilfe der De-Broglie-Beziehung
\begin{equation*}
    \lambda = \frac{\symup{h}}{m_\text{n}v}
\end{equation*}
übertragen werden, wobei $m_\text{n}$ die Masse eines Neutrons ist und das Planck'sche Wirkungsquantum durch ${\symup{h}=\SI{6.62607004e-34}{\joule\second}}$ \cite{scipy} gegeben ist. 
So kann jedem Neutron mit einer bestimmten Wellenlänge eine Geschwindigkeit zugeordnet werden. 
Zur Einordnung der Größe wird hier der Kernradius verwendet; ist die Wellenlänge sehr viel kleiner als der Kernradius ($\approx \SI{e-12}{\centi\meter}$), 
können die Streueffekte vernächlässigt werden und die geometrische Betrachtung des Wirkungsquerschnitts genügt. 

Die Absorption wird genau dann maximal -- Stichwort Resonanzabsorption --, wenn die Energie des Neutrons gerade der Energiedifferenz 
der zwei Niveaus des Atomkerns entspricht. Mithilfe einer harmonischen Annäherung und mit quantenmechanischen Annahmen begründet 
sich die sogenannte Breit-Wigner-Formel; diese sagt eine umgekehrte Proportionalität des Wirkungsquerschnitts und damit der 
Absorption der Strahlung zur Energie beziehungsweise Geschwindigkeit des Neutrons voraus
\begin{equation*}
    \sigma \propto \frac{1}{\sqrt{E}} \propto \frac{1}{v} \,,
\end{equation*}
was sich mit den vorangestellten Überlegungen deckt. 
Für eine hohe Reaktanz sollte die Energie der Neutronen somit eher gering gehalten werden.

\subsection{Erzeugung niederenergetischer Neutronen}

Neutronen haben eine begrenzte Lebensdauer, weshalb sie ebenfalls direkt vor der Messung hergestellt werden müssen. 
Die Reaktion 
\begin{equation*}
    \ce{^9_4Be + ^4_2 \alpha -> ^{12}_6C+^1_0n}
\end{equation*}
wird hier verwendet. 
Zerfallende $\ce{^{226}Ra$-Kerne emittieren die $\alpha$-Teilchen, die auf $\ce{^9Be}$-Kerne gelenkt werden. 
Die darauffolgende Kernrekation sorgt für die Produktion von freien Neutronen und die Umwandlung der Nuklide in $\ce{^{12}C$-Kerne. 

Da die Neutronen an dem Punkt noch eine Energie von maximal $\SI{13.7}{\mega\electronvolt}$ aufweisen, müssen sie 
entsprechend gebremst werden, um die für das Experiment günstige niedrige Geschwindigkeit zu erreichen. 
Deshalb wird die Neutronenquelle mit Paraffin ummantelt. Die darin enthaltenen leichten Wasserstoffkerne sorgen für einen 
möglichst großen Energieübertrag, wenn die Neutronen elastisch mit ihnen zusammenstoßen. 
Die Stöße finden solange statt, bis sich die Energie gleichmäßg auf die Teilchen verteilt und die Neutronen damit
die mittlere Energie der sie umgebenden Moleküle annehmen. Dies entspricht bei einer Temperatur von ${T=\SI{}{}}$ 
einer Energie von $E=\SI{}$