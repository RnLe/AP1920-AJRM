\section{Zielsetzung}
\label{sec:Zielsetzung}

Das Ziel des folgenden Experiments ist die Bestimmung der Halbwertszeiten einiger radioaktiver Stoffe, die sich zusätzlich 
in ihren Eigenschaften bezüglich ihres Zerfallsgeschehen unterscheiden. 
Als die Halbwertszeit $\tau$ wird die Zeit verstanden, in der die Hälfte einer hinreichend großen Stoffmenge zerfällt. 
Im Umkehrschluss bedeutet eine kleinere Halbwertszeit eine größere Zerfallswahrscheinlichkeit und einen instabileren Kern. 
Sie könnnen sehr stark in ihren Größenordnungen variieren, die sich in vielen Zehnerpotenzen unterscheiden. 

Hier werden ausschließlich Stoffe betrachtet, die unter Aussendung von $\symup{\beta}^-$-Strahlung, also Elektronen, zerfallen. 

Um dies zu bewerkstelligen, wird in Kapitel \ref{sec:Theorie} erläutert, inwiefern radioaktive Stoffe hergestellt werden können, 
deren instabile Kerne dann zu Beginn der Durchführung anfangen sollen zu zerfallen, und wie sinnvoll die Strahlungsmenge gemessen werden kann, mit deren Hilfe Rückschlüsse 
auf den Anteil der zerfallenen Menge gezogen werden können. 

Die Messung wird exemplarisch an zwei verschiedenen Stoffen durchgeführt. 